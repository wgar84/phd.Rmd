\documentclass[11pt,]{article}
\usepackage{lmodern}
\usepackage{amssymb,amsmath}
\usepackage{ifxetex,ifluatex}
\usepackage{fixltx2e} % provides \textsubscript
\ifnum 0\ifxetex 1\fi\ifluatex 1\fi=0 % if pdftex
  \usepackage[T1]{fontenc}
  \usepackage[utf8]{inputenc}
\else % if luatex or xelatex
  \ifxetex
    \usepackage{mathspec}
    \usepackage{xltxtra,xunicode}
  \else
    \usepackage{fontspec}
  \fi
  \defaultfontfeatures{Mapping=tex-text,Scale=MatchLowercase}
  \newcommand{\euro}{€}
\fi
% use upquote if available, for straight quotes in verbatim environments
\IfFileExists{upquote.sty}{\usepackage{upquote}}{}
% use microtype if available
\IfFileExists{microtype.sty}{%
\usepackage{microtype}
\UseMicrotypeSet[protrusion]{basicmath} % disable protrusion for tt fonts
}{}
\usepackage[margin=1in]{geometry}
\usepackage{graphicx}
\makeatletter
\def\maxwidth{\ifdim\Gin@nat@width>\linewidth\linewidth\else\Gin@nat@width\fi}
\def\maxheight{\ifdim\Gin@nat@height>\textheight\textheight\else\Gin@nat@height\fi}
\makeatother
% Scale images if necessary, so that they will not overflow the page
% margins by default, and it is still possible to overwrite the defaults
% using explicit options in \includegraphics[width, height, ...]{}
\setkeys{Gin}{width=\maxwidth,height=\maxheight,keepaspectratio}
\ifxetex
  \usepackage[setpagesize=false, % page size defined by xetex
              unicode=false, % unicode breaks when used with xetex
              xetex]{hyperref}
\else
  \usepackage[unicode=true]{hyperref}
\fi
\hypersetup{breaklinks=true,
            bookmarks=true,
            pdfauthor={},
            pdftitle={Type I and II Error Rates in Modularity Hypothesis Testing},
            colorlinks=true,
            citecolor=blue,
            urlcolor=blue,
            linkcolor=magenta,
            pdfborder={0 0 0}}
\urlstyle{same}  % don't use monospace font for urls
\setlength{\parindent}{0pt}
\setlength{\parskip}{6pt plus 2pt minus 1pt}
\setlength{\emergencystretch}{3em}  % prevent overfull lines
\setcounter{secnumdepth}{0}

%%% Use protect on footnotes to avoid problems with footnotes in titles
\let\rmarkdownfootnote\footnote%
\def\footnote{\protect\rmarkdownfootnote}

%%% Change title format to be more compact
\usepackage{titling}

% Create subtitle command for use in maketitle
\newcommand{\subtitle}[1]{
  \posttitle{
    \begin{center}\large#1\end{center}
    }
}

\setlength{\droptitle}{-2em}
  \title{Type I and II Error Rates in Modularity Hypothesis Testing}
  \pretitle{\vspace{\droptitle}\centering\huge}
  \posttitle{\par}
  \author{}
  \preauthor{}\postauthor{}
  \date{}
  \predate{}\postdate{}

%--- USEPACKAGES ---
\usepackage[utf8]{inputenc}
\usepackage{natbib}
\usepackage[brazilian,english]{babel}
\usepackage{hyperref}
\usepackage{subfigure, epsfig}
\usepackage{ae}
\usepackage{aecompl}
\usepackage{booktabs}
\usepackage[T1]{fontenc}
\usepackage{graphicx,wrapfig} % para incluir figuras
\usepackage{amsfonts, amssymb,amsthm, amsmath, amscd} % pacote AMS
\usepackage{color, bbm, multicol}
\usepackage{verbatim, listings, booktabs}
\usepackage{fancyhdr} % FANCYHEADER
\usepackage{setspace}
\usepackage{times} % bookman,palatino,courier,times FONTES
\usepackage{lineno}
\usepackage{url}
\usepackage{rotating}
\usepackage{longtable}
% \usepackage{rotfloat}
% \usepackage{appendix}
\usepackage{mathptmx}
% \usepackage{cmbright}
% \usepackage[adobe-utopia]{mathdesign}
\usepackage[flushleft]{threeparttable}
\usepackage{multirow}
% \usepackage{float}
\usepackage {tocvsec2} % controlar profundidade de table of contents
\setcounter {secnumdepth}{0}
\usepackage {caption}
\usepackage {floatrow}
\floatsetup[table]{capposition=top}

\usepackage {xr}
\externaldocument[s]{supplemental}

\selectlanguage{english}

\hypersetup{colorlinks=false}


\begin{document}

\maketitle


\linenumbers
\modulolinenumbers[2]

\onehalfspacing

\section{Introduction}\label{introduction}

\section{Methods}\label{methods}

\subsection{Sample}\label{sample}

The database used here (\autoref{tab:modcomp_otu}) consists of 21
species, distributed across all taxonomic denominations of Anthropoidea
above the genus level. These species were selected from a broader
database (Marroig \& Cheverud, 2001; Oliveira \emph{et al.}, 2009) in
order to reduce the effects of low sample sizes over estimates of
modularity patterns. Individuals in our sample are represented by 36
registered landmarks, using either a Polhemus 3Draw or a Microscribe 3DS
for Platyrrhini and Catarrhini, respectively. Twenty-two unique
landmarks represent each individual (\autoref{fig:landmarks},
\autoref{tab:lms}), since 14 of the 36 registered landmarks are
bilaterally symmetrical. For more details on landmark registration, see
Marroig \& Cheverud (2001) and Oliveira et al. (2009).

\begin{table}[t]
  \centering
  \begin{threeparttable}
    \caption{Twenty-one species used in the present work, along with sample sizes and linear models adjusted. \label{tab:modcomp_otu}}
    \begin{tabular}{lccr}
        \toprule
        Species & Group$^a$ & $n$ & Model$^b$ \\ 
      \midrule
        \emph{Alouatta belzebul} & P & 109 & X \\ 
        \emph{Ateles geoffroyi} & P & 78 & - \\ 
        \emph{Cacajao calvus} & P & 48 & S + X \\ 
        \emph{Callicebus moloch} & P & 93 & X \\ 
        \emph{Callithrix kuhlii} & P & 129 & - \\ 
        \emph{Cebus apella} & P & 110 & X \\ 
        \emph{Cercopithecus ascanius} & C & 61 & X \\ 
        \emph{Chiropotes chiropotes} & P & 56 & X \\ 
        \emph{Chlorocebus pygerythrus} & C & 110 & X \\ 
        \emph{Colobus guereza} & C & 140 & X \\ 
        \emph{Gorilla gorilla} & C & 115 & X \\ 
        \emph{Homo sapiens} & C & 160 & S * X \\ 
        \emph{Hylobates lar} & C & 66 & X \\ 
        \emph{Macaca fascicularis} & C & 69 & X \\ 
        \emph{Pan troglodytes} & C & 61 & X \\ 
        \emph{Papio anubis} & C & 46 & X \\ 
        \emph{Piliocolobus foai} & C & 83 & X \\ 
        \emph{Pithecia pithecia} & P & 69 & S + X \\ 
        \emph{Procolobus verus} & C & 88 & X \\ 
        \emph{Saguinus midas} & P & 50 & S \\ 
        \emph{Saimiri sciureus} & P & 87 & X \\ 
      \bottomrule
    \end{tabular}
      \begin{tablenotes}
        \footnotesize
        {
        \item[$a$] C: Catarrhini; P: Platyrrhini
        \item[$b$] S: subspecies/population; X: sex.
        }
      \end{tablenotes}
    \end{threeparttable}
  \end{table}


For each OTU, we estimated phenotypic covariance and correlation
matrices for three different types of variables: tangent space
residuals, estimated from a Procrustes superimposition for the entire
sample, using the set of landmarks described on both \autoref{tab:lms}
and \autoref{fig:landmarks}; interlandmark distances, as described in
\autoref{tab:dist}; and local shape variables (Márquez \emph{et al.},
2012), which are measurements of infinitesimal log volume
transformations between each sample unit and a reference (mean) shape.
These transformations were calculated at 38 points, corresponding to the
locations of the mipoints between pairs of landmarks used to calculate
interlandmark distances, in order to produce a dataset that represents
shape (i.e., form without isometric variation; Bookstein, 1991; Zelditch
\emph{et al.}, 2004) while retaining the same overall properties of the
interlandmark distance dataset.

For each dataset, we estimated covariance and correlation matrices after
removing fixed effects of little interest in the present context, such
as sexual dimorphism, for example. For interlandmark distances and local
shape variables, these effects were removed through a multivariate
linear model adjusted for each species, according to
\autoref{tab:modcomp_otu}; for Procrustes residuals, the same effects
were removed by centering all group means to each species' mean shape,
since the loss of degrees of freedom imposed by the GPA prohibits the
use of a full multivariate linear model over this kind of data to
removed fixed effects.

In order to consider the effects of size variation over modularity
patterns, we used a different procedure to remove the influence of size,
according to the different properties of each type of variable.

De maneira a considerar o efeito da variação de tamanho sobre padrões de
modularidade, utilizamos um procedimento de remoção de tamanho sobre
matrizes de covaripara cada tipo de variável: para distâncias
euclidianas, utilizamos o procedimento delineado em Marroig \& Cheverud
(2004), baseado em Bookstein et al. (1985). No caso dos resíduos de
Procrustes e variáveis locais de forma, apesar do efeito de escala ser
removido, ao normalizar cada indivíduo por seu tamanho do centróide
associado, o efeito da alometria se encontra presente. De maneira a
remover este efeito, utilizamos o procedimento delineado por
Mitteroecker et al. (2004). Desta forma, obtivemos matrizes de
covariância que representam os padrões de associação entre variáveis
livres dos efeitos integradores de tamanho.

\subsection{Testes de Hipótese de Modularidade \emph{a
priori}}\label{testes-de-hipotese-de-modularidade-a-priori}

A partir dos seis conjuntos de matrizes de covariância, testamos as
hipóteses de associação entre marcos anatômicos delimitadas na
\autoref{tab:lms} para matrizes derivadas de resíduos de Procrustes;
para distâncias euclidianas e variáveis locais de forma, utilizamos as
hipóteses conforme a delimitação exposta na \autoref{tab:dist}, visto
que estes dois conjuntos de dados possuem, por construção, o mesmo
número de caracteres posicionados de modo similar sobre o crânio.

Para cada hipótese de associação testada, calculamos o índice AVG (Porto
\emph{et al.}, 2013), da forma
\[ AVGi = \frac {\bar{\rho}_{+} - \bar{\rho}_{-}} {ICV} \] onde
$\bar{\rho}_{+}$ representa a média de correlações entre caracteres
pertencentes àquela hipótese de associação, $\bar{\rho}_{-}$ representa
a média de correlações do conjunto complementar a este, ou seja,
considerando tanto correlações entre aqueles caracteres que não
pertencem ao conjunto de associações considerada quanto correlações
entre os dois conjuntos; $ICV$ representa o coeficiente de variação dos
autovalores da matriz de covariância associada (Shirai \& Marroig,
2010). No caso da distinção entre caracteres Faciais e Neurocraniais,
calculamos também o contraste entre correlações contidas na união entre
os dois conjuntos e o conjunto complementar; neste caso, a hipótese
associada é denominada Neuroface.

Também calculamos o índice RV (Klingenberg, 2009), da forma \[
RV = \frac{tr(\mathbf{S}_{12}\mathbf{S}_{21})}{\sqrt{tr(\mathbf{S}_1 \mathbf{S}_1)tr(\mathbf{S}_2 \mathbf{S}_2)}}
\] onde $\mathbf{S}_{12} = \mathbf{S}'_{21}$ representa o bloco de
covariâncias entre o conjunto considerado e seu complemento, e
$\mathbf{S}_1$ e $\mathbf{S}_2$ representam os blocos de covariância
dentre caracteres do conjunto considerado e dentre caracteres do
conjunto complementar, respectivamente; a operação $tr$ consiste na soma
dos elementos da diagonal da matriz ($tr \mathbf{A} = \sum_i a_{ii}$).
Para o caso da distinção entre caracteres Faciais e Neurocraniais, visto
que estes dois conjuntos são complementares e contém conjuntamente todos
os caracteres, valores do índice RV calculados para os dois conjuntos
são idênticos; dessa forma, apenas um valor, denominado Neuroface é
reportado

De modo a testar a hipótese de que cada um destes conjuntos de
caracteres representam módulos variacionais, utilizamos um procedimento
de aleatorização sobre cada um dos conjuntos, de maneira que, para cada
matriz e hipótese, 1000 aleatorizações são geradas, e índices AVG e RV
são calculados para cada iteração. O valor do índice calculado sobre o
particionamento real é comparado às 1000 iterações: no caso do índice
AVG, considera-se válida a particão com índices maiores que a
distribuição de índices gerados pelo particionamento aleatório, dado um
certo valor de significância; no caso do índice RV, considera-se válida
aquela partição com índices menores que aqueles gerados ao acaso. Para o
caso particular dos resíduos de Procrustes, o procedimento de
aleatorização mantém unidas as estruturas de covariância associadas às
coordenadas de um mesmo marco anatômico, conforme Polly (2005).

\subsection{Simulações}\label{simulacoes}

De maneira a avaliar as propriedades estatísticas dos dois índices e do
procedimento de aleatorização, construímos matrizes de correlação
teóricas na forma \[
\mathbf{C}_{s} =
\begin{bmatrix}
\mathbf{W}_1 & \mathbf{B} \\
\mathbf{B} & \mathbf{W}_2 \\
\end{bmatrix}
\] onde $\mathbf{W}_1$ e $\mathbf{W}_2$ representam blocos de correlação
associados a dois conjuntos de associação entre caracteres e
$\mathbf{B}$ representa o bloco de correlações entre elementos dos dois
conjuntos; esta forma mimetiza o particionamento da matriz em blocos
necessário ao cômputo do índice RV.

A partir dos seis conjuntos de matrizes de correlação empíricas
utilizadas na seção anterior, obtivemos distribuições de correlações
médias dentre e entre as hipóteses de associação de caracteres
consideradas (\autoref{fig:cor_dist}). De maneira a construir um
conjunto de matrizes $\mathbf{C}_{s}$ representativas das distribuições
de correlação encontradas nos diferentes tipos de variáveis, sorteamos
valores das distribuições de correlação média obtidas para cada conjunto
de matrizes e utilizamos os valores sorteados para construir matrizes
$\mathbf{C}_{s}$. Assim, três valores são necessários para a construção
de uma destas matrizes: dois valores sorteados da distribuição de
correlação média dentre elementos de um mesmo conjunto, que irão
preencher as correlações dos blocos $\mathbf{W}_1$ e $\mathbf{W}_2$, e
um valor derivado da correlação dentre elementos de conjuntos distintos,
que irá preencher o bloco $\mathbf{B}$. Dessa forma, para quatro
caracteres divididos em dois conjuntos de mesmo tamanho, se sorteamos os
valores $0.5$ e $0.3$ da primeira distribuição e $-0.1$ da segunda, a
matriz $\mathbf{C}_{s}$ associada será da forma \[
\mathbf{C}_s =
\begin{bmatrix}
1 & 0.5 & -0.1 & -0.1 \\
0.5 & 1 & -0.1 & -0.1 \\
-0.1 & -0.1 & 1 & 0.3 \\
-0.1 & -0.1 & 0.3 & 1 \\
\end{bmatrix}
\] de modo que todos as correlações associadas a um determinado bloco
serão idênticas ao valor sorteado correspondente. Desta maneira,
abstraímos os diferentes tipos de variáveis em distribuições de
correlações distintas, e construímos matrizes representativas destas
diferentes situações.

\begin{figure}[htbp]
\centering
\includegraphics{Output/modcomp_files/figure-latex/cor_dist-1.pdf}
\caption{Distribution of correlation derived from empirical correlation
matrices. \label{fig:cor_dist}}
\end{figure}

Para cada par de distribuições de correlação obtidas a partir dos seis
tipos de variáveis, construímos 1000 matrizes de tamanho $40 \times 40$
desta forma; apenas matrizes de correlação positivas-definidas foram
consideradas, resorteando valores quando a matriz construída não se
adequava a este pré-requisito. Este procedimento garante que, ainda que
estas matrizes possuam estruturas de correlação simplificadas, elas são
de fato matrizes de correlação que poderiam ser estimadas em uma amostra
real. Para cada matriz, o número de caracteres pertecentes a cada
conjunto foi determinado arbitrariamente, desde que nenhuma partição em
nenhuma matriz construída tenha menos do que cinco caracteres.

Dessa forma, construímos matrizes de correlação associadas a cada um dos
tipos de variáveis cuja estrutura de particionamento é conhecida. Para
construir matrizes cuja estrutura não é conhecida, apenas tomamos cada
uma destas matrizes e embaralhamos as linhas e colunas desta matriz,
gerando assim para cada matriz $\mathbf{C}_s$ uma matriz $\mathbf{C}_r$
associada. A partir destes pares de matrizes, sorteamos amostras de 20,
40, 60, 80 e 100 indíviduos, e re-estimamos as matrizes de correlação
associadas, considerando dessa forma a incerteza derivada de amostragem.
A partir das matrizes estimadas, testamos a hipótese de associação que
gerou a matriz teórica da mesma maneira descrita na seção anterior,
considerando a situação onde testamos, no caso do índice AVG, a
distinção entre caracteres integrados e não integrados pelas duas
partições, de forma similar a hipótese da Neuroface.

Para o caso de amostras geradas a partir de uma matriz $\mathbf{C}_r$
(aleatórias), o teste é representativo de uma situação onde sabemos que
a hipótese nula do teste é verdadeira, em ambos os casos (RV ou AVGi),
visto que a matriz foi gerada a partir de uma aleatorização daquela
hipótese que está sendo testada. Assim, buscamos neste caso estimar as
taxas de erro tipo I do teste, isto é, a proporção de casos onde a
hipótese nula é rejeitada mesmo sendo verdadeira, dado um certo nível de
significância. Neste caso, esperamos que as taxas de erro I acompanhem o
nível de significância estabelecido de forma direta, isto é, seguindo
uma relação de identidade.

No caso das amostras geradas a partir das matrizes $\mathbf{C}_s$
(estruturadas), este teste é representativo de uma situação em que
sabemos que a hipótese nula do teste é falsa, isto é, de que aquela
partição testada de fato representa a estrutura da matriz de correlação
frente a partições geradas ao acaso. Buscamos portanto estimar as taxas
de erro tipo II dos testes, isto é, a proporção de casos em que a
hipótese nula não é rejeitada, mesmo sendo falsa, dado um certo nível de
significância; nesse caso, é possível parametrizar o problema utilizando
o poder associado ao teste, que é apenas complementar à taxa de erro
tipo II, representando a proporção de casos em que $H_0$ é rejeitada por
ser de fato falsa. No caso, esperamos que os testes sigam uma relação
crescente com o nível de significância, ainda que, ao contrário das
taxas de erro tipo I, esperamos que um aumento pequeno em $P(\alpha)$
implique em um aumento grande no poder do teste.

Neste caso, o tamanho do efeito é importante, visto que os valores de
correlação sorteados podem gerar um estrutura de correlação que não é
detectada pelo teste simplesmente devido a uma diferença pequena entre
correlações dentro e entre partições. Para contornar este problema,
utilizamos a correlação entre partições ao quadrado ($b^2$), uma medida
que não é particularmente associada a nenhuma das métricas testadas e
que deve ser suficiente para descrever e corrigir o problema. Esperamos
portanto que, para valores de $b^2$ pequenos, o poder de ambos testes
(AVGi e RV) será maior do que em casos onde este valor é maior.

\section{Resultados}\label{resultados}

\subsection{Testes Empíricos}\label{testes-empiricos}

Em relação às hipóteses locais (Oral, Nasal, Zigomático, Órbita, Base e
Abóbada), estimativas de AVGi (\autoref{fig:MI_Func}) e RV
(\autoref{fig:RV_Func}) demonstram que os testes discordam a respeito
das regiões detectadas como hipóteses válidas. No caso particular dos
testes sobre resíduos de Procrustes, ambos os testes detectam como
válidas um número reduzido de hipóteses, em regiões e espécies díspares.
Desconsiderando resíduos de Procrustes, o teste associado ao índice AVG
detecta partições nos dois tipos de variáveis de forma consistente entre
as diferentes espécies, ainda que as regiões detectadas em cada tipo de
variável sejam diferentes. No caso do índice RV, ainda que um número
maior de hipótese sejam detectados em relação aos resíduos de
Procrustes, não há consistência entre as partições detectadas em cada
tipo de variável dentre as diferentes espécies. O mesmo padrão pode ser
observado para a distinção entre Face e Neurocrânio (Figuras
\ref{fig:MI_Dev} e \ref{fig:RV_Dev}).

\begin{figure}[htbp]
\centering
\includegraphics{Output/modcomp_files/figure-latex/MI_Func-1.pdf}
\caption{AVG Index values for local functional/developmental hypotheses.
Circles indicate whether a hypothesis is recognized in a given OTU,
according to the legend \label{fig:MI_Func}}
\end{figure}

\begin{figure}[htbp]
\centering
\includegraphics{Output/modcomp_files/figure-latex/RV_Func-1.pdf}
\caption{RV Coefficient values for local functional/developmental
hypotheses. Circles indicate whether a hypothesis is recognized in a
given OTU, according to the legend \label{fig:RV_Func}}
\end{figure}

\begin{figure}[htbp]
\centering
\includegraphics{Output/modcomp_files/figure-latex/MI_Dev-1.pdf}
\caption{AVG Index values for global developmental hypotheses. Circles
indicate whether a hypothesis is recognized in a given OTU, according to
the legend \label{fig:MI_Dev}}
\end{figure}

\begin{figure}[htbp]
\centering
\includegraphics{Output/modcomp_files/figure-latex/RV_Dev-1.pdf}
\caption{RV Coefficient values for global developmental hypotheses.
Circles indicate whether a hypothesis is recognized in a given OTU,
according to the legend \label{fig:RV_Dev}}
\end{figure}

\subsection{Simulações}\label{simulacoes-1}

A distribuição de valores de AVGi e RV recuperadas da simulação
(\autoref{fig:stat_dist_sim}) indica que, no caso de variáveis locais de
forma, as distribuições de valores obtidos a partir de matrizes
$\mathbf{C}_s$ e $\mathbf{C}_r$ são disjuntas, enquanto nos demais tipos
de variáveis, há uma sobreposição razoável das distribuições para os
dois tipos de matrizes, ainda que essa sobreposição seja maior no caso
dos resíduos de Procustes e também no caso do índice RV. A distribuição
de índices AVG para matrizes $\mathbf{C}_r$ é bastante estreita e
centrada no valor nulo, enquanto que no caso do índice RV, essa
distribuição é mais larga.

\begin{figure}[htbp]
\centering
\includegraphics{Output/modcomp_files/figure-latex/stat_dist_sim-1.pdf}
\caption{Distribution of AVG Index and RV Coefficient for simulated
correlation matrices. \label{fig:stat_dist_sim}}
\end{figure}

Considerando a relação entre níveis de significância e taxas de erro
tipo I (\autoref{fig:type1}), esta relação, para os diferentes tipos de
variáveis e tamanhos amostrais, segue muito próxima à reta-identidade,
de maneira que ambos os testes baseados no dois índices apresentam um
comportamento próximo do ideal, exceto no caso das matrizes derivadas
das distribuições de correlação de variáveis locais de forma sem remoção
de tamanho, onde taxas de erro tipo I são menores que os níveis de
significância atribuídos (painel superior à esquerda da
\autoref{fig:type1}).

\begin{figure}[htbp]
\centering
\includegraphics{Output/modcomp_files/figure-latex/type1-1.pdf}
\caption{Type I error rates as a function of the chosen significance
level regarding tests for modular structure applied on random
($\mathbf{C}_r$) matrices. \label{fig:type1}}
\end{figure}

Quando consideramos o poder associado aos testes, é possível observar
que existem diferenças substanciais neste parâmetro dentre os diferentes
tipos de variáveis. No caso das distâncias euclidianas
(\autoref{fig:type2_ed}), é possiível observar que o poder rapidamente
se aproxima assintoticamente ao valor máximo no caso do índice AVG para
a distribuição onde o tamanho é mantido, enquanto que no caso onde
tamanho é removido, o poder é sempre menor, mas há um incremento em
função do tamanho amostral. Para o índice RV, o poder é sempre menor do
que no caso do AVGi; nas matrizes da distribuição sem tamanho, essa
diferença é ainda mais marcante.

Para as matrizes derivadas das correlações entre variáveis de forma
(\autoref{fig:type2_def}), a distribuição de poder em função de
$P(\alpha)$ se comporta de maneira similar em ambos os casos (AVGi e
RV), sempre em valores bastante altos. No caso das matrizes derivadas
das distribuições de correlação de resíduos de Procrustes
(\autoref{fig:type2_sym}), há uma diferença substancial no poder de
ambos os testes; ainda que para o índice AVG, o poder dos testes seja
consideravelmente mais baixo do que em outros tipos de variáveis, ele se
aproxima de 75\% para tamanhos amostrais elevados, no caso do índice RV,
o poder se aproxima da reta identidade em relação a $P(\alpha)$.

\begin{figure}[htbp]
\centering
\includegraphics{Output/modcomp_files/figure-latex/type2_ed-1.pdf}
\caption{Power for both AVGi and RV metrics as a function of the chosen
significance levels with respect to tests for modular structure applied
on $\mathbf{C}_s$ matrices, with correlations sampled from the
interlandmark correlation distribution. \label{fig:type2_ed}}
\end{figure}

\begin{figure}[htbp]
\centering
\includegraphics{Output/modcomp_files/figure-latex/type2_def-1.pdf}
\caption{Power for both AVGi and RV metrics as a function of the chosen
significance levels with respect to tests for modular structure applied
on $\mathbf{C}_s$ matrices, with correlations sampled from the local
shape variables correlation distribution. \label{fig:type2_def}}
\end{figure}

\begin{figure}[htbp]
\centering
\includegraphics{Output/modcomp_files/figure-latex/type2_sym-1.pdf}
\caption{Power for both AVGi and RV metrics as a function of the chosen
significance levels with respect to tests for modular structure applied
on $\mathbf{C}_s$ matrices, with correlations sampled from the local
shape variables correlation distribution. \label{fig:type2_sym}}
\end{figure}

\section{Discussão}\label{discussao}

Considerando os resultados obtidos a partir das matrizes teóricas,
observamos que a distribuição de correlações derivadas de cada tipo de
variável (\autoref{fig:cor_dist}) são particularmente importantes para o
seu entendimento, visto que estas matrizes teóricas são derivadas destas
distribuições. No caso dos resíduos de Procrustes, o procedimento de
superposição e a premissa de variação isotrópica em torno de cada marco
anatômico acaba por gerar uma distribuição de correlações médias dentro
e entre regiões que é marcada por uma diferença pequena entre estes dois
tipos de correlações, independentemente se consideramos a distribuição
com tamanho mantido ou removido. Visto que esta diferença é pequena, o
poder de ambos os teste se reduz bastante em testes realizados nas
matrizes $\mathbf{C}_s$ derivadas desta distribuição.

Ademais, estas correlações médias derivadas de resíduos de Procrustes
são bastante próximas de zero, de modo que o poder do índice RV é
bastante prejudicado neste caso, visto que o cômputo do RV envolve
variâncias e covariâncias ao quadrado. Como o índice AVG lida com
correlações em sua escala natural, sua performance em matrizes
$\mathbf{C}_s$ não é drasticamente prejudicada.

Bookstein, F.L. 1991. \emph{Morphometric tools for landmark data:
geometry and biology}. Cambridge University Press, Cambridge.

Bookstein, F.L., Chernoff, B., Elder, R., Humphries, Smith, G. \&
Strauss, R. 1985. \emph{Morphometrics in Evolutionary Biology}. The
Academy of Natural Sciences of Philadelphia, Philadelphia.

Klingenberg, C.P. 2009. Morphometric integration and modularity in
configurations of landmarks: tools for evaluating a priori hypotheses.
\emph{Evolution \& Development} \textbf{11}: 405--421.

Marroig, G. \& Cheverud, J.M. 2001. A comparison of phenotypic variation
and covariation patterns and the role of phylogeny, ecology, and
ontogeny during cranial evolution of new world monkeys. \emph{Evolution}
\textbf{55}: 2576--2600.

Marroig, G. \& Cheverud, J.M. 2004. Cranial evolution in sakis
(Pithecia, Platyrrhini) I: Interspecific differentiation and allometric
patterns. \emph{American Journal of Physical Anthropology} \textbf{125}:
266--278.

Márquez, E.J., Cabeen, R., Woods, R.P. \& Houle, D. 2012. The
Measurement of Local Variation in Shape. \emph{Evolutionary Biology}
\textbf{39}: 419--439.

Mitteroecker, P., Gunz, P., Bernhard, M., Bookstein, F.L. \& Schaefer,
K. 2004. Comparison of cranial ontogenetic trajectories among great apes
and humans. \emph{Journal of Human Evolution} \textbf{46}: 679--697.

Oliveira, F.B., Porto, A. \& Marroig, G. 2009. Covariance structure in
the skull of Catarrhini: a case of pattern stasis and magnitude
evolution. \emph{Journal of Human Evolution} \textbf{56}: 417--430.

Polly, P.D. 2005. Development and phenotypic correlations: the evolution
of tooth shape in sorex araneus. \emph{Evolution \& Development}
\textbf{7}: 29--41.

Porto, A., Shirai, L.T., Oliveira, F.B. de \& Marroig, G. 2013. Size
Variation, Growth Strategies, and the Evolution of Modularity in the
Mammalian Skull. \emph{Evolution} \textbf{67}: 3305--3322.

Shirai, L.T. \& Marroig, G. 2010. Skull modularity in neotropical
marsupials and monkeys: size variation and evolutionary constraint and
flexibility. \emph{Journal of experimental zoology. Part B, Molecular
and developmental evolution} \textbf{314B}: 663--683.

Zelditch, M.L., Swiderski, D.L., Sheets, H.D. \& Fink, W.L. 2004.
\emph{Geometric Morphometrics for Biologists: A Primer}, 1st ed.
Elsevier.

\end{document}
