\begin {table}[hp]
  \centering
  \caption {As 38 distâncias euclidianas calculadas sobre os marcos anatômicos e as regiões e sub-regiões às quais cada caráter pertence. \label{tab:dist}}
  
  \hr
  \begin {tabularx} {\textwidth} {X X X}
    \bf{Distância} & \bf{Região} & \bf{Sub-região}  \\
    \hline
    IS.PM & Face & Oral \\
    IS.NSL & Face & Nasal \\
    IS.PNS & Face & Oral/Nasal \\
    PM.ZS & Face & Oral \\
    PM.ZI & Face & Oral \\
    PM.MT & Face & Oral \\
    NSL.NA & Face & Nasal \\
    NSL.ZS & Face & Nasal \\
    NSL.ZI & Face & Oral/Nasal \\
    NA.BR & Neurocrânio & Abóbada \\
    NA.FM & Neurocrânio & Órbita \\
    NA.PNS & Face & Nasal \\
    BR.PT & Neurocrânio & Abóbada \\
    BR.APET & Neurocrânio & Abóbada \\
    PT.FM & Neurocrânio & Órbita \\
    PT.APET & Neurocrânio & Abóbada \\
    PT.BA & Neurocrânio & Abóbada \\
    PT.EAM & Neurocrânio & Abóbada \\
    PT.ZYGO & Face & Zigomático \\
    FM.ZS & Neurocrânio & Órbita \\
    ZS.ZI & Face & Oral \\
    ZI.MT & Face & Oral \\
    ZI.ZYGO & Face & Zigomático \\
    ZI.TSP & Face & Zigomático \\
    MT.PNS & Face & Oral \\
    PNS.APET & Neurocrânio & Base \\
    APET.BA & Neurocrânio & Base \\
    APET.TS & Neurocrânio & Base \\
    BA.EAM & Neurocrânio & Base \\
    EAM.ZYGO & Face & Zigomático \\
    ZYGO.TSP & Face & Zigomático \\
    LD.AS & Neurocrânio & Abóbada \\
    BR.LD & Neurocrânio & Abóbada \\
    OPI.LD & Neurocrânio & Abóbada \\
    PT.AS & Neurocrânio & Abóbada \\
    JP.AS & Neurocrânio & Base \\
    BA.OPI & Neurocrânio & Base \\
  \end {tabularx}
  \hr
\end {table} % sk:tab:dist