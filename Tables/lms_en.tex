\begin {sidewaystable} [htp]
  \centering
  \begin{threeparttable}
    \caption {Description of the twenty-two registered landmarks. Cranial regions and subregions to which each landmark is assigned are also indicated.}
    \begin {tabularx} {\textwidth} { l c p{3 cm} p{5.5 cm} X }
      \toprule
      {\bf Landmark} & {\bf Position$^a$} & {\bf Region} & {\bf Subregion} & {\bf Description$^b$} \\
      \midrule
      IS & A & Face & Oral/Nasal
      & {\it intradentale superior}
      \\
      PM & A & Face & Oral
      & premaxillary suture at the alveolus
      \\
      NSL & A & Face & Nasal/Oral
      & {\it nasale} 
      \\
      NA & A & Face & Nasal/Órbita
      & {\it nasion} 
      \\
      BR & AP & Neurocrânio & Abóbada
      & {\it bregma} 
      \\
      PT & AP & Neurocrânio & Abóbada 
      & pterion
      \\
      FM & A & Face & Zigomático/Órbita
      & fronto-malare
      \\
      ZS & A & Face & Oral/Zigomático/Órbita
      & zygomaxillare superior
      \\
      ZI & A & Face & Zigomático 
      & zygomaxillary inferior
      \\
      MT & A & Face & Oral
      & maxillary tuberosity
      \\
      PNS & A & Face & Oral
      & {\it posterior nasal spine} 
      \\
      APET & A & Neurocrânio & Base
      & anterior petrous temporal 
      \\
      BA & AP & Neurocrânio & Base
      & {\it basion} 
      \\
      OPI & AP & Neurocrânio & Base/Abóbada
      & {\it opisthion} 
      \\
      EAM & A & Neurocrânio & Zigomático/Abóbada
      & anterior external auditory meatus
      \\
      PEAM & A & Neurocrânio & Abóbada 
      & posterior external auditory meatus
      \\
      ZYGO & A & Face & Zigomático 
      & inferior zygo-temporal suture
      \\
      TSP & A & Neurocrânio & Zigómatico/Base/Abóbada
      & temporo-spheno-parietal junction
      \\
      TS & AP & Neurocrânio & Base 
      & temporo-sphenoidal junction at the petrous
      \\
      JP & AP & Neurocrânio & Base 
      & jugular process
      \\
      LD & P & Neurocrânio & Abóbada
      & {\it lambda} 
      \\
      AS & P & Neurocrânio & Abóbada 
      & asterion
      \\
      \bottomrule
    \end {tabularx}
    \begin{tablenotes}
      \footnotesize
      {
      \item[$^a$] view in which each landmark was registered (A: anterior, P: posterior, AP: both);
      \item[$^b$] italic descriptions indicate landmarks registered over the saggital plane; remaining landmarks are bilaterally symmetrical.
      }
      \label {tab:lms}
    \end{tablenotes}
  \end{threeparttable}
\end {sidewaystable}