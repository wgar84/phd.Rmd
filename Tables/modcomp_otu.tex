\begin{table}[t]
  \centering
  \begin{threeparttable}
    \caption{Twenty-one species used in the present work. Sample sizes and linear models adjusted over eac OTU are also indicated. \label{tab:modcomp_otu}}
    \begin{tabular}{lccr}
        \toprule
        Species & Group$^a$ & $n$ & Model$^b$ \\ 
      \midrule
        \emph{Alouatta belzebul} & P & 109 & X \\ 
        \emph{Ateles geoffroyi} & P & 78 & - \\ 
        \emph{Cacajao calvus} & P & 48 & S + X \\ 
        \emph{Callicebus moloch} & P & 93 & X \\ 
        \emph{Callithrix kuhlii} & P & 129 & - \\ 
        \emph{Cebus apella} & P & 110 & X \\ 
        \emph{Cercopithecus ascanius} & C & 61 & X \\ 
        \emph{Chiropotes chiropotes} & P & 56 & X \\ 
        \emph{Chlorocebus pygerythrus} & C & 110 & X \\ 
        \emph{Colobus guereza} & C & 140 & X \\ 
        \emph{Gorilla gorilla} & C & 115 & X \\ 
        \emph{Homo sapiens} & C & 160 & S * X \\ 
        \emph{Hylobates lar} & C & 66 & X \\ 
        \emph{Macaca fascicularis} & C & 69 & X \\ 
        \emph{Pan troglodytes} & C & 61 & X \\ 
        \emph{Papio anubis} & C & 46 & X \\ 
        \emph{Piliocolobus foai} & C & 83 & X \\ 
        \emph{Pithecia pithecia} & P & 69 & S + X \\ 
        \emph{Procolobus verus} & C & 88 & X \\ 
        \emph{Saguinus midas} & P & 50 & S \\ 
        \emph{Saimiri sciureus} & P & 87 & X \\ 
      \bottomrule
    \end{tabular}
      \begin{tablenotes}
        \footnotesize
        {
        \item[$a$] C: Catarrhini; P: Platyrrhini
        \item[$b$] S: subspecies/population; X: sex.
        }
      \end{tablenotes}
    \end{threeparttable}
  \end{table}
