\documentclass[11pt,]{article}
\usepackage{lmodern}
\usepackage{amssymb,amsmath}
\usepackage{ifxetex,ifluatex}
\usepackage{fixltx2e} % provides \textsubscript
\ifnum 0\ifxetex 1\fi\ifluatex 1\fi=0 % if pdftex
  \usepackage[T1]{fontenc}
  \usepackage[utf8]{inputenc}
\else % if luatex or xelatex
  \ifxetex
    \usepackage{mathspec}
    \usepackage{xltxtra,xunicode}
  \else
    \usepackage{fontspec}
  \fi
  \defaultfontfeatures{Mapping=tex-text,Scale=MatchLowercase}
  \newcommand{\euro}{€}
\fi
% use upquote if available, for straight quotes in verbatim environments
\IfFileExists{upquote.sty}{\usepackage{upquote}}{}
% use microtype if available
\IfFileExists{microtype.sty}{%
\usepackage{microtype}
\UseMicrotypeSet[protrusion]{basicmath} % disable protrusion for tt fonts
}{}
\usepackage[margin=1in]{geometry}
\usepackage{graphicx}
\makeatletter
\def\maxwidth{\ifdim\Gin@nat@width>\linewidth\linewidth\else\Gin@nat@width\fi}
\def\maxheight{\ifdim\Gin@nat@height>\textheight\textheight\else\Gin@nat@height\fi}
\makeatother
% Scale images if necessary, so that they will not overflow the page
% margins by default, and it is still possible to overwrite the defaults
% using explicit options in \includegraphics[width, height, ...]{}
\setkeys{Gin}{width=\maxwidth,height=\maxheight,keepaspectratio}
\ifxetex
  \usepackage[setpagesize=false, % page size defined by xetex
              unicode=false, % unicode breaks when used with xetex
              xetex]{hyperref}
\else
  \usepackage[unicode=true]{hyperref}
\fi
\hypersetup{breaklinks=true,
            bookmarks=true,
            pdfauthor={},
            pdftitle={Supplemental Material for Type I and II Error Rates in Modularity Hypothesis Testing},
            colorlinks=true,
            citecolor=blue,
            urlcolor=blue,
            linkcolor=magenta,
            pdfborder={0 0 0}}
\urlstyle{same}  % don't use monospace font for urls
\setlength{\parindent}{0pt}
\setlength{\parskip}{6pt plus 2pt minus 1pt}
\setlength{\emergencystretch}{3em}  % prevent overfull lines
\setcounter{secnumdepth}{0}

%%% Use protect on footnotes to avoid problems with footnotes in titles
\let\rmarkdownfootnote\footnote%
\def\footnote{\protect\rmarkdownfootnote}

%%% Change title format to be more compact
\usepackage{titling}

% Create subtitle command for use in maketitle
\newcommand{\subtitle}[1]{
  \posttitle{
    \begin{center}\large#1\end{center}
    }
}

\setlength{\droptitle}{-2em}
  \title{Supplemental Material for `Type I and II Error Rates in Modularity
Hypothesis Testing'}
  \pretitle{\vspace{\droptitle}\centering\huge}
  \posttitle{\par}
  \author{}
  \preauthor{}\postauthor{}
  \date{}
  \predate{}\postdate{}

%--- USEPACKAGES ---
\usepackage[utf8]{inputenc}
\usepackage{natbib}
\usepackage[brazilian,english]{babel}
\usepackage{hyperref}
\usepackage{subfigure, epsfig}
\usepackage{ae}
\usepackage{aecompl}
\usepackage{booktabs}
\usepackage[T1]{fontenc}
\usepackage{graphicx,wrapfig} % para incluir figuras
\usepackage{amsfonts, amssymb,amsthm, amsmath, amscd} % pacote AMS
\usepackage{color, bbm, multicol}
\usepackage{verbatim, listings, booktabs}
\usepackage{fancyhdr} % FANCYHEADER
\usepackage{setspace}
\usepackage{times} % bookman,palatino,courier,times FONTES
\usepackage{lineno}
\usepackage{url}
\usepackage{rotating}
\usepackage{longtable}
% \usepackage{rotfloat}
% \usepackage{appendix}
\usepackage{mathptmx}
% \usepackage{cmbright}
% \usepackage[adobe-utopia]{mathdesign}
\usepackage[flushleft]{threeparttable}
\usepackage{multirow}
% \usepackage{float}
\usepackage {tocvsec2} % controlar profundidade de table of contents
\setcounter {secnumdepth}{0}
\usepackage {caption}
\usepackage {tabularx}
\usepackage {floatrow}
\floatsetup[table]{capposition=top}

\usepackage{mathpazo} % fonte palatino

\newcommand{\barra}{\backslash}
\newcommand{\To}{\longrightarrow}
\newcommand{\abs}[1]{\left\vert#1\right\vert}
\newcommand{\set}[1]{\left\{#1\right\}}
\newcommand{\seq}[1]{\left<#1\right>}
\newcommand{\norma}[1]{\left\Vert#1\right\Vert}
\newcommand{\hr}{\par\noindent\hrulefill\par}

\usepackage {xr}
\externaldocument{modcomp}

\selectlanguage{english}

\hypersetup{colorlinks=false}


\begin{document}

\maketitle


\linenumbers
\modulolinenumbers[2]

\onehalfspacing

\renewcommand{\thefigure}{S\arabic{figure}}
\renewcommand{\thetable}{S\arabic{table}}

\begin {sidewaystable} [htp]
  \centering
  \begin{threeparttable}
    \caption {Description of the twenty-two registered landmarks. Cranial regions and subregions to which each landmark is assigned are also indicated.}
    \begin {tabularx} {\textwidth} { l c p{3 cm} p{5.5 cm} X }
      \toprule
      {\bf Landmark} & {\bf Position$^a$} & {\bf Region} & {\bf Subregion} & {\bf Description$^b$} \\
      \midrule
      IS & A & Face & Oral/Nasal
      & {\it intradentale superior}
      \\
      PM & A & Face & Oral
      & premaxillary suture at the alveolus
      \\
      NSL & A & Face & Nasal/Oral
      & {\it nasale} 
      \\
      NA & A & Face & Nasal/Órbita
      & {\it nasion} 
      \\
      BR & AP & Neurocrânio & Abóbada
      & {\it bregma} 
      \\
      PT & AP & Neurocrânio & Abóbada 
      & pterion
      \\
      FM & A & Face & Zigomático/Órbita
      & fronto-malare
      \\
      ZS & A & Face & Oral/Zigomático/Órbita
      & zygomaxillare superior
      \\
      ZI & A & Face & Zigomático 
      & zygomaxillary inferior
      \\
      MT & A & Face & Oral
      & maxillary tuberosity
      \\
      PNS & A & Face & Oral
      & {\it posterior nasal spine} 
      \\
      APET & A & Neurocrânio & Base
      & anterior petrous temporal 
      \\
      BA & AP & Neurocrânio & Base
      & {\it basion} 
      \\
      OPI & AP & Neurocrânio & Base/Abóbada
      & {\it opisthion} 
      \\
      EAM & A & Neurocrânio & Zigomático/Abóbada
      & anterior external auditory meatus
      \\
      PEAM & A & Neurocrânio & Abóbada 
      & posterior external auditory meatus
      \\
      ZYGO & A & Face & Zigomático 
      & inferior zygo-temporal suture
      \\
      TSP & A & Neurocrânio & Zigómatico/Base/Abóbada
      & temporo-spheno-parietal junction
      \\
      TS & AP & Neurocrânio & Base 
      & temporo-sphenoidal junction at the petrous
      \\
      JP & AP & Neurocrânio & Base 
      & jugular process
      \\
      LD & P & Neurocrânio & Abóbada
      & {\it lambda} 
      \\
      AS & P & Neurocrânio & Abóbada 
      & asterion
      \\
      \bottomrule
    \end {tabularx}
    \begin{tablenotes}
      \footnotesize
      {
      \item[$^a$] view in which each landmark was registered (A: anterior, P: posterior, AP: both);
      \item[$^b$] italic descriptions indicate landmarks registered over the saggital plane; remaining landmarks are bilaterally symmetrical.
      }
      \label {tab:lms}
    \end{tablenotes}
  \end{threeparttable}
\end {sidewaystable}

\begin{figure}[htbp]
\centering
\includegraphics{Figures/landmarks.png}
\caption{Landmark configuration. Lines connecting landmarks indicate
traits considered, either as interlandmark distances or local shape
variables. Dotted and dashed lines indicate the association of each
trait to \emph{a priori} regional hypotheses of association (Face and
Neurocranium, respectively). \label{fig:landmarks}}
\end{figure}

\begin {table}[hp]
  \centering
  \caption {List of thirty-eight interlandmark distances calculated over all configurations. Cranial regions and sub-regions to which each trait is assigned are also indicated. This table also indicates the location of local shape variables calculated, since this type of morphometric variable was estimated at the midpoints of each of these distances.}
  \label {tab:dist}
  \hr
  \begin {tabularx} {\textwidth} {X X X}
    {\bf Trait} & {\bf Region} & {\bf Sub-region}  \\
    \hline
    IS.PM & Face & Oral \\
    IS.NSL & Face & Nasal \\
    IS.PNS & Face & Oral/Nasal \\
    PM.ZS & Face & Oral \\
    PM.ZI & Face & Oral \\
    PM.MT & Face & Oral \\
    NSL.NA & Face & Nasal \\
    NSL.ZS & Face & Nasal \\
    NSL.ZI & Face & Oral/Nasal \\
    NA.BR & Neurocranium & Vault \\
    NA.FM & Neurocranium & Orbit \\
    NA.PNS & Face & Nasal \\
    BR.PT & Neurocranium & Vault \\
    BR.APET & Neurocranium & Vault \\
    PT.FM & Neurocranium & Orbit \\
    PT.APET & Neurocranium & Vault \\
    PT.BA & Neurocranium & Vault \\
    PT.EAM & Neurocranium & Vault \\
    PT.ZYGO & Face & Zygomatic \\
    FM.ZS & Neurocranium & Orbit \\
    ZS.ZI & Face & Oral \\
    ZI.MT & Face & Oral \\
    ZI.ZYGO & Face & Zygomatic \\
    ZI.TSP & Face & Zygomatic \\
    MT.PNS & Face & Oral \\
    PNS.APET & Neurocranium & Base \\
    APET.BA & Neurocranium & Base \\
    APET.TS & Neurocranium & Base \\
    BA.EAM & Neurocranium & Base \\
    EAM.ZYGO & Face & Zygomatic \\
    ZYGO.TSP & Face & Zygomatic \\
    LD.AS & Neurocranium & Vault \\
    BR.LD & Neurocranium & Vault \\
    OPI.LD & Neurocranium & Vault \\
    PT.AS & Neurocranium & Vault \\
    JP.AS & Neurocranium & Base \\
    BA.OPI & Neurocranium & Base \\
  \end {tabularx}
  \hr
\end {table} % sk:tab:dist

\end{document}
