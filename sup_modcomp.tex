\documentclass[11pt,]{article}
\usepackage{lmodern}
\usepackage{amssymb,amsmath}
\usepackage{ifxetex,ifluatex}
\usepackage{fixltx2e} % provides \textsubscript
\ifnum 0\ifxetex 1\fi\ifluatex 1\fi=0 % if pdftex
  \usepackage[T1]{fontenc}
  \usepackage[utf8]{inputenc}
\else % if luatex or xelatex
  \ifxetex
    \usepackage{mathspec}
    \usepackage{xltxtra,xunicode}
  \else
    \usepackage{fontspec}
  \fi
  \defaultfontfeatures{Mapping=tex-text,Scale=MatchLowercase}
  \newcommand{\euro}{€}
\fi
% use upquote if available, for straight quotes in verbatim environments
\IfFileExists{upquote.sty}{\usepackage{upquote}}{}
% use microtype if available
\IfFileExists{microtype.sty}{%
\usepackage{microtype}
\UseMicrotypeSet[protrusion]{basicmath} % disable protrusion for tt fonts
}{}
\usepackage[margin=1in]{geometry}
\usepackage{graphicx}
\makeatletter
\def\maxwidth{\ifdim\Gin@nat@width>\linewidth\linewidth\else\Gin@nat@width\fi}
\def\maxheight{\ifdim\Gin@nat@height>\textheight\textheight\else\Gin@nat@height\fi}
\makeatother
% Scale images if necessary, so that they will not overflow the page
% margins by default, and it is still possible to overwrite the defaults
% using explicit options in \includegraphics[width, height, ...]{}
\setkeys{Gin}{width=\maxwidth,height=\maxheight,keepaspectratio}
\ifxetex
  \usepackage[setpagesize=false, % page size defined by xetex
              unicode=false, % unicode breaks when used with xetex
              xetex]{hyperref}
\else
  \usepackage[unicode=true]{hyperref}
\fi
\hypersetup{breaklinks=true,
            bookmarks=true,
            pdfauthor={Guilherme Garcia1,2, Felipe Bandoni de Oliveira1 \& Gabriel Marroig1},
            pdftitle={Modularity and Morphometrics: Error Rates in Hypothesis Testing - Supplemental Information},
            colorlinks=true,
            citecolor=blue,
            urlcolor=blue,
            linkcolor=magenta,
            pdfborder={0 0 0}}
\urlstyle{same}  % don't use monospace font for urls
\setlength{\parindent}{0pt}
\setlength{\parskip}{6pt plus 2pt minus 1pt}
\setlength{\emergencystretch}{3em}  % prevent overfull lines
\setcounter{secnumdepth}{0}

%%% Use protect on footnotes to avoid problems with footnotes in titles
\let\rmarkdownfootnote\footnote%
\def\footnote{\protect\rmarkdownfootnote}

%%% Change title format to be more compact
\usepackage{titling}

% Create subtitle command for use in maketitle
\newcommand{\subtitle}[1]{
  \posttitle{
    \begin{center}\large#1\end{center}
    }
}

\setlength{\droptitle}{-2em}
  \title{Modularity and Morphometrics: Error Rates in Hypothesis Testing -
Supplemental Information}
  \pretitle{\vspace{\droptitle}\centering\huge}
  \posttitle{\par}
  \author{Guilherme Garcia\textsuperscript{1,2}, Felipe Bandoni de
Oliveira\textsuperscript{1} \& Gabriel Marroig\textsuperscript{1}}
  \preauthor{\centering\large\emph}
  \postauthor{\par}
  \predate{\centering\large\emph}
  \postdate{\par}
  \date{06 November 2015}

%--- USEPACKAGES ---
\usepackage[utf8]{inputenc}
\usepackage{natbib}
\usepackage[brazilian,english]{babel}
\usepackage{hyperref}
\usepackage{subfigure, epsfig}
\usepackage{ae}
\usepackage{aecompl}
\usepackage{booktabs}
\usepackage[T1]{fontenc}
\usepackage{graphicx,wrapfig} % para incluir figuras
\usepackage{amsfonts, amssymb,amsthm, amsmath, amscd} % pacote AMS
\usepackage{color, bbm, multicol}
\usepackage{verbatim, listings, booktabs}
\usepackage{fancyhdr} % FANCYHEADER
\usepackage{setspace}
\usepackage{times} % bookman,palatino,courier,times FONTES
\usepackage{lineno}
\usepackage{url}
\usepackage{rotating}
\usepackage{longtable}
% \usepackage{rotfloat}
% \usepackage{appendix}
\usepackage{mathptmx}
% \usepackage{cmbright}
% \usepackage[adobe-utopia]{mathdesign}
\usepackage[flushleft]{threeparttable}
\usepackage{multirow}
% \usepackage{float}
\usepackage {tocvsec2} % controlar profundidade de table of contents
\setcounter {secnumdepth}{0}
\usepackage {caption}
\usepackage {tabularx}
\usepackage {floatrow}
\floatsetup[table]{capposition=top}

\usepackage{mathpazo} % fonte palatino

\newcommand{\barra}{\backslash}
\newcommand{\To}{\longrightarrow}
\newcommand{\abs}[1]{\left\vert#1\right\vert}
\newcommand{\set}[1]{\left\{#1\right\}}
\newcommand{\seq}[1]{\left<#1\right>}
\newcommand{\norma}[1]{\left\Vert#1\right\Vert}
\newcommand{\hr}{\par\noindent\hrulefill\par}

\usepackage {xr}
\externaldocument{modcomp}

\selectlanguage{english}

\hypersetup{colorlinks=false}


\begin{document}

\maketitle


\linenumbers
\modulolinenumbers[2]

\onehalfspacing

\textsuperscript{1}Laboratório de Evolução de Mamíferos, Departamento de
Genética e Biologia Evolutiva, Instituto de Biociências, Universidade de
São Paulo, CP 11.461, CEP 05422-970, São Paulo, Brasil

\textsuperscript{2}\href{mailto:wgar@usp.br}{wgar@usp.br}

\renewcommand{\thefigure}{S\arabic{figure}}
\renewcommand{\thetable}{S\arabic{table}}

\begin {sidewaystable} [htp]
  \centering
  \begin{threeparttable}
    \caption {Description of the twenty-two registered landmarks. Cranial regions and subregions to which each landmark is assigned are also indicated.}
    \begin {tabularx} {\textwidth} { l c p{3 cm} p{5.5 cm} X }
      \toprule
      {\bf Landmark} & {\bf Position$^a$} & {\bf Region} & {\bf Sub-region} & {\bf Description$^b$} \\
      \midrule
      IS & A & Face & Oral/Nasal
      & {\it intradentale superior}
      \\
      PM & A & Face & Oral
      & premaxillary suture at the alveolus
      \\
      NSL & A & Face & Nasal/Oral
      & {\it nasale} 
      \\
      NA & A & Face & Nasal/Orbit
      & {\it nasion} 
      \\
      BR & AP & Neurocranium & Vault
      & {\it bregma} 
      \\
      PT & AP & Neurocranium & Vault 
      & pterion
      \\
      FM & A & Face & Zygomatic/Orbit
      & fronto-malare
      \\
      ZS & A & Face & Oral/Zygomatic/Orbit
      & zygomaxillare superior
      \\
      ZI & A & Face & Zygomatic 
      & zygomaxillary inferior
      \\
      MT & A & Face & Oral
      & maxillary tuberosity
      \\
      PNS & A & Face & Oral
      & {\it posterior nasal spine} 
      \\
      APET & A & Neurocranium & Base
      & anterior petrous temporal 
      \\
      BA & AP & Neurocranium & Base
      & {\it basion} 
      \\
      OPI & AP & Neurocranium & Base/Vault
      & {\it opisthion} 
      \\
      EAM & A & Neurocranium & Zygomatic/Vault
      & anterior external auditory meatus
      \\
      PEAM & A & Neurocranium & Vault 
      & posterior external auditory meatus
      \\
      ZYGO & A & Face & Zygomatic 
      & inferior zygo-temporal suture
      \\
      TSP & A & Neurocranium & Zygomatic/Base/Vault
      & temporo-spheno-parietal junction
      \\
      TS & AP & Neurocranium & Base 
      & temporo-sphenoidal junction at the petrous
      \\
      JP & AP & Neurocranium & Base 
      & jugular process
      \\
      LD & P & Neurocranium & Vault
      & {\it lambda} 
      \\
      AS & P & Neurocranium & Vault 
      & asterion
      \\
      \bottomrule
    \end {tabularx}
    \begin{tablenotes}
      \footnotesize
      {
      \item[$^a$] view in which each landmark was registered (A: anterior, P: posterior, AP: both);
      \item[$^b$] italic descriptions indicate landmarks registered over the sagittal plane; remaining landmarks are bilaterally symmetrical.
      }
      \label {tab:lms}
    \end{tablenotes}
  \end{threeparttable}
\end {sidewaystable}

\begin{figure}[htbp]
\centering
\includegraphics{Figures/landmarks.png}
\caption{Landmark configuration. Lines connecting landmarks indicate
traits considered, either as interlandmark distances or local shape
variables. Dotted and dashed lines indicate the association of each
trait to \emph{a priori} regional hypotheses of association (Face and
Neurocranium, respectively). \label{fig:landmarks}}
\end{figure}

\begin {table}[hp]
  \centering
  \caption {List of thirty-eight interlandmark distances calculated over all configurations. Cranial regions and sub-regions to which each trait is assigned are also indicated.}
  \label {tab:dist}
  \hr
  \begin {tabularx} {\textwidth} {X X X}
    \bf{Trait} & \bf{Region} & \bf{Subregion}  \\
    \hline
    IS.PM & Face & Oral \\
    IS.NSL & Face & Nasal \\
    IS.PNS & Face & Oral/Nasal \\
    PM.ZS & Face & Oral \\
    PM.ZI & Face & Oral \\
    PM.MT & Face & Oral \\
    NSL.NA & Face & Nasal \\
    NSL.ZS & Face & Nasal \\
    NSL.ZI & Face & Oral/Nasal \\
    NA.BR & Neurocranium & Vault \\
    NA.FM & Neurocranium & Orbit \\
    NA.PNS & Face & Nasal \\
    BR.PT & Neurocranium & Vault \\
    BR.APET & Neurocranium & Vault \\
    PT.FM & Neurocranium & Orbit \\
    PT.APET & Neurocranium & Vault \\
    PT.BA & Neurocranium & Vault \\
    PT.EAM & Neurocranium & Vault \\
    PT.ZYGO & Face & Zygomatic \\
    FM.ZS & Neurocranium & Orbit \\
    ZS.ZI & Face & Oral \\
    ZI.MT & Face & Oral \\
    ZI.ZYGO & Face & Zygomatic \\
    ZI.TSP & Face & Zygomatic \\
    MT.PNS & Face & Oral \\
    PNS.APET & Neurocranium & Base \\
    APET.BA & Neurocranium & Base \\
    APET.TS & Neurocranium & Base \\
    BA.EAM & Neurocranium & Base \\
    EAM.ZYGO & Face & Zygomatic \\
    ZYGO.TSP & Face & Zygomatic \\
    LD.AS & Neurocranium & Vault \\
    BR.LD & Neurocranium & Vault \\
    OPI.LD & Neurocranium & Vault \\
    PT.AS & Neurocranium & Vault \\
    JP.AS & Neurocranium & Base \\
    BA.OPI & Neurocranium & Base \\
  \end {tabularx}
  \hr
\end {table} % sk:tab:dist

\subsection{Constructing Theoretical
Matrices}\label{constructing-theoretical-matrices}

To evaluate the statistical properties of metrics, types of morphometric
variables, and the randomization procedure, we built theoretical
correlation matrices \[
\mathbf{C}_{s} =
\begin{bmatrix}
\mathbf{W}_1 & \mathbf{B} \\
\mathbf{B}^t & \mathbf{W}_2 \\
\end{bmatrix}
\] where $\mathbf{W}_1$ and $\mathbf{W}_2$ are correlation blocks
associated with two distinct trait sets, and $\mathbf{B}$ represents the
correlation block between sets.

For each correlation matrix in our empirical dataset, we estimated
average correlations within and between all trait sets we considered,
obtaining a distribution of within and between sets correlations
associated with each type of morphometric variable (Procrustes
Residuals, Interlandmark Distances and Local Shape Variables), also
considering whether size was retained or removed
(\autoref{fig:cor_dist}). We constructed corresponding sets of
$\mathbf{C}_{s}$ matrices sampling each distribution obtained; for each
matrix, we sampled two within-set correlations and one between-set
correlation, filling the corresponding blocks ($\mathbf{W}_1$,
$\mathbf{W}_2$ and $\mathbf{B}$) in each theoretical matrix with the
sampled correlations. For example, with only four traits divided into
two blocks of two traits, sampling the values $0.5$ and $0.3$ from the
within-set distribution and $-0.1$ from the between-set distribution
produces \[
\mathbf{C}_s =
\begin{bmatrix}
1 & 0.5 & -0.1 & -0.1 \\
0.5 & 1 & -0.1 & -0.1 \\
-0.1 & -0.1 & 1 & 0.3 \\
-0.1 & -0.1 & 0.3 & 1 \\
\end{bmatrix}
\] filling all cells in each block with the associated sampled
correlation. Thus, we represent each type of morphometric variable as
two correlation distributions (\autoref{fig:cor_dist}), building six
sets of theoretical correlation matrices that are representative of each
type (considering the presence or absence of size variation) which
retain their statistical properties.

\begin{figure}[htbp]
\centering
\includegraphics{Figures/cor_dist-1.pdf}
\caption{Distribution of within-set and between-set mean correlations
derived from the six types of empirical correlation matrices.
\label{fig:cor_dist}}
\end{figure}

For each pair of correlation distributions, we built 10000 correlation
matrices for 40 traits; previous tests indicate that matrix
dimensionality does not qualitatively affect our results. We also
sampled variances from each type of morphometric variable, thus
constructing an associated covariance matrix, since the estimation of RV
values uses covariance matrices. We considered only positive-definite
matrices; if any given matrix did not fit this criterion, we discarded
that matrix and sampled new correlations; thus, we can sample
observations from a multivariate normal distribution using each of these
60000 matrices as the $\Sigma$ parameter. For each matrix, we also
randomly determine the number of traits in both sets, establishing a
minimum value of five for set size.

We used this set of 60000 covariance matrices of known structure
($\mathbf{C}_s$) to built another set of covariance matrices of unknown,
random structure ($\mathbf{C}_r$) by simply shuffling both lines and
columns of each matrix. For all matrices ($\mathbf{C}_r$ and
$\mathbf{C}_s$) we obtained samples of increasing size (20, 40, 60, 80,
100 individuals) and re-estimated a covariance matrix for each sample,
thus considering in our tests the uncertainty derived from sampling. For
each matrix estimated, both $\mathbf{C}_r$ and $\mathbf{C}_s$, we test
the hypothesis that the two sets used to generate each $\mathbf{C}_s$
matrix represent two variational modules, using both MHI and RV
coefficients, as described in the section `Empirical Tests'.

\end{document}
