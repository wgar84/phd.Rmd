\documentclass[11pt,]{article}
\usepackage{lmodern}
\usepackage{amssymb,amsmath}
\usepackage{ifxetex,ifluatex}
\usepackage{fixltx2e} % provides \textsubscript
\ifnum 0\ifxetex 1\fi\ifluatex 1\fi=0 % if pdftex
  \usepackage[T1]{fontenc}
  \usepackage[utf8]{inputenc}
\else % if luatex or xelatex
  \ifxetex
    \usepackage{mathspec}
    \usepackage{xltxtra,xunicode}
  \else
    \usepackage{fontspec}
  \fi
  \defaultfontfeatures{Mapping=tex-text,Scale=MatchLowercase}
  \newcommand{\euro}{€}
\fi
% use upquote if available, for straight quotes in verbatim environments
\IfFileExists{upquote.sty}{\usepackage{upquote}}{}
% use microtype if available
\IfFileExists{microtype.sty}{%
\usepackage{microtype}
\UseMicrotypeSet[protrusion]{basicmath} % disable protrusion for tt fonts
}{}
\usepackage[margin=1in]{geometry}
\usepackage{graphicx}
\makeatletter
\def\maxwidth{\ifdim\Gin@nat@width>\linewidth\linewidth\else\Gin@nat@width\fi}
\def\maxheight{\ifdim\Gin@nat@height>\textheight\textheight\else\Gin@nat@height\fi}
\makeatother
% Scale images if necessary, so that they will not overflow the page
% margins by default, and it is still possible to overwrite the defaults
% using explicit options in \includegraphics[width, height, ...]{}
\setkeys{Gin}{width=\maxwidth,height=\maxheight,keepaspectratio}
\ifxetex
  \usepackage[setpagesize=false, % page size defined by xetex
              unicode=false, % unicode breaks when used with xetex
              xetex]{hyperref}
\else
  \usepackage[unicode=true]{hyperref}
\fi
\hypersetup{breaklinks=true,
            bookmarks=true,
            pdfauthor={Guilherme Garcia1,2, Felipe Bandoni de Oliveira1 \& Gabriel Marroig1},
            pdftitle={A Phylogenetic Analysis of Covariance Structure in Anthropoids},
            colorlinks=true,
            citecolor=blue,
            urlcolor=blue,
            linkcolor=magenta,
            pdfborder={0 0 0}}
\urlstyle{same}  % don't use monospace font for urls
\setlength{\parindent}{0pt}
\setlength{\parskip}{6pt plus 2pt minus 1pt}
\setlength{\emergencystretch}{3em}  % prevent overfull lines
\setcounter{secnumdepth}{0}

%%% Use protect on footnotes to avoid problems with footnotes in titles
\let\rmarkdownfootnote\footnote%
\def\footnote{\protect\rmarkdownfootnote}

%%% Change title format to be more compact
\usepackage{titling}

% Create subtitle command for use in maketitle
\newcommand{\subtitle}[1]{
  \posttitle{
    \begin{center}\large#1\end{center}
    }
}

\setlength{\droptitle}{-2em}
  \title{A Phylogenetic Analysis of Covariance Structure in Anthropoids}
  \pretitle{\vspace{\droptitle}\centering\huge}
  \posttitle{\par}
  \author{Guilherme Garcia\textsuperscript{1,2}, Felipe Bandoni de
Oliveira\textsuperscript{1} \& Gabriel Marroig\textsuperscript{1}}
  \preauthor{\centering\large\emph}
  \postauthor{\par}
  \predate{\centering\large\emph}
  \postdate{\par}
  \date{16 September 2015}

%--- USEPACKAGES ---
\usepackage[utf8]{inputenc}
\usepackage{natbib}
\usepackage[brazilian,english]{babel}
\usepackage{hyperref}
\usepackage{subfigure, epsfig}
\usepackage{ae}
\usepackage{aecompl}
\usepackage{booktabs}
\usepackage[T1]{fontenc}
\usepackage{graphicx,wrapfig} % para incluir figuras
\usepackage{amsfonts, amssymb,amsthm, amsmath, amscd} % pacote AMS
\usepackage{color, bbm, multicol}
\usepackage{verbatim, listings, booktabs}
\usepackage{fancyhdr} % FANCYHEADER
\usepackage{setspace}
\usepackage{times} % bookman,palatino,courier,times FONTES
\usepackage{lineno}
\usepackage{url}
\usepackage{rotating}
\usepackage{longtable}
% \usepackage{rotfloat}
% \usepackage{appendix}
\usepackage{mathptmx}
% \usepackage{cmbright}
% \usepackage[adobe-utopia]{mathdesign}
\usepackage[flushleft]{threeparttable}
\usepackage{multirow}
% \usepackage{float}
\usepackage {tocvsec2} % controlar profundidade de table of contents
\setcounter {secnumdepth}{0}
\usepackage {caption}
\usepackage {tabularx}
\usepackage {floatrow}
\floatsetup[table]{capposition=top}

\usepackage{mathpazo} % fonte palatino

\newcommand{\barra}{\backslash}
\newcommand{\To}{\longrightarrow}
\newcommand{\abs}[1]{\left\vert#1\right\vert}
\newcommand{\set}[1]{\left\{#1\right\}}
\newcommand{\seq}[1]{\left<#1\right>}
\newcommand{\norma}[1]{\left\Vert#1\right\Vert}
\newcommand{\hr}{\par\noindent\hrulefill\par}

\usepackage {xr}
\externaldocument{sup_ppca}

\selectlanguage{english}

\hypersetup{colorlinks=false}


\begin{document}

\maketitle


\linenumbers
\modulolinenumbers[2]

\onehalfspacing

\textsuperscript{1}Laboratório de Evolução de Mamíferos, Departamento de
Genética e Biologia Evolutiva, Instituto de Biociências, Universidade de
São Paulo, CP 11.461, CEP 05422-970, São Paulo, Brasil

\textsuperscript{2}\href{mailto:wgar@usp.br}{wgar@usp.br}

running title: PCA of a PCA

key words: $P$-matrix; modularity; primates; development;

\section{Introduction}\label{introduction}

In the present work, we build upon Haber's (2015) contribution on the
subject of comparing patterns of morphological integration under a
phylogenetic framework, using Anthropoid Primates as a model organism,
under the hypothesis that changes in covariance structure among
Anthropoid species will follow a non-random pattern consistent with
functional and developmental interactions.

\section{Methods}\label{methods}

\subsection{Sample}\label{sample}

Our sample consists of 5108 individuals, distributes across 109 species.
These species are distributed throughout all major Anthropoid clades
above the genus level, comprising all Platyrrhini genera and a
substantial portion of Catarrhini genera. We associate this database
with a ultrametric phylogenetic hypothesis for Anthropoidea
(\autoref{fig:phylo_model}), derived from Springer et al. (2012).

Individuals in our sample are represented by 36 registered landmarks,
using either a Polhemus 3Draw or a Microscribe 3DS for Platyrrhini and
Catarrhini, respectively. Twenty-two unique landmarks represent each
individual (Figure \ref{fig:landmarks}), since fourteen of the 36
registered landmarks are bilaterally symmetrical. For more details on
landmark registration, see Marroig \& Cheverud (2001) and Oliveira et
al. (2009). Databases from both previous studies were merged into a
single database, retaining only those individuals in which all landmarks
from both sides were present. In the present work, we considered only
covariance structure for the symmetrical component of variation;
therefore, prior to any analysis, we controlled the effects of variation
in assymmetry. We followed the procedure outlined in Klingenberg et al.
(2002) for bilateral structures by obtaning for each individual a
symmetrical landmark configuration, averaging each actual shape with its
reflection along the sagittal plane.

We used this database to obtain local shape variables (Márquez \emph{et
al.}, 2012), which represent infinitesimal volumetric expansions or
retractions, calculated as the natural logarithm determinants of
derivatives of the TPS function between each individual in our sample
and a reference shape (in our case, the mean shape for the entire
sample, estimated from a Generalized Procrustes algorithm). Such
derivatives were evaluated at the midpoints between pairs of landmarks
represented in \autoref{fig:landmarks}, for a total of 38 local shape
variables.

After obtaining these values, we estimated covariance
$\mathbf{P}$-matrices for size (represented by the logarithm of Centroid
Size) and local shape variables after removing fixed effects of little
interest in the present context, such as sexual dimorphism, for example.
These effects were removed through a multivariate linear model adjusted
for each species, according to \autoref{fig:phylo_model}. We adjusted
such models under a Bayesian framework, sampling $100$ residual
covariance matrices from the posterior distribution of each model. These
distributions allow us to estimate uncertainty for any parameters
derived from these matrices as credibility intervals; furthermore, since
posterior distributions are conditional upon the prior distribution we
used --- a uniform Wishart distribution --- every matrix sampled from
these posterior distributions is also a realization of a Wishart
distribution, therefore positive-definite regardless of sample size
(Gelman \emph{et al.}, 2004); in this framework, lower sample sizes
imply in broader, less informative, credibility intervals. For each
posterior sample, we estimated geometric mean covariance matrices, since
such mean respects the underlying structure of the Riemannian manifold
in which positive-definite symmetric matrices lie (Moakher, 2005, 2006);
thus, these mean $\mathbf{P}$-matrices are also positive-definite,
regardless of sample size. For each species, we ran independent models,
with 13000 iterations of MCMC sampling, discarding the 3000 initial
iterations as a burnin period and further sampling one covariance matrix
per 100 iterations to avoid autocorrelations induced by sequential
sampling.

\subsection{Phylogenetic Decompostion of Matrix
Diversity}\label{phylogenetic-decompostion-of-matrix-diversity}

In order to evaluate the distribution of covariance structure diversity
during Anthropoid diversification, we estimated Riemannian distances
among all pairs of mean $\mathbf{P}$-matrices, according to the
definition given by Mitteroecker \& Bookstein (2009); for any pair of
positive-definite covariance matrices $\mathbf{C}_i$ and $\mathbf{C}_j$
of size $p \times p$, the distance $d(\mathbf{C}_i, \mathbf{C}_j)$ is
given by

\begin{equation}
d(\mathbf{C}_i, \mathbf{C}_j) = \sqrt{\sum_{k = 1}^p \ln^2 \lambda_k(\mathbf{C}_i\mathbf{C}_j^{-1})}
\label{eq:riemdist}
\end{equation}

where $\lambda_k(\cdot)$ refers to the $k$-th eigenvalue obtained from
the spectral decomposition of a given matrix, in this case the product
$\mathbf{C}_i\mathbf{C}_j^{-1}$. This distance among pairs of
$\mathbf{P}$-matrices is negatively correlated with Random Skewers
comparisons (\autoref{fig:rs_riem}), a measurement of matrix similarity
explored elsewhere (Cheverud \& Marroig, 2007); such correlation
indicates that our exploration of Riemmanian distances among
$\mathbf{P}$-matrices could be extended to other types of measurement of
similarity or dissimilarity among these objects without hindrances.

Using these distances among $\mathbf{P}$-matrices, we estimate matrix
diversity at each node of the phylogenetic tree of Anthropoidea using a
measurement of the weighted distance among the distributions of matrix
distances for the two descending edges, based on Pavoine et al. (2010).
For a fully resolved tree, diversity $w_i$ on node $i$ is given by

\begin{equation}
w_i = \frac{1}{2} \frac{n_\alpha n_\beta} {n_i n_T} D_{\Delta}^2(P_\alpha, P_\beta)
\label{eq:div}
\end{equation}

where $\alpha$ and $\beta$ represent the subsets of descendants from
node $i$, and $n$ refers to the number of OTUs on each set ($n_i$ for
the total descendants of node $i$; $n_T$ for the total number of OTUs
considered; $n_\alpha$ and $n_\beta$ for the size of descending
subsets).

$D_{\Delta}(P_\alpha, P_\beta)$ represents the actual distance between
the two distributions $P_\alpha$ and $P_\beta$ for descending nodes, as
formulated by Rao (1982):

\begin{equation}
D_{\Delta} (P_\alpha, P_\beta) =
\sqrt{2 \bigg( 2 H_{\Delta} \bigg( \frac{P_\alpha + P_\beta}{2} \bigg) -
H_{\Delta}(P_\alpha) - H_{\Delta}(P_\beta) \bigg)}
\label{eq:distdist}
\end{equation}

where

\begin{equation}
H_{\Delta} (P) = \sum_{i,j \in P} \frac{d^2(\mathbf{C}_i, \mathbf{C}_j)}{2}
\label{eq:rao}
\end{equation}

represents Rao's quadratic entropy among Riemannian distances
$d(\mathbf{C}_i, \mathbf{C}_j)$ as defined in \autoref{eq:riemdist}.

Following the framework estabilished by Pavoine et al. (2010), diversity
$w_i$ can be normalized as $v_i = w_i / \sum_i w_i$ to represent the
percentage of diversity with respect to the total diversity on the
phylogenetic tree. We test three different hypothesis regarding the
distribution of $v_i$ values through Anthropoid diversification: (1)
that $\mathbf{P}$-matrix diversity is concentrated in a single node; (2)
that $\mathbf{P}$-matrix diversity is concentrated in a reduced number
of nodes; (3) that $\mathbf{P}$-matrix diversity is skewed towards
either the root or tips of the phylogeny, in a two-tailed test. We test
each hypothesis against the null hypothesis that the distribution of
matrix diversity is randomly arranged over the phylogeny; such null
hypothesis is represented by randomizing the association between
terminal branches and covariance matrices, constructing $9999$
distributions of $v_i$ values that represent this scenario. Each test is
carried out using a different parameter derived from the distribution of
$v_i$ values (described in detail by Pavoine \emph{et al.}, 2010),
comparing the actual value obtained with a null distribution constructed
using permutations. The third hypothesis can be tested either by
considering only the topology of the tree and by also considering branch
lengths; both tests are similar to Blomberg's (2003) $K$ test, as they
search for a phylogenetic signal in covariance structure diversity.

\subsection{Characterizing Covariance Matrix
Variation}\label{characterizing-covariance-matrix-variation}

The tests described in the previous section allow us to pinpoint which
nodes contribute mostly to divergence in covariance structure; however,
these tests are not designed to properly describe the actual changes in
$\mathbf{P}$-matrix structure that are responsible for such divergence.
To actually represent such changes in a comprehensible manner, we
combine a number of ordination techniques to reduce the dimensionality
of the manifold that contains covariance matrices of size $p \times p$
(\autoref{fig:matrix_variation}).

For a Riemannian manifold, there exists at least one bijective function
defined in the neighbourhood of a given covariance matrix $\mathbf{M}$
that maps the manifold to an Euclidean space --- a hyperplane with
$p (p - 1) / 2$ dimensions also contained in $\mathbb{R}^{p \times p}$
--- and equips the manifold with a notion of inner product, thus
allowing the construction of an orthonormal basis that can be used to
describe variation in $\mathbf{P}$-matrix structure. For a covariance
matrix $\mathbf{X}$ in the neighbourhood of $\mathbf{M}$,

\begin{equation}
f(\mathbf{X}) = \log (\mathbf{M}^{- \frac{1}{2}} \mathbf{X} \mathbf{M}^{- \frac{1}{2}})
\label{eq:map}
\end{equation}

represents one possible function. Here, the logarithm operator refers to
matrix logarithm; for symmetric positive-definite matrices, this
transformation is equivalent to applying the usual logarithm function to
the eigenvalues of such matrix and reverting the spectral decomposition.
The function defined in \autoref{eq:map} also transforms the Riemannian
distance among covariance matrices defined in \autoref{eq:riemdist} into
Euclidean distances between transformed matrices (Moakher, 2005).

\begin{figure}[htbp]
\centering
\includegraphics{Figures/matrix_variation.png}
\caption{Representation of the steps used to characterize covariance
matrix variation. In (a), the set of covariance matrices $\mathbf{A}$,
$\mathbf{B}$ and $\mathbf{C}$ in the neighbourhood of $\mathbf{M}$ are
projected into an Euclidean space and eigentensors are estimated (PM1
and PM2); in (b), these eigentensors are rotated to incorporate
phylogenetic relatedness; in (c), covariance matrices corresponding to
the upper and lower bounds of the confidence intervals for each axis are
returned back to the original manifold. See text for more details.
\label{fig:matrix_variation}}
\end{figure}

We defined the average matrix among all sampled $\mathbf{P}$-matrices as
the location parameter $\mathbf{M}$ to map the entire set of posterior
$\mathbf{P}$-matrices into an Euclidean space. We then used these
$\mathbf{P}$-matrices to produce axes of matrix variation using an
eigentensor decomposition (Basser \& Pajevic, 2007; Hine \emph{et al.},
2009) obtaining a set of eigentensors and eigenvalues that summarise
matrix variation (\autoref{fig:matrix_variation}a). As a consequence of
using the mean covariance matrix for the entire sample over
\autoref{eq:map}, the projections over eigentensors we obtained are
naturally centered on $\mathbf{M}$.

We used the projections of $\mathbf{P}$-matrices over these eigentensors
as traits in a phylogenetic Principal Component Analysis (pPCA; Jombart
\emph{et al.}, 2010), which produces a new set of axes of matrix
variation that considers both trait dispersal and phylogenetic
relationships among OTUs simultaneously. If $\mathbf{Z}$ represents a
matrix with projections of $\mathbf{P}$-matrices over each eigentensor
on its columns, phylogenetic PCs are the eigenvectors obtained from

\begin{equation}
\frac{1}{2n} \mathbf{Z}^t(\mathbf{W} + \mathbf{W}^t) \mathbf{Z}
\end{equation}

where $\mathbf{W}$ represents the matrix of phylogenetic distances
between OTUs; here, the distance $w_{ij}$ between tips $i$ and $j$ is
the sum of branch lengths from their last common ancestor to both tips.
Such analysis produces both positive and negative eigenvalues, which are
respectively associated with variation close to the root of the tree
(`Global') and variation close to the tips (`Local') in matrix structure
(pPC1 and pPC2 in \autoref{fig:matrix_variation}b, respectively).
Pavoine et al. (2010) argues that this contrast between Global and Local
structures in phylogenetic PCs reflects phylogenetic signal and
convergence in trait values, respectively.

For each pPC obtained in this manner, we obtained two covariance
matrices by estimating the upper and lower limits of the 95\% confidence
interval for each axis and mapping these values back to the manifold of
symmetric positive-definite matrices (\autoref{fig:matrix_variation}c),
defining the inverse operation associated with \autoref{eq:map} as

\begin{equation}
f^{-1}(\mathbf{X}) = \mathbf{M}^{\frac{1}{2}}\exp(\mathbf{X})\mathbf{M}^{\frac{1}{2}}
\label{eq:inv}
\end{equation}

where the exponential operator refers to matrix exponential. We used
these covariance matrices to describe matrix variation associated with
each axis comparing each pair of matrices with the Selection Response
Decomposition tool (Marroig \emph{et al.}, 2011), in order to pinpoint
which traits are associated with the divergence in covariance structure
associated with each pPC.

In order to characterize such divergence in covariance structure with
respect to the uncertainty in $\mathbf{P}$-matrix estimation, we carried
out the analyses described in this section with both mean
$\mathbf{P}$-matrices obtained from posterior samples, and with
posterior samples themselves, obtaining $100$ sets of phylogenetic PCs
and $100$ sets of paired covariance matrices for each axis, thus
allowing us to estimate a posterior distribution of mean SRD scores for
each trait in all pPCs. We use the phylogenetic PCA estimated over mean
$\mathbf{P}$-matrices only to visualize the phylogenetic patterns
described by each pPC, although these patterns do not change in the
subset of posterior samples we examined.

We use the posterior distribution of mean SRD scores over traits and pPC
to investigate whether these changes in trait-specific covariance
structure along Anthropoid diversification are randomly distributed with
respect to the skull regions delimited in \autoref{tab:dist} by
comparing SRD scores estimated within each region for all pPCs.

\subsubsection{Software}\label{software}

We performed all analyses under R 3.2.2 (R Core Team, 2015). We fitted
Bayesian linear models for estimating posterior $\mathbf{P}$-matrix
samples using the \emph{MCMCglmm} package (Hadfield, 2010). Source code
for the eigentensor decomposition can be found at
\url{http://github.com/wgar84}. The phylogenetic decomposition of
diversity was provided by Pavoine et al. (2010) in their Supplemental
Material, while pPCA is implemented in the \emph{adephylo} package
(Jombart \& Dray, 2010); the SRD method is provided by the \emph{evolqg}
package (Melo \emph{et al.}, in prep.). In order to obtain symmetrical
landmarks configurations, we used code provided by Annat Haber,
available at \url{http://life.bio.sunysb.edu/morph/soft-R.html}.

\section{Results}\label{results}

Os testes de disparidade de estrutura de covariância ao longo da
filogenia (\autoref{tab:riem_decdiv}) indicam, em conjunto, que a maior
parte das diferenças em estrutura de covariância se localizam na base da
filogenia, associada à divergência entre Platyrrhini e Catarrhini
(\autoref{fig:phylo_decdiv}).

\begin{table}[ht]
  \centering
  \begin{threeparttable}
  \caption{Phylogenetic diversity test for covariance matrix structure. \label{tab:riem_decdiv}}
  \begin{tabular}{lrrrr}
    \toprule
    & \textbf{Value} & \textbf{Expected$^a$} & \textbf{Distance$^b$} & \textbf{P-value} \\ 
    \midrule
    Single Node & 0.106 & 0.029 & 13.455 & < $10^{-4}$ \\ 
    Few Nodes & 0.248 & 0.139 & 13.545 & < $10^{-4}$ \\ 
    Root/Tip Skewness$^c$ & 0.632 & 0.505 & 12.197 & < $10^{-4}$ \\ 
    Root/Tip Skewness$^d$ & 0.381 & 0.505 & -11.067 & < $10^{-4}$ \\ 
    \bottomrule
  \end{tabular}
  \begin{tablenotes}
    \footnotesize
    {
    \item[$a$] refers to the distribuition of permutated values;
    \item[$b$] difference between empirical and expected values in standard deviations of the permutated values distribution;
    \item[$c$] considering only topology;
    \item[$d$] including branch lengths.
    }
  \end{tablenotes}
\end{threeparttable}
\end{table}


\begin{figure}[htbp]
\centering
\includegraphics{Figures/phylo_full-1.pdf}
\caption{Phylogenetic decomposition of $\mathbf{P}$-matrix variation.
(a) Decomposition of matrix diversity over the phylogenetic hypothesis
for Anthropoidea; the size of each circle indicates the percentage of
diversity on each node, according to the legend. (b) Mean
$\mathbf{P}$-matrices of each species projected over the first and last
two phylogenetic Principal Components (G1-2 and L1-2, respectively);
cell colors represent projection values, according to the legend.
\label{fig:phylo_full}}
\end{figure}

A análise de componentes principais filogenéticos recupera um conjunto
de autovalores positivos que, em geral, possuem magnitudes maiores do
que suas contrapartes negativas (\autoref{fig:ppca_eval}). Isso indica
que a maior parte da divergência de estrutura de covariância está
localizada na divergência entre linhagens mais próximas da raiz da
filogenia do que entre espécies irmãs, em concordância com a análise de
disparidade.

\begin{figure}[htbp]
\centering
\includegraphics{Figures/ppca_eval-1.pdf}
\caption{Characterizing $\mathbf{P}$-matrix variation using phylogenetic
Principal Components. (a) Posterior distribution of eigenvalues obtained
for pPCs; positive eigenvalues are associated with phylogenetic signal;
negative eigenvalues are associated with convergence. (b) Distribution
of mean SRD scores for each trait with respect to pPCs; each line
represents a loess interpolation adjusted for each skull region,
according to the legend. \label{fig:ppca_eval}}
\end{figure}

Podemos observar que as projeções das espécies sobre cada componente
principal filogenético considerado (\autoref{fig:phylo_gl}) indicam que
o primeiro componente global está associado a um contraste entre
Platyrrhini e Catarrhini, enquanto o segundo componente global se
associa a contrastes dentre estas duas linhagens. Os componentes locais
representam diferenças locais entre espécies-irmãs, como por exemplo
entre as duas espécies de \emph{Gorilla} e entre as espécies do gênero
\emph{Callithrix}.

Ao reconstruírmos a variação em termos de estrutura de covariância
associada a estes componentes, a análise de SRD (Figuras
\ref{fig:srd_gl} e \ref{fig:srd_shape}) sobre os limites do intervalo de
confiança de 95\% de cada eixo indicam que em cada nível de divergência
(associados aos contrastes entre linhagens na pPCA) as maiores
diferenças em termos de estrutura de covariância se localizam dentre os
caracteres da Órbita e da Base, conforme indicado pelas médias menores e
desvios-padrão maiores na \autoref{fig:srd_gl}. Além dos caracteres
nestas duas regiões, a estrutura de covariância associada ao tamanho do
centróide também se mostrou divergente nos diferentes níveis delimitados
pela pPCA.

\begin{figure}[htbp]
\centering
\includegraphics{Figures/srd_gl-1.pdf}
\caption{Covariance structure variation associated with the first and
last two pPCs (Global 1-2 and Local 1-2, respectively), represented
using posterior mean SRD scores. Dotted lines represent average SRD
scores for each comparison. Traits are colored according to their
assocation with each skull region, according to the legend.
\label{fig:srd_gl}}
\end{figure}

\begin{figure}[htbp]
\centering
\includegraphics{Figures/srd_shape-1.pdf}
\caption{Mean SRD scores for the first and last two phylogenetic PCs
(Global 1-2 and Local 1-2, respectively) represented over the mean skull
shape of rhesus macaques (\emph{Macaca mulatta}). \label{fig:srd_shape}}
\end{figure}

\section{Discussion}\label{discussion}

Os resultados obtidos aqui indicam que a magnitude de disparidade em
estrutura de covariância para variáveis de forma e tamanho do centróide
está associada à distância filogenética (\autoref{tab:riem_decdiv},
\autoref{fig:phylo_decdiv}); entretanto, aqueles caracteres que
representam estas divergências em estrutura de covariância são os mesmo
independente da escala taxonômica considerada.

O Basicrânio é composto de um mosaico de células originadas de tecidos
distintos, que ossificam muito cedo durante o desenvolvimento (Lieberman
\emph{et al.}, 2008; Lieberman, 2011); ademais, os ossos do basicrânio
se associam topologicamente a grande maioria dos ossos do crânio
(Esteve-Altava \& Rasskin-Gutman, 2014), sofrendo, ao longo do
desenvolvimento, influência do crescimento e diferenciação dos ossos
circundantes. Em contrapartida, a estrutura de covariância associada ao
tamanho do centróide reflete principalmente o crescimento e
diferenciação pós-natal (Porto \emph{et al.}, 2013).

Assim sendo, mudanças na estrutura de covariância estão associadas a
eventos particulares do desenvolvimento, mostrando um padrão consistente
ao longo da diversificação dos primatas antropóides, refletindo o efeito
de interações funcionais e ontogenéticas.

\section*{References}\label{references}
\addcontentsline{toc}{section}{References}

Basser, P.J. \& Pajevic, S. 2007. Spectral decomposition of a 4th-order
covariance tensor: Applications to diffusion tensor MRI. \emph{Signal
Processing} \textbf{87}: 220--236.

Blomberg, S.P., Garland Jr., T. \& Ives, A.R. 2003. Testing for
phylogenetic signal in comparative data: behavioral traits are more
labile. \emph{Evolution} \textbf{57}: 717--745.

Cheverud, J.M. \& Marroig, G. 2007. Comparing covariance matrices:
Random skewers method compared to the common principal components model.
\emph{Genetics and Molecular Biology} \textbf{30}: 461--469.

Esteve-Altava, B. \& Rasskin-Gutman, D. 2014. Beyond the functional
matrix hypothesis: a network null model of human skull growth for the
formation of bone articulations. \emph{Journal of Anatomy} \textbf{225}:
306--316.

Gelman, A., Carlin, J.B., Stern, H.S. \& Rubin, D.B. 2004.
\emph{Bayesian data analysis}, 2ª ed. CRC Press, New York.

Haber, A. 2015. The Evolution of Morphological Integration in the
Ruminant Skull. \emph{Evolutionary Biology} \textbf{42}: 99--114.

Hadfield, J.D. 2010. MCMC Methods for Multi-Response Generalized Linear
Mixed Models: The \{MCMCglmm\} \{R\} Package. \emph{Journa of
Statistical Software} \textbf{33}: 1--22.

Hine, E., Chenoweth, S.F., Rundle, H.D. \& Blows, M.W. 2009.
Characterizing the evolution of genetic variance using genetic
covariance tensors. \emph{Philosophical transactions of the Royal
Society of London. Series B, Biological sciences} \textbf{364}:
1567--78.

Jombart, T. \& Dray, S. 2010. adephylo: exploratory analyses for the
phylogenetic comparative method. \emph{Bioinformatics} \textbf{26}:
1907--1909.

Jombart, T., Pavoine, S., Devillard, S. \& Pontier, D. 2010. Putting
phylogeny into the analysis of biological traits: A methodological
approach. \emph{Journal of Theoretical Biology} \textbf{264}: 693--701.

Klingenberg, C.P., Barluenga, M. \& Meyer, A. 2002. Shape analysis of
symmetric structures: Quantifying variation among individuals and
asymmetry. \emph{Evolution} \textbf{56}: 1909--1920.

Lieberman, D.E. 2011. Epigenetic Integration, Complexity and
Evolvability of the Head: Rethinking the Functional Matrix Hypothesis.
In: \emph{Epigenetics: Linking Genotype and Phenotype in Development and
Evolution} (B. Hallgrímsson \& B. K. Hall, eds), pp. 271--289.
University of California Press.

Lieberman, D.E., Hallgrímsson, B., Liu, W., Parsons, T.E. \& Jamniczky,
H.A. 2008. Spatial packing, cranial base angulation, and craniofacial
shape variation in the mammalian skull: testing a new model using mice.
\emph{Journal of Anatomy} \textbf{212}: 720--735.

Marroig, G. \& Cheverud, J.M. 2001. A comparison of phenotypic variation
and covariation patterns and the role of phylogeny, ecology, and
ontogeny during cranial evolution of new world monkeys. \emph{Evolution}
\textbf{55}: 2576--2600.

Marroig, G., Melo, D., Porto, A., Sebastião, H. \& Garcia, G. 2011.
Selection Response Decomposition (SRD): A New Tool for Dissecting
Differences and Similarities Between Matrices. \emph{Evolutionary
Biology} \textbf{38}: 225--241.

Márquez, E.J., Cabeen, R., Woods, R.P. \& Houle, D. 2012. The
Measurement of Local Variation in Shape. \emph{Evolutionary Biology}
\textbf{39}: 419--439.

Mitteroecker, P. \& Bookstein, F.L. 2009. The ontogenetic trajectory of
the phenotypic covariance matrix, with examples from craniofacial shape
in rats and humans. \emph{Evolution} \textbf{63}: 727--737.

Moakher, M. 2005. A differential geometric approach to the geometric
mean of symmetric positive-definite matrices. \emph{SIAM journal on
matrix analysis and applications} \textbf{26}: 735--747.

Moakher, M. 2006. On the Averaging of Symmetric Positive-Definite
Tensors. \emph{Journal of Elasticity} \textbf{82}: 273--296.

Oliveira, F.B., Porto, A. \& Marroig, G. 2009. Covariance structure in
the skull of Catarrhini: a case of pattern stasis and magnitude
evolution. \emph{Journal of Human Evolution} \textbf{56}: 417--430.

Pavoine, S., Baguette, M. \& Bonsall, M.B. 2010. Decomposition of trait
diversity among the nodes of a phylogenetic tree. \emph{Ecological
Monographs} \textbf{80}: 485--507.

Porto, A., Shirai, L.T., Oliveira, F.B. de \& Marroig, G. 2013. Size
Variation, Growth Strategies, and the Evolution of Modularity in the
Mammalian Skull. \emph{Evolution} \textbf{67}: 3305--3322.

R Core Team. 2015. \emph{R: A Language and Environment for Statistical
Computing}. R Foundation for Statistical Computing, Vienna, Austria.

Rao, C.R. 1982. Diversity and dissimilarity coefficients: a unified
approach. \emph{Theoretical population biology} \textbf{21}: 24--43.

Springer, M.S., Meredith, R.W., Gatesy, J., Emerling, C.A., Park, J. \&
Rabosky, D.L.\emph{et al.} 2012. Macroevolutionary Dynamics and
Historical Biogeography of Primate Diversification Inferred from a
Species Supermatrix. \emph{PLoS ONE} \textbf{7}: e49521.

\end{document}
