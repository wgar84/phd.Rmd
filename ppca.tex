\documentclass[11pt,]{article}
\usepackage{lmodern}
\usepackage{amssymb,amsmath}
\usepackage{ifxetex,ifluatex}
\usepackage{fixltx2e} % provides \textsubscript
\ifnum 0\ifxetex 1\fi\ifluatex 1\fi=0 % if pdftex
  \usepackage[T1]{fontenc}
  \usepackage[utf8]{inputenc}
\else % if luatex or xelatex
  \ifxetex
    \usepackage{mathspec}
    \usepackage{xltxtra,xunicode}
  \else
    \usepackage{fontspec}
  \fi
  \defaultfontfeatures{Mapping=tex-text,Scale=MatchLowercase}
  \newcommand{\euro}{€}
\fi
% use upquote if available, for straight quotes in verbatim environments
\IfFileExists{upquote.sty}{\usepackage{upquote}}{}
% use microtype if available
\IfFileExists{microtype.sty}{%
\usepackage{microtype}
\UseMicrotypeSet[protrusion]{basicmath} % disable protrusion for tt fonts
}{}
\usepackage[margin=1in]{geometry}
\usepackage{graphicx}
\makeatletter
\def\maxwidth{\ifdim\Gin@nat@width>\linewidth\linewidth\else\Gin@nat@width\fi}
\def\maxheight{\ifdim\Gin@nat@height>\textheight\textheight\else\Gin@nat@height\fi}
\makeatother
% Scale images if necessary, so that they will not overflow the page
% margins by default, and it is still possible to overwrite the defaults
% using explicit options in \includegraphics[width, height, ...]{}
\setkeys{Gin}{width=\maxwidth,height=\maxheight,keepaspectratio}
\ifxetex
  \usepackage[setpagesize=false, % page size defined by xetex
              unicode=false, % unicode breaks when used with xetex
              xetex]{hyperref}
\else
  \usepackage[unicode=true]{hyperref}
\fi
\hypersetup{breaklinks=true,
            bookmarks=true,
            pdfauthor={},
            pdftitle={A Phylogenetic Analysis of Covariance Structure in Anthropoids},
            colorlinks=true,
            citecolor=blue,
            urlcolor=blue,
            linkcolor=magenta,
            pdfborder={0 0 0}}
\urlstyle{same}  % don't use monospace font for urls
\setlength{\parindent}{0pt}
\setlength{\parskip}{6pt plus 2pt minus 1pt}
\setlength{\emergencystretch}{3em}  % prevent overfull lines
\setcounter{secnumdepth}{0}

%%% Use protect on footnotes to avoid problems with footnotes in titles
\let\rmarkdownfootnote\footnote%
\def\footnote{\protect\rmarkdownfootnote}

%%% Change title format to be more compact
\usepackage{titling}

% Create subtitle command for use in maketitle
\newcommand{\subtitle}[1]{
  \posttitle{
    \begin{center}\large#1\end{center}
    }
}

\setlength{\droptitle}{-2em}
  \title{A Phylogenetic Analysis of Covariance Structure in Anthropoids}
  \pretitle{\vspace{\droptitle}\centering\huge}
  \posttitle{\par}
  \author{}
  \preauthor{}\postauthor{}
  \date{}
  \predate{}\postdate{}

%--- USEPACKAGES ---
\usepackage[utf8]{inputenc}
\usepackage{natbib}
\usepackage[brazilian,english]{babel}
\usepackage{hyperref}
\usepackage{subfigure, epsfig}
\usepackage{ae}
\usepackage{aecompl}
\usepackage{booktabs}
\usepackage[T1]{fontenc}
\usepackage{graphicx,wrapfig} % para incluir figuras
\usepackage{amsfonts, amssymb,amsthm, amsmath, amscd} % pacote AMS
\usepackage{color, bbm, multicol}
\usepackage{verbatim, listings, booktabs}
\usepackage{fancyhdr} % FANCYHEADER
\usepackage{setspace}
\usepackage{times} % bookman,palatino,courier,times FONTES
\usepackage{lineno}
\usepackage{url}
\usepackage{rotating}
\usepackage{longtable}
% \usepackage{rotfloat}
% \usepackage{appendix}
\usepackage{mathptmx}
% \usepackage{cmbright}
% \usepackage[adobe-utopia]{mathdesign}
\usepackage[flushleft]{threeparttable}
\usepackage{multirow}
% \usepackage{float}
\usepackage {tocvsec2} % controlar profundidade de table of contents
\setcounter {secnumdepth}{0}
\usepackage {caption}
\usepackage {tabularx}
\usepackage {floatrow}
\floatsetup[table]{capposition=top}

\usepackage{mathpazo} % fonte palatino

\newcommand{\barra}{\backslash}
\newcommand{\To}{\longrightarrow}
\newcommand{\abs}[1]{\left\vert#1\right\vert}
\newcommand{\set}[1]{\left\{#1\right\}}
\newcommand{\seq}[1]{\left<#1\right>}
\newcommand{\norma}[1]{\left\Vert#1\right\Vert}
\newcommand{\hr}{\par\noindent\hrulefill\par}

\usepackage {xr}
\externaldocument{sup_base}

\selectlanguage{english}

\hypersetup{colorlinks=false}


\begin{document}

\maketitle


\linenumbers
\modulolinenumbers[2]

\onehalfspacing

\section{Introdução}\label{introducao}

Sistemas morfológicas tendem a apresentar covariação entre seus
componentes devido à interações durante o desenvolvimento. No crânio de
mamíferos, por exemplo, fenótipos adultos são compostos pela
sobreposição de diferentes processos de desenvolvimento; portanto, a
variação no tempo, ritmo e escopo de tais processos estrutura os padrões
de covariância entre caracteres cranianos em adultos. Esta estrutura de
covariância em fenótipos adultos é também influenciada pelos efeitos da
canalização e seleção estabilizadora, que podem limitar a variação nos
processos de desenvolvimento. Portanto, espera-se que as alterações na
estrutura de covariância entre linhagens irmãs irão exibir um padrão
não-aleatório consistente com estas interações de desenvolvimento e
funcionais.

\section{Métodos}\label{metodos}

Nós investigamos a evolução da estrutura de covariância de tamanho e
forma cranianos em primatas antropóides em um contexto filogenético. Nós
representamos esta estrutura de covariância para cada espécie,
utilizando matrizes de covariância para o log Tamanho do Centróide e
variáveis locais de forma (Márquez \emph{et al.}, 2012). Estimamos
distâncias Riemannianas entre essas matrizes (Mitteroecker \& Bookstein,
2009), e sujeitamos estas distâncias a uma decomposição de disparidade
em estrutura de covariância ao longo da filogenia(Pavoine \emph{et al.},
2010). Usando um procedimento de randomização, avaliamos uma série de
hipóteses sobre a distribuição desta disparidade ao longo da
diversificação dos antropóides. Além disso, a fim de descrever
adequadamente a variação em matrizes de covariância, usamos uma
decomposição em autotensores (Hine \emph{et al.}, 2009; Aguirre \emph{et
al.}, 2013). Em seguida, usamos projeções da matriz sobre os
autotensores como caracteres em uma Análise de Componentes Principais
filogenética (pPCA; Jombart \emph{et al.}, 2010), que decompõe estes
caracteres em eixos que estão associados a disparidade global (perto de
raiz) e local (perto do topo). Para cada eixo recuperado pela pPCA,
reconstruímos duas matrizes de covariância, associadas aos limites
inferior e superior do intervalo de confiança de 95\%. Nós sujeitamos
cada um destes pares de matrizes à decomposição de resposta a seleção
(SRD; Marroig \emph{et al.}, 2011), a fim de avaliar quais caracteres
são associados à divergência na estrutura de covariância para cada nível
hierárquico na filogenia, definidos pelos componentes principais
filogenéticos.

\section{Resultados}\label{resultados}

Os testes de disparidade de estrutura de covariância ao longo da
filogenia (\autoref{tab:riem_decdiv}) indicam, em conjunto, que a maior
parte das diferenças em estrutura de covariância se localizam na base da
filogenia, associada à divergência entre Platyrrhini e Catarrhini
(\autoref{fig:phylo_decdiv}).

\begin{table}[ht]
  \centering
  \begin{threeparttable}
  \caption{Phylogenetic diversity test for covariance matrix structure. \label{tab:riem_decdiv}}
  \begin{tabular}{lrrrr}
    \toprule
    & \textbf{Value} & \textbf{Expected$^a$} & \textbf{Distance$^b$} & \textbf{P-value} \\ 
    \midrule
    Single Node & 0.106 & 0.029 & 13.455 & < $10^{-4}$ \\ 
    Few Nodes & 0.248 & 0.139 & 13.545 & < $10^{-4}$ \\ 
    Root/Tip Skewness$^c$ & 0.632 & 0.505 & 12.197 & < $10^{-4}$ \\ 
    Root/Tip Skewness$^d$ & 0.381 & 0.505 & -11.067 & < $10^{-4}$ \\ 
    \bottomrule
  \end{tabular}
  \begin{tablenotes}
    \footnotesize
    {
    \item[$a$] refers to the distribuition of permutated values;
    \item[$b$] difference between empirical and expected values in standard deviations of the permutated values distribution;
    \item[$c$] considering only topology;
    \item[$d$] including branch lengths.
    }
  \end{tablenotes}
\end{threeparttable}
\end{table}


\begin{figure}[htbp]
\centering
\includegraphics{Figures/phylo_decdiv-1.pdf}
\caption{Decomposição de diversidade filogenética em respeito à
estrutura de covariância utilizando distâncias Riemannianas. O tamanho
do círculo indica o valor de diversidade associada cada nó, de acordo
com a legenda. \label{fig:phylo_decdiv}}
\end{figure}

A análise de componentes principais filogenéticos recupera um conjunto
de autovalores positivos que, em geral, possuem magnitudes maiores do
que suas contrapartes negativas (\autoref{fig:ppca_eval}). Isso indica
que a maior parte da divergência de estrutura de covariância está
localizada na divergência entre linhagens mais próximas da raiz da
filogenia do que entre espécies irmãs, em concordância com a análise de
disparidade.

\begin{figure}[htbp]
\centering
\includegraphics{Figures/ppca_eval-1.pdf}
\caption{Autovalores da decomposição espectral associada à Análise de
Componentes Principais Filogenéticos. No painel superior, barras indicam
o valor associado à cada autovalor; barras em cinza indicam autovalores
cujos autovetores associados foram considerados nas análises
subsequentes. No painel inferior, gráfico de dispersão entre a variância
associada a cada autovetor e o índice de autocorrelação filogenética de
Moran. \label{fig:ppca_eval}}
\end{figure}

Podemos observar que as projeções das espécies sobre cada componente
principal filogenético considerado (\autoref{fig:phylo_gl}) indicam que
o primeiro componente global está associado a um contraste entre
Platyrrhini e Catarrhini, enquanto o segundo componente global se
associa a contrastes dentre estas duas linhagens. Os componentes locais
representam diferenças locais entre espécies-irmãs, como por exemplo
entre as duas espécies de \emph{Gorilla} e entre as espécies do gênero
\emph{Callithrix}.

\begin{figure}[htbp]
\centering
\includegraphics{Figures/phylo_gl-1.pdf}
\caption{Projeções da variação de estrutura de covariância de cada
espécie sobre os Componentes Principais Filogenéticos considerados,
segundo a \autoref{fig:ppca_eval}. Círculos azuis e vermelhos indicam
valores positivos e negativos, respectivamente. A magnitude de cada
projeção é indicada pelo tamanho do círculo, de acordo com a legenda.
\label{fig:phylo_gl}}
\end{figure}

Ao reconstruírmos a variação em termos de estrutura de covariância
associada a estes componentes, a análise de SRD (Figuras
\ref{fig:srd_gl} e \ref{fig:srd_shape}) sobre os limites do intervalo de
confiança de 95\% de cada eixo indicam que em cada nível de divergência
(associados aos contrastes entre linhagens na pPCA) as maiores
diferenças em termos de estrutura de covariância se localizam dentre os
caracteres da Órbita e da Base, conforme indicado pelas médias menores e
desvios-padrão maiores na \autoref{fig:srd_gl}. Além dos caracteres
nestas duas regiões, a estrutura de covariância associada ao tamanho do
centróide também se mostrou divergente nos diferentes níveis delimitados
pela pPCA.

\begin{figure}[htbp]
\centering
\includegraphics{Figures/srd_gl-1.pdf}
\caption{Comparações utilizando o método SRD dos limites associados ao
intervalo de confiança de 95\% cada Componente Principal Filogenético
considerado, segundo a \autoref{fig:ppca_eval}. \label{fig:srd_gl}}
\end{figure}

\begin{figure}[htbp]
\centering
\includegraphics{Figures/srd_shape-1.pdf}
\caption{Comparações utilizando o método SRD representados sobre a forma
craniana de um macaco rhesus (\emph{Macaca mulatta}). Cores próximas ao
vermelho indicam caracteres que divergem mais em relação à estrutura de
covariância em cada comparação; cores próximas ao verde indicam pouca
divergência. \label{fig:srd_shape}}
\end{figure}

\section{Discussão}\label{discussao}

Os resultados obtidos aqui indicam que a magnitude de disparidade em
estrutura de covariância para variáveis de forma e tamanho do centróide
está associada à distância filogenética (\autoref{tab:riem_decdiv},
\autoref{fig:phylo_decdiv}); entretanto, aqueles caracteres que
representam estas divergências em estrutura de covariância são os mesmo
independente da escala taxonômica considerada.

O Basicrânio é composto de um mosaico de células originadas de tecidos
distintos, que ossificam muito cedo durante o desenvolvimento (Lieberman
\emph{et al.}, 2008; Lieberman, 2011); ademais, os ossos do basicrânio
se associam topologicamente a grande maioria dos ossos do crânio
(Esteve-Altava \& Rasskin-Gutman, 2014), sofrendo, ao longo do
desenvolvimento, influência do crescimento e diferenciação dos ossos
circundantes. Em contrapartida, a estrutura de covariância associada ao
tamanho do centróide reflete principalmente o crescimento e
diferenciação pós-natal (Porto \emph{et al.}, 2013).

Assim sendo, mudanças na estrutura de covariância estão associadas a
eventos particulares do desenvolvimento, mostrando um padrão consistente
ao longo da diversificação dos primatas antropóides, refletindo o efeito
de interações funcionais e ontogenéticas.

Aguirre, J.D., Hine, E., McGuigan, K. \& Blows, M.W. 2013. Comparing G :
multivariate analysis of genetic variation in multiple populations.
\emph{Heredity} \textbf{112}: 21--29.

Esteve-Altava, B. \& Rasskin-Gutman, D. 2014. Beyond the functional
matrix hypothesis: a network null model of human skull growth for the
formation of bone articulations. \emph{Journal of Anatomy} \textbf{225}:
306--316.

Hine, E., Chenoweth, S.F., Rundle, H.D. \& Blows, M.W. 2009.
Characterizing the evolution of genetic variance using genetic
covariance tensors. \emph{Philosophical transactions of the Royal
Society of London. Series B, Biological sciences} \textbf{364}:
1567--78.

Jombart, T., Pavoine, S., Devillard, S. \& Pontier, D. 2010. Putting
phylogeny into the analysis of biological traits: A methodological
approach. \emph{Journal of Theoretical Biology} \textbf{264}: 693--701.

Lieberman, D.E. 2011. Epigenetic Integration, Complexity and
Evolvability of the Head: Rethinking the Functional Matrix Hypothesis.
In: \emph{Epigenetics: Linking Genotype and Phenotype in Development and
Evolution} (B. Hallgrímsson \& B. K. Hall, eds), pp. 271--289.
University of California Press.

Lieberman, D.E., Hallgrímsson, B., Liu, W., Parsons, T.E. \& Jamniczky,
H.A. 2008. Spatial packing, cranial base angulation, and craniofacial
shape variation in the mammalian skull: testing a new model using mice.
\emph{Journal of Anatomy} \textbf{212}: 720--735.

Marroig, G., Melo, D., Porto, A., Sebastião, H. \& Garcia, G. 2011.
Selection Response Decomposition (SRD): A New Tool for Dissecting
Differences and Similarities Between Matrices. \emph{Evolutionary
Biology} \textbf{38}: 225--241.

Márquez, E.J., Cabeen, R., Woods, R.P. \& Houle, D. 2012. The
Measurement of Local Variation in Shape. \emph{Evolutionary Biology}
\textbf{39}: 419--439.

Mitteroecker, P. \& Bookstein, F.L. 2009. The ontogenetic trajectory of
the phenotypic covariance matrix, with examples from craniofacial shape
in rats and humans. \emph{Evolution} \textbf{63}: 727--737.

Pavoine, S., Baguette, M. \& Bonsall, M.B. 2010. Decomposition of trait
diversity among the nodes of a phylogenetic tree. \emph{Ecological
Monographs} \textbf{80}: 485--507.

Porto, A., Shirai, L.T., Oliveira, F.B. de \& Marroig, G. 2013. Size
Variation, Growth Strategies, and the Evolution of Modularity in the
Mammalian Skull. \emph{Evolution} \textbf{67}: 3305--3322.

\end{document}
