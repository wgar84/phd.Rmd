\documentclass[11pt,]{article}
\usepackage{lmodern}
\usepackage{amssymb,amsmath}
\usepackage{ifxetex,ifluatex}
\usepackage{fixltx2e} % provides \textsubscript
\ifnum 0\ifxetex 1\fi\ifluatex 1\fi=0 % if pdftex
  \usepackage[T1]{fontenc}
  \usepackage[utf8]{inputenc}
\else % if luatex or xelatex
  \ifxetex
    \usepackage{mathspec}
    \usepackage{xltxtra,xunicode}
  \else
    \usepackage{fontspec}
  \fi
  \defaultfontfeatures{Mapping=tex-text,Scale=MatchLowercase}
  \newcommand{\euro}{€}
\fi
% use upquote if available, for straight quotes in verbatim environments
\IfFileExists{upquote.sty}{\usepackage{upquote}}{}
% use microtype if available
\IfFileExists{microtype.sty}{%
\usepackage{microtype}
\UseMicrotypeSet[protrusion]{basicmath} % disable protrusion for tt fonts
}{}
\usepackage[margin=1in]{geometry}
\usepackage{graphicx}
\makeatletter
\def\maxwidth{\ifdim\Gin@nat@width>\linewidth\linewidth\else\Gin@nat@width\fi}
\def\maxheight{\ifdim\Gin@nat@height>\textheight\textheight\else\Gin@nat@height\fi}
\makeatother
% Scale images if necessary, so that they will not overflow the page
% margins by default, and it is still possible to overwrite the defaults
% using explicit options in \includegraphics[width, height, ...]{}
\setkeys{Gin}{width=\maxwidth,height=\maxheight,keepaspectratio}
\ifxetex
  \usepackage[setpagesize=false, % page size defined by xetex
              unicode=false, % unicode breaks when used with xetex
              xetex]{hyperref}
\else
  \usepackage[unicode=true]{hyperref}
\fi
\hypersetup{breaklinks=true,
            bookmarks=true,
            pdfauthor={},
            pdftitle={Allometric Constraints and Modularity},
            colorlinks=true,
            citecolor=blue,
            urlcolor=blue,
            linkcolor=magenta,
            pdfborder={0 0 0}}
\urlstyle{same}  % don't use monospace font for urls
\setlength{\parindent}{0pt}
\setlength{\parskip}{6pt plus 2pt minus 1pt}
\setlength{\emergencystretch}{3em}  % prevent overfull lines
\setcounter{secnumdepth}{0}

%%% Use protect on footnotes to avoid problems with footnotes in titles
\let\rmarkdownfootnote\footnote%
\def\footnote{\protect\rmarkdownfootnote}

%%% Change title format to be more compact
\usepackage{titling}

% Create subtitle command for use in maketitle
\newcommand{\subtitle}[1]{
  \posttitle{
    \begin{center}\large#1\end{center}
    }
}

\setlength{\droptitle}{-2em}
  \title{Allometric Constraints and Modularity}
  \pretitle{\vspace{\droptitle}\centering\huge}
  \posttitle{\par}
  \author{}
  \preauthor{}\postauthor{}
  \date{}
  \predate{}\postdate{}

%--- USEPACKAGES ---
\usepackage[utf8]{inputenc}
\usepackage{natbib}
\usepackage[brazilian,english]{babel}
\usepackage{hyperref}
\usepackage{subfigure, epsfig}
\usepackage{ae}
\usepackage{aecompl}
\usepackage{booktabs}
\usepackage[T1]{fontenc}
\usepackage{graphicx,wrapfig} % para incluir figuras
\usepackage{amsfonts, amssymb,amsthm, amsmath, amscd} % pacote AMS
\usepackage{color, bbm, multicol}
\usepackage{verbatim, listings, booktabs}
\usepackage{fancyhdr} % FANCYHEADER
\usepackage{setspace}
\usepackage{times} % bookman,palatino,courier,times FONTES
\usepackage{lineno}
\usepackage{url}
\usepackage{rotating}
\usepackage{longtable}
% \usepackage{rotfloat}
% \usepackage{appendix}
\usepackage{mathptmx}
% \usepackage{cmbright}
% \usepackage[adobe-utopia]{mathdesign}
\usepackage[flushleft]{threeparttable}
\usepackage{multirow}
% \usepackage{float}
\usepackage {tocvsec2} % controlar profundidade de table of contents
\setcounter {secnumdepth}{0}
\usepackage {caption}
\usepackage {tabularx}
\usepackage {floatrow}
\floatsetup[table]{capposition=top}

\usepackage{mathpazo} % fonte palatino

\newcommand{\barra}{\backslash}
\newcommand{\To}{\longrightarrow}
\newcommand{\abs}[1]{\left\vert#1\right\vert}
\newcommand{\set}[1]{\left\{#1\right\}}
\newcommand{\seq}[1]{\left<#1\right>}
\newcommand{\norma}[1]{\left\Vert#1\right\Vert}
\newcommand{\hr}{\par\noindent\hrulefill\par}

\usepackage {xr}
\externaldocument{sup_base}

\selectlanguage{english}

\hypersetup{colorlinks=false}


\begin{document}

\maketitle


\linenumbers
\modulolinenumbers[2]

\onehalfspacing

\section{Introduction}\label{introduction}

A fundamental feature of morphological systems is their tendency to
exhibit correlations due to commom developmental processes and
functional interactions, a phenomenon called morphological integration
(Olson \& Miller, 1958; Cheverud, 1996a; Hallgrímsson \emph{et al.},
2009). Integration can be characterized both by the magnitude of
correlation between morphological traits and the pattern described by
the inter-trait correlation structure (Marroig \& Cheverud, 2004; Porto
\emph{et al.}, 2013). In mammal morphological systems, traditional
morphometrics analysis of integration emphasize the role of size
variation in determining magnitude of morphological integration and,
through allometric relationships, it is argued that size can also affect
integration patterns (Porto \emph{et al.}, 2009, 2013).

Esta variação de tamanho emerge a partir do processo de crescimento;
ademais, o crescimento heterogêneo de partes distintas em uma dada
estrutura morfológica produz alometria, isto é, a associação entre
qualquer caráter organísmico de interesse e tamanho corporal (Huxley,
1932; Pélabon \emph{et al.}, 2014). É possível definir diferentes tipos
de alometria de acordo com o nível de organização biológica considerado
(Pélabon \emph{et al.}, 2013). Assim, alometria ontogénetica refere-se à
associação entre caracteres e tamanho corpóreo ao longo do
desenvolvimento de um indivíduo; alometria estática refere-se à esta
associação medida em uma população, entre indivíduos em uma mesma faixa
etária; e alometria evolutiva, medida entre médias em populações ou
espécies diferentes. Associações alométricas são classicamente medidas
na escala log (Huxley, 1932; Jolicoeur, 1963); portanto, estabelece-se
uma relação alométrica log-linear entre o caráter $x$ e o tamanho
corporal $m$ da forma \[
\log x = a + b \log m
\] onde $a$ representa o intercepto alométrico e $b$ representa a
inclinação de reta representativa da associação entre $x$ e $m$. Esta
equação representa a relação entre $x$ e $m$ em um nível de organização
arbitrário; é possível definir interceptos e inclinações estáticos
($a_s$; $b_s$), ontogenéticos ($a_o$; $b_o$) ou evolutivos ($a_e$;
$b_e$; Pélabon \emph{et al.}, 2013; Voje \emph{et al.}, 2013).

Due to the entanglement between size and shape in representing
morphological variation using euclidean distances, traditional
morphometrics struggle in representing allometry properly (Bookstein,
1989; Swiderski, 2003). In this respect, geometric morphometrics
(Bookstein, 1991; Zelditch \emph{et al.}, 2004) are a proper tool to
evaluate allometric relationships. However, due to the influence of the
Procrustes superimposition (Walker, 2000; Linde \& Houle, 2009),
landmark-based geometric morphometrics are unable to provide good
estimates for shape covariance patterns.

Considering both the spatial and temporal dynamics of mammalian cranial
development, we expect that changes in allometric parameters within
populations will contribute to variation in the strength of integration
within Facial and Neurocranial traits, albeit in opposing ways, due to
the distinction between early- and late-developmental factors. In order
to test this hypothesis, we use a combined approach using both geometric
morphometrics to estimate allometric parameters, and traditional
morphometrics to evaluate morphological integration.

Using a Bayesian phylogenetic random regression model of allometric
shape against size, we estimate static allometric parameters along the
phylogeny of Anthropoidea. We show that allometric slopes have remained
quite stable through the diversification of anthropoids, except for
\emph{Homo} and \emph{Gorilla}, while allometric intercepts have
suffered changes in many different branches, such as Callithrichids and
\emph{Alouatta}. Using Bayesian phylogenetic regressions, we also
demonstrate that allometric slope variation among taxa is responsible
for increased Facial integration and decreased Neurocranial integration,
indicating that differences in the timing and rate of development,
mediated by the spatio-temporal dynamics of the genetic developmental
network are paramount to understanding the evolution of morphological
integration.

\section{Methods}\label{methods}

\subsection{Sample}\label{sample}

Our database consists of 5108 individuals, distributed across 109
species. These species are spread throughout all major Anthropoid clades
above the genus level, also comprising all Platyrrhini genera and a
substantial portion of Catarrhini genera. We associate this database
with a ultrametric phylogenetic hypothesis for Anthropoidea (Figure
\ref{sfig:phylo_model}), derived from Springer \emph{et al}.(Springer
\emph{et al.}, 2012).

Individuals in our sample are represented by 36 registered landmarks,
using either a Polhemus 3Draw or a Microscribe 3DS for Platyrrhini and
Catarrhini, respectively. Twenty-two unique landmarks represent each
individual (Figure \ref{sfig:landmarks}), since fourteen of the 36
registered landmarks are bilaterally symmetrical. For more details on
landmark registration, see Marroig \& Cheverud (2001) and Oliveira et
al. (2009). Databases from both previous studies were merged into a
single database, retaining only those individuals in which all
landmarks, from both sides, were present.

This database of landmark registration data was used to estimate both
interlandmark distances and shape variables. For interlandmark
distances, those measurements that involve bilaterally symmetrical
landmarks were averaged after computing distances; for shape variables,
a symmetrical landmark configuration was obtained by taking the mean
shape between each individual configuration and its reflection along the
sagittal plane (Klingenberg \emph{et al.}, 2002).

\subsection{Allometric Slopes and
Intercepts}\label{allometric-slopes-and-intercepts}

We estimated allometric parameters using local shape variables (Márquez
\emph{et al.}, 2012), which are measurements of infinitesimal log volume
transformations, calculated as the natural logarithm determinants of
derivatives of the TPS function between each individual in our sample
and a reference shape (in our case, the mean shape for the entire
sample, estimated from a Generalized Procrustes algorithm). Such
derivatives were evaluated at 38 locations, and we chose these locations
to match our interlandmark distance database; therefore, these locations
are the midpoints between pairs of landmarks used to calculate
interlandmark distances (Figure \ref{sfig:landmarks}). After estimating
local shape variables, sources of variation of little interest in the
present context, such as sexual dimorphism and variation between
subspecies or populations were controlled within each OTU, according to
Figure \ref{sfig:phylo_model}, using generalized linear models.

We used a phylogenetic random regression model under a Bayesian
framework in order to estimate static allometric intercepts ($a_s$) and
slopes ($b_s$) for all OTUs simultaneously while considering their
phylogenetic structure. This model assumes that both $a_s$ and $b_s$
evolve under Brownian motion, as both parameters are defined as random
variables with a correlation structure among OTUs derived from the
phylogenetic hypothesis. This allows the model to estimate $a_s$ and
$b_s$ for each terminal OTU and also for ancestral nodes, enabling us to
track changes in both parameters along the phylogeny.

We projected all individuals in our sample along the Common Allometric
Component (CAC; Mitteroecker \emph{et al.}, 2004), which is the pooled
within-species slopes between local shape variables and log Centroid
Size (logCS). The CAC summarizes all allometric shape variation, and we
regress this single variable against logCS in our random regression
model. This reduction in dimensionality is necessary because a full
multivariate random regression model using all 38 shape variables would
be computationally untractable, considering the actual state of MCMC
samplers available. Therefore, we limit ourselves to test whether the
strength of association between shape and size with respect to the CAC
(which we consider the best representation of the ancestral allometric
shape variation) has changed during the diversification of anthropoid
primates; in order to test whether the direction of allometric shape
variation has changed during anthropoid diversification, a full
multivariate random regression model would be necessary.

We used uniform prior distributions for all $a_s$ and $b_s$. In order to
sample the posterior distribution for our model, we used a MCMC sampler
with $100000$ iterations, comprising a burnin period of $50000$
iterations and a thinning interval of $50$ iterations after burnin to
avoid autocorrelations in the posterior sample, thus generating $1000$
posterior samples for all parameters we estimate. We performed a handful
of runs with different starting values and pseudo-random number
generator seeds to ensure convergence; with these values for iteration
steps, we achieved convergence in all MCMC runs.

Using these posterior samples for static allometric parameters, we test
whether a given intercept or slope in any node of the tree (terminal or
ancestral) deviates from the values estimated for the root of the tree
by computing 95\% credible intervals for the difference between the
parameter estimated at both points. If this interval excludes the null
value, we consider that as evidence that the parameter ($a_s$ or $b_s$)
has changed in that particular node.

\subsection{Morphological Integration}\label{morphological-integration}

Using our landmark configuration database, we calculated 38
interlandmark distances, based on previous works on mammalian covariance
patterns (Cheverud, 1995, 1996b; Marroig \& Cheverud, 2001; Oliveira
\emph{et al.}, 2009) (Figure \ref{sfig:landmarks}). Within each OTU, we
estimate and remove those fixed effects of little interest within the
present context using generalized linear models (Figure
\ref{sfig:phylo_model}). Using residuals from each model, we estimated
covariance and correlation $P$-matrices for all OTUs.

We estimate magnitude of morphological integration for each OTU using
the coefficient of variation of the eigenvalues (ICV(Shirai \& Marroig,
2010; Porto \emph{et al.}, 2013)), which is a scale-free measure of the
overall strength of trait association in a given covariance matrix.

Estimamos magnitude geral de integração morfológica para cada espécie
usando o coeficiente de variação dos autovalores da matriz de
covariância (ICV; Shirai \& Marroig, 2010; Porto \emph{et al.}, 2013),
que é uma mensuração da força de associação entre caracteres
independente de escala. De maneira a representar a modularidade
associada aos conjuntos de caracteres localizados nas regiões e
sub-regiões definidas (\autoref{tab:dist}), estimamos o índice AVG
(Porto \emph{et al.}, 2013), que é definido como a diferença entre a
média das correlações entre elementos de um conjunto e a média do
correlações do conjunto complementar, dividida pelo ICV associado. Desta
forma, o índice AVG captura a força de associação entre caracteres em
uma região em relação à associação entre os demais caracteres e também
em relação à magnitude global de integração na matriz.

Para avaliar a relação entre parâmetros alométricos e índices AVG,
ajustamos modelos lineares controlados para efeitos filogenéticos
utilizando os índices AVG separadamente como variáveis resposta e os
parâmetros alométricos como variáveis preditoras; tais modelos foram
ajustados também segundo uma perspectiva Bayesiana. De modo a avaliar
qual combinação destes parâmetros é suficiente para explicar a variação
em índices AVG, ajustamos três modelos diferentes: dois utilizando
intercepto ($a_s$) e inclinação ($b_s$) separadamente, e um modelo que
considera o efeito conjunto dos dois parâmetros ($a_s + b_s$). Estes
três modelos foram comparados utilizando o Critério de Informação de
Deviância (DIC; Gelman \emph{et al.}, 2004), de modo que o modelo com o
menor DIC frente ao conjunto de modelos considerados é aquele que melhor
se ajusta aos dados. De maneira equivalente ao Critério de Informação de
Akaike (Akaike, 1974), modelos concorrentes cuja diferença em DIC é
menor do que dois são considerados equivalentes.

\subsection{Software}\label{software}

All analysis were performed under R 3.1.2 (R Core Team, 2014). Our code
for the estimation of local shape variables can be found at
\url{http://github.com/wgar84}. We performed MCMC sampling for all
models using the MCMCglmm package in R (Hadfield, 2010). In order to
obtain symmetrical landmarks configurations, we use code provided by
Annat Haber, available at
\url{http://life.bio.sunysb.edu/morph/soft-R.html}.

\section{Results}\label{results}

\begin{figure}[htbp]
\centering
\includegraphics{Figures/cac_shape-1.pdf}
\caption{Shape variation associated with the Common Allometric
Component. \label{fig:cac_shape}}
\end{figure}

% latex table generated in R 3.1.2 by xtable 1.7-4 package
% Thu May  7 14:31:10 2015
\begin{table}[ht]
  \centering
  \caption{Correlations between functional vectors representing cranial regions and the Common Allometric Component. \label{tab:cac_dir}}
  \begin{tabular}{rr}
    \toprule
    Region & Correlation \\ 
    \midrule
    Face & 0.4077 \\          
    Neurocranium & -0.6033 \\ 
    Oral & 0.3462 \\ 
    Nasal & 0.0966 \\ 
    Zygomatic & 0.2895 \\ 
    Orbit & -0.1156 \\ 
    Base & -0.1507 \\ 
    Vault & -0.6513 \\ 
    \bottomrule
  \end{tabular}
\end{table}

Em relação à associação entre o CAC e o log Tamanho do Centróide
(\autoref{fig:cac_vs_logCS}), examinando a distribuição \emph{a
posteriori} de inclinações (\autoref{fig:phylo_slopeW}) e interceptos
(\autoref{fig:phylo_interW}) estáticos dentre os vértices da filogenia,
observamos que as inclinações são mais estáveis que interceptos. Apenas
em duas linhagens (\emph{Homo} e \emph{Gorilla}) observamos desvios
negativos em $b_s$; desvios de $a_s$ negativos e positivos são
observados em diferentes linhagens.

\begin{figure}[htbp]
\centering
\includegraphics{Figures/cac_vs_logCS-1.pdf}
\caption{The relationship between log Centroid Size and the Common
Allometric Component for all anthropoid primates.
\label{fig:cac_vs_logCS}}
\end{figure}

\begin{figure}[htbp]
\centering
\includegraphics{Figures/phylo_slopeW-1.pdf}
\caption{Static allometric slopes. Distribution of $b_s$ for the
regression between the Common Allometric Component and log Centroid
Size. Red circles indicate positive slope deviations, while blue circles
indicate negative ones. The outer circles indicate whether a given slope
deviation is distinct from zero, using 95\% credible intervals.
\label{fig:phylo_slopeW}}
\end{figure}

\begin{figure}[htbp]
\centering
\includegraphics{Figures/phylo_interW-1.pdf}
\caption{Distribution of $a_s$ for the regression between the Common
Allometric Component and log Centroid Size. Red circles indicate
positive slope deviations, while blue circles indicate negative ones.
The outer circles indicate whether a given intercept deviation is
distinct from zero, using 95\% credible intervals.
\label{fig:phylo_interW}}
\end{figure}

Em relação à associação entre parâmetros alométricos e índices AVG, os
modelos de regressão indicam, em geral, que o modelo que considera
apenas a inclinação alométrica estática $b_s$ como variável preditora
possui um Critério de Informação de Deviância menor do que os demais
modelos (\autoref{tab:dic_allo_im}), ainda que, na maioria dos casos, o
modelo que contém o efeito conjunto de $a_s$ e $b_s$ é equivalente. No
caso da subregião da Abóbada, o modelo que contém $a_s$ e $b_s$ possuem
valores de DIC menores. No entanto, ao observar a distribuição \emph{a
posteriori} de coeficientes de regressão em cada um desses modelos de
efeito conjunto, o efeito de $a_s$ não se mostra significativo,
considerando um intervalo de credibilidade de 95\% (resultado não
apresentado). Assim, consideramos os modelos que contém apenas $b_s$
como variável preditora como aqueles que melhor representam a associação
entre parâmetros alométricos e índices AVG.

É importante destacar que, no caso das subregiões da Órbita e Base,
todos os modelos de regressão testados indicam que não há associação
entre parâmetros alométricos e índices AVG, de maneira que os valores de
DIC associados à estas duas regiões não foram incluídos.

\begin{table}[ht]
  \centering
  \begin{threeparttable}
    \caption{Model selection for phylogenetic linear models of AVG indexes over allometric parameters. \label{tab:dic_allo_im}}
    \begin{tabular}{lccc}
      \toprule
      & \multicolumn{3}{c}{\small{Deviance Information Criterion$^{a}$}} \\
      & $b_s$ & $a_s$ & $a_s + b_s$ \\ 
      \midrule
      Face & {\bf -568.58} & -526.41 & -568.04 \\ 
      Neurocranium & {\bf -477.64} & -447.44 & -463.70 \\ 
      Oral & -527.06 & -512.59 & {\bf -527.27} \\ 
      Nasal & {\bf -370.57} & -364.70 & -366.84 \\ 
      Zygomatic & {\bf -433.86} & -424.77 & -431.43 \\ 
      Orbit$^{b}$ & -261.30 & -262.52 & -261.61 \\ 
      Base$^{b}$ & -471.45 & -473.83 & -469.62 \\ 
      Vault & -412.95 & -400.78 & {\bf -414.07} \\ 
      \bottomrule
    \end{tabular}
    \begin{tablenotes}
      \footnotesize{
        \item [a] Bold values indicate the model with the lowest Deviance Information Criterion.
        \item [b] None of these models indicate significant effects of both parameters over integration.
        }
    \end{tablenotes}
  \end{threeparttable}
\end{table}

Dessa forma, é possível observar que inclinações alométricas possuem
efeitos contrários sobre os índices AVG da Face e do Neurocrânio
(\autoref{fig:MI_vs_slopeW}); aquelas subregiões que são contidas em
cada uma destas regiões segue a mesma tendência da partição mais
inclusiva que as contém.

\begin{figure}[htbp]
\centering
\includegraphics{Figures/MI_vs_slopeW-1.pdf}
\caption{AVG Index \emph{vs.} static allometric slopes. For each trait
set, the shaded region around regression lines indicate 95\% credible
intervals around estimated regression parameters.
\label{fig:MI_vs_Wslope}}
\end{figure}

\section{Discussion}\label{discussion}

Using estimates of allometry derived from the relationship between local
shape variables and centroid size, the relationship between magnitude
and pattern of morphological integration becomes clear. In the mammalian
skull, this relationship is dominated by the interaction between the
masticatory muscles and facial bones (Zelditch \& Carmichael, 1989;
Hallgrímsson \& Lieberman, 2008; Hallgrímsson \emph{et al.}, 2009),
which are more affected by skull growth during the preweaning period and
afterwards. Therefore, allometric relationships play a major role in the
distinction between early- and late-developmental factors in skull
correlation structure.

Geometric approaches to study morphological integration often eschew the
role of size variation in covariance structure, focusing on shape alone;
furthermore, landmark-based shape covariance/correlation patterns are
poorly estimated due to the effects of Procrustes superimposition.
Therefore, conclusions regarding pattern and magnitude of integration
considering landmark-based shape alone usually disagree with traditional
morphometric approaches. Our results demonstrate that, with a careful
consideration of the strengths and weaknesses of both approaches,
geometric and traditional morphometrics can be used together to test the
effects of the multiple factors that influence morphological
integration.

Akaike, H. 1974. A new look at the statistical model identification.
\emph{IEEE Transactions on Automatic Control} \textbf{19}: 716--723.

Bookstein, F.L. 1991. \emph{Morphometric tools for landmark data:
geometry and biology}. Cambridge University Press, Cambridge.

Bookstein, F.L. 1989. ``Size and Shape'': A Comment on Semantics.
\emph{Systematic Zoology} \textbf{38}: 173--180.

Cheverud, J.M. 1996a. Developmental integration and the evolution of
pleiotropy. \emph{American Zoology} \textbf{36}: 44--50.

Cheverud, J.M. 1995. Morphological integration in the saddle-back
tamarin (Saguinus fuscicollis) cranium. \emph{American Naturalist}
\textbf{145}: 63--89.

Cheverud, J.M. 1996b. Quantitative genetic analysis of cranial
morphology in the cotton-top (Saguinus oedipus) and saddle-back (S.
fuscicollis) tamarins. \emph{Journal of Evolutionary Biology}
\textbf{9}: 5--42.

Gelman, A., Carlin, J.B., Stern, H.S. \& Rubin, D.B. 2004.
\emph{Bayesian data analysis}, 2ª ed. CRC Press, New York.

Hadfield, J.D. 2010. MCMC Methods for Multi-Response Generalized Linear
Mixed Models: The \{MCMCglmm\} \{R\} Package. \emph{Journa of
Statistical Software} \textbf{33}: 1--22.

Hallgrímsson, B. \& Lieberman, D.E. 2008. Mouse models and the
evolutionary developmental biology of the skull. \emph{Integrative and
Comparative Biology} \textbf{48}: 373--384.

Hallgrímsson, B., Jamniczky, H., Young, N.M., Rolian, C., Parsons, T.E.
\& Boughner, J.C.\emph{et al.} 2009. Deciphering the Palimpsest:
Studying the Relationship Between Morphological Integration and
Phenotypic Covariation. \emph{Evolutionary Biology} \textbf{36}:
355--376.

Huxley, J.S. 1932. \emph{Problems of relative growth}.

Jolicoeur, P. 1963. The Multivariate Generalization of the Allometry
Equation. \emph{Biometrics}.

Klingenberg, C.P., Barluenga, M. \& Meyer, A. 2002. Shape analysis of
symmetric structures: Quantifying variation among individuals and
asymmetry. \emph{Evolution} \textbf{56}: 1909--1920.

Linde, K. van der \& Houle, D. 2009. Inferring the Nature of Allometry
from Geometric Data. \emph{Evolutionary Biology} \textbf{36}: 311--322.

Marroig, G. \& Cheverud, J.M. 2001. A comparison of phenotypic variation
and covariation patterns and the role of phylogeny, ecology, and
ontogeny during cranial evolution of new world monkeys. \emph{Evolution}
\textbf{55}: 2576--2600.

Marroig, G. \& Cheverud, J.M. 2004. Cranial evolution in sakis
(Pithecia, Platyrrhini) I: Interspecific differentiation and allometric
patterns. \emph{American Journal of Physical Anthropology} \textbf{125}:
266--278.

Márquez, E.J., Cabeen, R., Woods, R.P. \& Houle, D. 2012. The
Measurement of Local Variation in Shape. \emph{Evolutionary Biology}
\textbf{39}: 419--439.

Mitteroecker, P., Gunz, P., Bernhard, M., Bookstein, F.L. \& Schaefer,
K. 2004. Comparison of cranial ontogenetic trajectories among great apes
and humans. \emph{Journal of Human Evolution} \textbf{46}: 679--697.

Oliveira, F.B., Porto, A. \& Marroig, G. 2009. Covariance structure in
the skull of Catarrhini: a case of pattern stasis and magnitude
evolution. \emph{Journal of Human Evolution} \textbf{56}: 417--430.

Olson, E. \& Miller, R. 1958. \emph{Morphological integration}.
University of Chicago Press, Chicago.

Pélabon, C., Bolstad, G.H., Egset, C.K., Cheverud, J.M., Pavlicev, M. \&
Rosenqvist, G. 2013. On the Relationship between Ontogenetic and Static
Allometry. \emph{The American Naturalist} \textbf{181}: 195--212.

Pélabon, C., Firmat, C., Bolstad, G.H., Voje, K.L., Houle, D. \&
Cassara, J.\emph{et al.} 2014. Evolution of morphological allometry.
\emph{Annals of the New York Academy of Sciences} n/a--n/a.

Porto, A., Oliveira, F.B., Shirai, L.T., Conto, V. de \& Marroig, G.
2009. The evolution of modularity in the mammalian skull I:
morphological integration patterns and magnitudes. \emph{Evolutionary
Biology} \textbf{36}: 118--135.

Porto, A., Shirai, L.T., Oliveira, F.B. de \& Marroig, G. 2013. Size
Variation, Growth Strategies, and the Evolution of Modularity in the
Mammalian Skull. \emph{Evolution} \textbf{67}: 3305--3322.

R Core Team. 2014. \emph{R: A Language and Environment for Statistical
Computing}. R Foundation for Statistical Computing, Vienna, Austria.

Shirai, L.T. \& Marroig, G. 2010. Skull modularity in neotropical
marsupials and monkeys: size variation and evolutionary constraint and
flexibility. \emph{Journal of experimental zoology. Part B, Molecular
and developmental evolution} \textbf{314B}: 663--683.

Springer, M.S., Meredith, R.W., Gatesy, J., Emerling, C.A., Park, J. \&
Rabosky, D.L.\emph{et al.} 2012. Macroevolutionary Dynamics and
Historical Biogeography of Primate Diversification Inferred from a
Species Supermatrix. \emph{PLoS ONE} \textbf{7}: e49521.

Swiderski, D.L. 2003. Separating Size from Allometry: Analysis of Lower
Jaw Morphology in the Fox Squirrel, Sciurus niger. \emph{Journal of
Mammalogy} \textbf{84}: 861--876.

Voje, K.L., Hansen, T.F., Egset, C.K., Bolstad, G.H. \& Pélabon, C.
2013. Allometric Constraints and the Evolution of Allometry.
\emph{Evolution} n/a--n/a.

Walker, J.A. 2000. Ability of Geometric Morphometric Methods to Estimate
a Known Covariance Matrix. \emph{Systematic Biology} \textbf{49}:
686--696.

Zelditch, M.L. \& Carmichael, A.C. 1989. Ontogenetic variation in
patterns of developmental and functional integration in skulls of
Sigmodon fulviventer. \emph{Evolution} \textbf{43}: 814--824.

Zelditch, M.L., Swiderski, D.L., Sheets, H.D. \& Fink, W.L. 2004.
\emph{Geometric Morphometrics for Biologists: A Primer}, 1st ed.
Elsevier.

\end{document}
