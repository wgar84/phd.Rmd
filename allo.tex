\documentclass[12pt,]{article}
\usepackage{lmodern}
\usepackage{amssymb,amsmath}
\usepackage{ifxetex,ifluatex}
\usepackage{fixltx2e} % provides \textsubscript
\ifnum 0\ifxetex 1\fi\ifluatex 1\fi=0 % if pdftex
  \usepackage[T1]{fontenc}
  \usepackage[utf8]{inputenc}
\else % if luatex or xelatex
  \ifxetex
    \usepackage{mathspec}
    \usepackage{xltxtra,xunicode}
  \else
    \usepackage{fontspec}
  \fi
  \defaultfontfeatures{Mapping=tex-text,Scale=MatchLowercase}
  \newcommand{\euro}{€}
\fi
% use upquote if available, for straight quotes in verbatim environments
\IfFileExists{upquote.sty}{\usepackage{upquote}}{}
% use microtype if available
\IfFileExists{microtype.sty}{%
\usepackage{microtype}
\UseMicrotypeSet[protrusion]{basicmath} % disable protrusion for tt fonts
}{}
\usepackage[margin=1in]{geometry}
\usepackage{graphicx}
\makeatletter
\def\maxwidth{\ifdim\Gin@nat@width>\linewidth\linewidth\else\Gin@nat@width\fi}
\def\maxheight{\ifdim\Gin@nat@height>\textheight\textheight\else\Gin@nat@height\fi}
\makeatother
% Scale images if necessary, so that they will not overflow the page
% margins by default, and it is still possible to overwrite the defaults
% using explicit options in \includegraphics[width, height, ...]{}
\setkeys{Gin}{width=\maxwidth,height=\maxheight,keepaspectratio}
\ifxetex
  \usepackage[setpagesize=false, % page size defined by xetex
              unicode=false, % unicode breaks when used with xetex
              xetex]{hyperref}
\else
  \usepackage[unicode=true]{hyperref}
\fi
\hypersetup{breaklinks=true,
            bookmarks=true,
            pdfauthor={Guilherme Garcia1,2, Thiago Zahn1, Felipe Bandoni de Oliveira1 \& Gabriel Marroig1},
            pdftitle={The Evolution of Skull Allometry in Anthropoid Primates},
            colorlinks=true,
            citecolor=blue,
            urlcolor=blue,
            linkcolor=magenta,
            pdfborder={0 0 0}}
\urlstyle{same}  % don't use monospace font for urls
\setlength{\parindent}{0pt}
\setlength{\parskip}{6pt plus 2pt minus 1pt}
\setlength{\emergencystretch}{3em}  % prevent overfull lines
\setcounter{secnumdepth}{0}

%%% Use protect on footnotes to avoid problems with footnotes in titles
\let\rmarkdownfootnote\footnote%
\def\footnote{\protect\rmarkdownfootnote}

%%% Change title format to be more compact
\usepackage{titling}

% Create subtitle command for use in maketitle
\newcommand{\subtitle}[1]{
  \posttitle{
    \begin{center}\large#1\end{center}
    }
}

\setlength{\droptitle}{-2em}
  \title{The Evolution of Skull Allometry in Anthropoid Primates}
  \pretitle{\vspace{\droptitle}\centering\huge}
  \posttitle{\par}
  \author{Guilherme Garcia\textsuperscript{1,2}, Thiago Zahn\textsuperscript{1},
Felipe Bandoni de Oliveira\textsuperscript{1} \& Gabriel
Marroig\textsuperscript{1}}
  \preauthor{\centering\large\emph}
  \postauthor{\par}
  \predate{\centering\large\emph}
  \postdate{\par}
  \date{11 September 2015}

%--- USEPACKAGES ---
\usepackage[utf8]{inputenc}
\usepackage{natbib}
\usepackage[brazilian,english]{babel}
\usepackage{hyperref}
\usepackage{subfigure, epsfig}
\usepackage{ae}
\usepackage{aecompl}
\usepackage{booktabs}
\usepackage[T1]{fontenc}
\usepackage{graphicx,wrapfig} % para incluir figuras
\usepackage{amsfonts, amssymb,amsthm, amsmath, amscd} % pacote AMS
\usepackage{color, bbm, multicol}
\usepackage{verbatim, listings, booktabs}
\usepackage{fancyhdr} % FANCYHEADER
\usepackage{setspace}
\usepackage{times} % bookman,palatino,courier,times FONTES
\usepackage{lineno}
\usepackage{url}
\usepackage{rotating}
\usepackage{longtable}
% \usepackage{rotfloat}
% \usepackage{appendix}
\usepackage{mathptmx}
% \usepackage{cmbright}
% \usepackage[adobe-utopia]{mathdesign}
\usepackage[flushleft]{threeparttable}
\usepackage{multirow}
% \usepackage{float}
\usepackage {tocvsec2} % controlar profundidade de table of contents
\setcounter {secnumdepth}{0}
\usepackage {caption}
\usepackage {tabularx}
\usepackage {floatrow}
\floatsetup[table]{capposition=top}

\usepackage{mathpazo} % fonte palatino

\newcommand{\barra}{\backslash}
\newcommand{\To}{\longrightarrow}
\newcommand{\abs}[1]{\left\vert#1\right\vert}
\newcommand{\set}[1]{\left\{#1\right\}}
\newcommand{\seq}[1]{\left<#1\right>}
\newcommand{\norma}[1]{\left\Vert#1\right\Vert}
\newcommand{\hr}{\par\noindent\hrulefill\par}

\usepackage {xr}
\externaldocument{sup_allo}

\selectlanguage{english}

\hypersetup{colorlinks=false}


\begin{document}

\maketitle


\linenumbers
\modulolinenumbers[2]

\onehalfspacing

\textsuperscript{1}Laboratório de Evolução de Mamíferos, Departamento de
Genética e Biologia Evolutiva, Instituto de Biociências, Universidade de
São Paulo, CP 11.461, CEP 05422-970, São Paulo, Brasil

\textsuperscript{2}\href{mailto:wgar@usp.br}{wgar@usp.br}

running title: Allometry in Anthropoids

key words:

\section{Introduction}\label{introduction}

Anthropoid primates display an approximately thousand-fold variation of
body sizes, from pygmy marmosets ($\sim 110g$) to lowland gorillas
($\sim 120kg$). This astonishing size diversity implies substantial
morphological, physiological and ecological differences among these
organisms, since changes in body size fundamentally affect how living
organisms (and their composing parts) interact with their environments
(Gould, 1966; West \emph{et al.}, 1997). These differential scaling
relationships between organismal traits and body size are referred to as
allometry (Huxley, 1932; Gould, 1966; Cheverud, 1982; Lande, 1985) and
may be represented as power laws, which become linear on a log-log scale
(Huxley, 1932; but see Nijhout \& German, 2012). Allometry may thus be
represented by two parameters: (allometric) slopes and intercepts.

Allometry may act as a constraint, diverting evolutionary trajectories
in natural populations (Gould \& Lewontin, 1979; Lande, 1979; Schluter,
1996). For example, selection for body size driven by shifts in feeding
habits has dominated diversification in New World monkeys, and
differences in skull morphology among species in this group appear to be
merely an indirect consequence of the allometric relationships between
skull traits and body size (Marroig \& Cheverud, 2005, 2010; Marroig
\emph{et al.}, 2012). With respect to Old World Monkeys, skull
morphological evolution may also be adaptive and related to allometric
associations, at least within Cercopithecines (Cardini \emph{et al.},
2007; Cardini \& Elton, 2008), while Ackermann \& Cheverud (2004) have
shown that selection for the reduction of facial elements has operated
during the early diversification of the genus \emph{Homo}.

Regarding morphological traits, allometric slopes may be constrained
throughout development due to stabilizing selection (Lande, 1980;
Cheverud, 1984, 1996a; Hansen, 2011; Zelditch \& Swiderski, 2011), while
intercepts may be more labile on an evolutionary scale (Gould, 1974;
Egset \emph{et al.}, 2012). However, these parameters are dependant upon
the mean individual growth rate and the covariance between these growth
rates and body size achieved at adulthood; thus, selection on body size
may also change allometric slopes (Pélabon \emph{et al.}, 2013; Voje
\emph{et al.}, 2013). Experimental evidence on the evolution of
allometric relationships has accumulated, indicating that growth
parameters harbor genetic variation and thus may be available to
selection, an effect that will then cascade to populational allometric
relationships (e.g. Frankino \emph{et al.}, 2005; Tobler \& Nijhout,
2010; Egset \emph{et al.}, 2012), although methodological issues may
hamper some of this evidence (Houle \emph{et al.}, 2011; Voje \emph{et
al.}, 2013). Therefore, whether and to which extent allometric slopes
can evolve remain as open questions.

In the present work, we investigate the relationship between skull size
and shape in a phylogenetic framework using anthropoid primates as model
organisms. We focus on testing whether and to which proportion
intercepts and slopes have changed during the diversification of this
group. Moreover, integration patterns in skull form are quite stable in
both New World (Marroig \& Cheverud, 2001) and Old World Monkeys, with
some exceptions (Oliveira \emph{et al.}, 2009). Several authors have
argued that allometric constraints contribute to integration patterns
(e.g. Porto \emph{et al.}, 2009; Armbruster \emph{et al.}, 2014), but
this association has never been formalized. Thus, we test the hypothesis
that variation in allometric parameters, if present, will contribute to
variation in the strength of association between skull traits.
Considering both the spatial and temporal dynamics of mammalian skull
development (Zelditch \& Carmichael, 1989; Hallgrímsson \emph{et al.},
2009; Cardini \& Polly, 2013), we expect that such changes will affect
Facial and Neurocranial traits in opposing ways, due to the distinction
between early- and late-developmental factors in skull development.

Using a Bayesian phylogenetic random regression model of allometric
shape against size, we estimate allometric parameters along the
phylogeny of Anthropoidea. We show that allometric slopes have remained
quite stable throughout anthropoid diversification, while allometric
intercepts have suffered changes in several groups, and that the little
variation associated with allometric slopes produces opposite changes in
Facial and Neurocranial integration, providing a clear link between
allometry and integration. Despite the general stability in allometric
slopes, we find remarkable changes occurring in Homo and Gorilla,
demonstrating that constraints imposed by allometry are in no way
absolute and, in some situations, may be overcome.

\section{Results}\label{results}

We used the Common Allometric Component (CAC; Mitteroecker \emph{et
al.}, 2004), which represents a pooled estimate of skull shape allometry
(represented as local shape variables; Márquez \emph{et al.}, 2012) as a
proxy for the ancestral allometric relationships for anthropoid
primates. This axis is associated with positive loadings for Facial
traits and negative loadings for Neurocranial traits, thus representing
a contrast between these regional sets (\autoref{fig:cac_logCS}a).

\begin{figure}[htbp]
\centering
\includegraphics{Figures/cac_logCS-1.pdf}
\caption{\textbf{Allometric shape variation.} (a) Shape variation
associated with the Common Allometric Component (CAC). (b) Random
regression of projections on the CAC over log Centroid Size. A
regression line is adjusted for each species, and dots represent mean
values. \label{fig:cac_logCS}}
\end{figure}

To investigate whether species deviate from such ancestral
relationships, we carried out a Bayesian mixed model, considering that
slopes and intercepts for the regression between individuals projected
over the CAC and log Centroid Size (logCS) within each species
(\autoref{fig:cac_logCS}b) are codependent upon their phylogenetic
relationships. The posterior distribution of deviations from mean for
both parameters estimated in this manner (\autoref{fig:phylo_W})
indicates that intercepts deviate from the mean in at least seven
lineages: increases occur in \emph{Pithecia} and \emph{Callicebus}
within Pitheciids, in the clade composed of Callithrichinae and Aotinae,
and in \emph{Alouatta} within Atelids, while lower intercepts are found
in the clade composed of \emph{Homo} and \emph{Pan} within Hominidae, in
\emph{Hylobates} within Hylobatids, and also in \emph{Ateles} within
Atelids (\autoref{fig:phylo_W}a). Allometric slopes for terminals, on
the other hand, deviate from the mean in only three species: \emph{Homo
sapiens} and both representatives of \emph{Gorilla}, all with slopes
shallower than the mean (\autoref{fig:phylo_W}b).

\begin{figure}[htbp]
\centering
\includegraphics{Figures/phylo_new-1.pdf}
\caption{\textbf{Deviations from ancestral allometry.} (a) Intercepts;
(b) Slopes. Red circles indicate positive deviations, while blue circles
indicate negative ones. The outer circles indicate whether a given
deviation is distinct from zero, using 95\% credible intervals.
Particular clades are indicated as abbreviations (Platyrrhini:
\textbf{Pit}hecidae, \textbf{Ate}lidae, \textbf{Ceb}inae,
\textbf{Cal}lithrichinae, and \textbf{Aot}inae; Catarrhini:
\textbf{Hom}inidae, \textbf{Hyl}obatidae, \textbf{Pap}ionini,
\textbf{Cer}copithecini, and \textbf{Col}obinae). \label{fig:phylo_W}}
\end{figure}

The association between allometric parameters and modularity was
measured using regression models of Modularity Indexes for two regions
(Face and Neurocranium) and three subregions within each region (Oral,
Nasal and Zygomatic in the Face; Orbit, Vault and Basicranium in the
Neurocranium) on the estimated allometric parameters. Model selection
using the Deviance Information Criterion (DIC; Gelman \emph{et al.},
2004) show that models considering only the intercept always have the
worst fit. In most cases, models including only the slope are better
than or equal ($\Delta DIC < 2$) to models with both parameters
(intercept and slope), suggesting a negligible effect of the intercept.
For the Neurocranium region and the Vault subregion, the model including
intercept and slope shows a better fit than the slope-only model; in
both cases, however, the effects of the intercept are not significant
when one takes into account the posterior distribution of the regression
coefficients (results not shown). We thus consider that the models
including only allometric slopes provide the best representation of the
association between allometric parameters and modularity indexes
(\autoref{tab:dic_allo_im}). It should be noted that none of the models
tested for the Orbit and Basicranium subregions show any association
between allometric parameters and Modularity Indexes, therefore these
subregions were not included in the model selection. Allometric slopes
show opposite effects on the Modularity Index for Face and Neurocranium
(\autoref{fig:MI_vs_slopeW_main}), and, with the exception of the Orbit
and Basicranium, subregions follow the same pattern as the more
inclusive partition in which they are contained
(\autoref{fig:MI_vs_slopeW_si}).

\begin{figure}[htbp]
\centering
\includegraphics{Figures/MI_vs_slopeW_main-1.pdf}
\caption{\textbf{Allometry and modularity.} Modularity Hypothesis Index
(MHI) regressed over allometric slope deviations ($b_s$). For each
global trait set (Face and Neurocranium), the shaded region around
regression lines indicate 95\% credible intervals around estimated
regression parameters. \label{fig:MI_vs_slopeW_main}}
\end{figure}

\section{Discussion}\label{discussion}

The hypothesis that allometry may act as a constraint, deflecting
changes in morphological, physiological and other organismal aspects
along a size-trait gradient (e.g. Gould, 1974) has historically been
confronted with the opposing view that allometry may be in fact much
more dynamic and evolvable (e.g. Kodric-Brown \emph{et al.}, 2006;
Bonduriansky, 2007; Frankino \emph{et al.}, 2009). As recently stated by
some authors (Voje \emph{et al.}, 2013; Pélabon \emph{et al.}, 2014),
this seeming paradox may in large part owe its existence to different
meanings for allometry: one related to any monotonic relationship
between variables, while the other specifically deals with power-law
relationships between variables as defined by Huxley (1932). This
imprecision of definition greatly hampers evidence in favor of a dynamic
interpretation of allometry; considering the few studies estimating
evolvability of narrow-sense allometry, only one clearly demonstrates
changes in allometric slopes (Tobler \& Nijhout, 2010).

Considering the results we obtained from the phylogenetic random
regression model, substantial changes have occurred on intercepts,
especially in New World Monkeys (Figure 1). As we expected, the role of
natural selection on the evolution of body size, as explored elsewhere
(Marroig \& Cheverud, 2005, 2010; Marroig \emph{et al.}, 2012) can be
related to these changes. For example, in the lineage composed of
Callithrichines and \emph{Aotus}, selection for reduced body size is
pervasive, and it has been associated with shifts in pre-natal growth
rates, considering a maintenance in relative gestation lengths with
respect to their sizes (Marroig \& Cheverud, 2009); such change in the
timing of developmental events may be responsible for the changes in
intercepts we observe in tamarins, marmosets and owl monkeys. For
Atelids, while selection for increased body size has played a role in
the differentiation between this lineage and its sister group Cebidae,
the shifts in allometric intercepts occurred within this clade in two
different lineages, and in opposing directions.

It is also noteworthy that, in some lineages of New World Monkeys,
selection for changes in body size did not affect intercepts or slopes
at all; with respect to \emph{Cebus}, for instance, selection for
increased body size (Marroig \emph{et al.}, 2012) produced only a
correlated response on skull shape, along the same allometric
relationships observed in its sister genus, \emph{Saimiri} (Marroig,
2007). The converse is also true: while saki monkeys (\emph{Pithecia})
show increased intercepts, this group has diversified mostly as a
consequence of drift, at least with respect to skull morphology (Marroig
\& Cheverud, 2004).

\section{Methods}\label{methods}

\subsection{Sample}\label{sample}

Our database consists of 5108 individuals, distributed across 109
species. These species are spread throughout all major Anthropoid clades
above the genus level, also comprising all Platyrrhini genera and a
substantial portion of Catarrhini genera. We associate this database
with a ultrametric phylogenetic hypothesis for Anthropoidea (Figure
\ref{fig:phylo_model}), derived from Springer et al. (2012).

Individuals in our sample are represented by 36 registered landmarks,
using either a Polhemus 3Draw or a Microscribe 3DS for Platyrrhini and
Catarrhini, respectively. Twenty-two unique landmarks represent each
individual (Figure \ref{fig:landmarks}), since fourteen of the 36
registered landmarks are bilaterally symmetrical. For more details on
landmark registration, see Marroig \& Cheverud (2001) and Oliveira et
al. (2009). Databases from both previous studies were merged into a
single database, retaining only those individuals in which all
landmarks, from both sides, were present.

This database of landmark registration data was used to estimate both
interlandmark distances and shape variables. For interlandmark
distances, those measurements that involve bilaterally symmetrical
landmarks were averaged after computing distances; for shape variables,
a symmetrical landmark configuration was obtained by taking the mean
shape between each individual configuration and its reflection along the
sagittal plane (Klingenberg \emph{et al.}, 2002).

\subsection{Allometric Slopes and
Intercepts}\label{allometric-slopes-and-intercepts}

We estimated allometric parameters using local shape variables (Márquez
\emph{et al.}, 2012), which are measurements of infinitesimal log volume
transformations, calculated as the natural logarithm determinants of
derivatives of the TPS function between each individual in our sample
and a reference shape (in our case, the mean shape for the entire
sample, estimated from a Generalized Procrustes algorithm). Such
derivatives were evaluated at 38 locations, the midpoints between pairs
of landmarks connected in \autoref{fig:landmarks}. After estimating
local shape variables, sources of variation of little interest in the
present context, such as sexual dimorphism and variation between
subspecies or populations were controlled within each OTU, according to
Figure \ref{fig:phylo_model}, using generalized linear models.

We used a phylogenetic random regression model under a Bayesian
framework in order to estimate static (within populations; Cheverud,
1982; Pélabon \emph{et al.}, 2013) allometric intercepts ($a_s$) and
slopes ($b_s$) for all OTUs simultaneously while considering their
phylogenetic structure. This model assumes that both $a_s$ and $b_s$
evolve under Brownian motion, as both parameters are defined as random
variables with a correlation structure among OTUs derived from branch
lengths obtained from the phylogenetic hypothesis we use. This allows
the model to estimate $a_s$ and $b_s$ for each terminal OTU and also for
ancestral nodes, enabling us to track changes in both parameters along
the phylogeny.

We projected all individuals in our sample along the Common Allometric
Component (CAC; Mitteroecker \emph{et al.}, 2004), which is the pooled
within-species slopes between local shape variables and log Centroid
Size (logCS). The CAC summarizes all allometric shape variation, and we
regress this single variable against logCS in our random regression
model. This reduction in dimensionality is necessary because a full
multivariate random regression model using all 38 shape variables would
be computationally untractable, considering the actual state of MCMC
samplers available. Therefore, we limit ourselves to test whether the
strength of association between size and allometric shape, represented
by projections over this CAC (which we consider the best representation
of the ancestral allometric shape variation) has changed during the
diversification of anthropoid primates; in order to test whether the
direction of allometric shape variation has changed during anthropoid
diversification, a full multivariate random regression model would be
necessary.

We used uniform prior distributions for all $a_s$ and $b_s$. In order to
sample the posterior distribution for our model, we used a MCMC sampler
with $100000$ iterations, comprising a burnin period of $50000$
iterations and a thinning interval of $50$ iterations after burnin to
avoid autocorrelations in the posterior sample, thus generating $1000$
posterior samples for all parameters we estimate. We performed a handful
of runs with different starting values and pseudo-random number
generator seeds to ensure convergence; with these values for iteration
steps, we achieved convergence in all MCMC runs.

Using these posterior samples for static allometric parameters, we test
whether a given intercept or slope in any node of the tree (terminal or
ancestral) deviates from the values estimated for the root of the tree
by computing 95\% credible intervals for the difference between the
parameter estimated at both points. If this interval excludes the null
value, we consider that as evidence that the parameter ($a_s$ or $b_s$)
has changed in that particular node.

\subsection{Morphological Integration}\label{morphological-integration}

Using our landmark configuration database, we calculated 38
interlandmark distances (\autoref{fig:landmarks}), based on previous
works on mammalian covariance patterns (Cheverud, 1995, 1996b; Marroig
\& Cheverud, 2001; Oliveira \emph{et al.}, 2009). Within each OTU, we
estimate and remove those fixed effects of little interest within the
present context using generalized linear models (Figure
\ref{fig:phylo_model}). Using residuals from each model, we estimated
covariance and correlation $P$-matrices for all OTUs.

In order to represent modularity patterns embedded in these
$P$-matrices, we use Modularity Indexes (Porto \emph{et al.}, 2013),
which is a scale-free measurement of variational modularity; we
estimated MIs for each trait subset depicted in \autoref{tab:dist} for
all 109 $P$-matrices. We estimated Modularity Indexes using the equation
\[
MI = \frac {\bar{\rho}_{+} - \bar{\rho}_{-}} {ICV}
\] where $\bar{\rho}_{+}$ represents the average correlation within a
given subset, $\bar{\rho}_{-}$ represents the average correlation of the
complementary set, and $ICV$ is the coefficient of variation of
eigenvalues of the associated covariance matrix, which is a measurement
of the overall integration between all traits (Shirai \& Marroig, 2010).

We estimated the relationship between Modularity Indexes and static
allometric parameters using phylogenetic mixed linear models; we
adjusted such models separately for each trait subset, using MI values
as response variables and allometric parameters as predictors. We also
use a Bayesian framework to estimate such models. In order to evaluate
which parameters are sufficient to explain variation in Modularity
Indexes, we adjusted three different models: one for static intercepts
alone ($a_s$), one for static slopes ($b_s$), and a third model that
considers the joint effect of both parameters, without interactions
($a_s$ + $b_s$). We compared these three models using Deviance
Information Criterion (DIC; Gelman \emph{et al.}, 2004), which increases
as a function of the average posterior likelihood for a particular model
and decreases as a function of the number of parameters considered; in a
equivalent manner to Akaike's (1974) Information Criterion, the model
with the smallest DIC is the best fit to the data considered, and models
whose difference in DICs is lower than two are considered equivalent.

\subsection{Software}\label{software}

All analysis were performed under R 3.2.2 (R Core Team, 2015). Our code
for the estimation of local shape variables can be found at
\url{http://github.com/wgar84}. We performed MCMC sampling for all
models using the MCMCglmm package in R (Hadfield, 2010). In order to
obtain symmetrical landmarks configurations, we use code provided by
Annat Haber, available at
\url{http://life.bio.sunysb.edu/morph/soft-R.html}.

\section*{References}\label{references}
\addcontentsline{toc}{section}{References}

Ackermann, R.R. \& Cheverud, J.M. 2004. Detecting genetic drift versus
selection in human evolution. \emph{Proceedings of the National Academy
of Sciences of the United States of America} \textbf{101}: 17946--17951.

Akaike, H. 1974. A new look at the statistical model identification.
\emph{IEEE Transactions on Automatic Control} \textbf{19}: 716--723.

Armbruster, W.S., Pélabon, C., Bolstad, G.H. \& Hansen, T.F. 2014.
Integrated phenotypes: understanding trait covariation in plants and
animals. \emph{Philosophical Transactions of the Royal Society of London
B: Biological Sciences} \textbf{369}: 20130245.

Bonduriansky, R. 2007. Sexual Selection and Allometry: A Critical
Reappraisal of the Evidence and Ideas. \emph{Evolution} \textbf{61}:
838--849.

Cardini, A. \& Elton, S. 2008. Variation in guenon skulls (I): species
divergence, ecological and genetic differences. \emph{Journal of Human
Evolution} \textbf{54}: 615--637.

Cardini, A. \& Polly, P.D. 2013. Larger mammals have longer faces
because of size-related constraints on skull form. \emph{Nature
Communications} \textbf{4}.

Cardini, A., Jansson, A.-U. \& Elton, S. 2007. A geometric morphometric
approach to the study of ecogeographical and clinal variation in vervet
monkeys. \emph{Journal of Biogeography} \textbf{34}: 1663--1678.

Cheverud, J.M. 1996a. Developmental integration and the evolution of
pleiotropy. \emph{American Zoology} \textbf{36}: 44--50.

Cheverud, J.M. 1995. Morphological integration in the saddle-back
tamarin (Saguinus fuscicollis) cranium. \emph{American Naturalist}
\textbf{145}: 63--89.

Cheverud, J.M. 1996b. Quantitative genetic analysis of cranial
morphology in the cotton-top (Saguinus oedipus) and saddle-back (S.
fuscicollis) tamarins. \emph{Journal of Evolutionary Biology}
\textbf{9}: 5--42.

Cheverud, J.M. 1984. Quantitative genetics and developmental constraints
on evolution by selection. \emph{Journal of Theoretical Biology}
\textbf{110}: 155--172.

Cheverud, J.M. 1982. Relationships among ontogenetic, static, and
evolutionary allometry. \emph{American Journal of Physical Anthropology}
\textbf{59}: 139--149.

Egset, C.K., Hansen, T.F., Le Rouzic, A., Bolstad, G.H., Rosenqvist, G.
\& Pélabon, C. 2012. Artificial selection on allometry: change in
elevation but not slope. \emph{Journal of Evolutionary Biology}
\textbf{25}: 938--948.

Frankino, W.A., Emlen, D.J. \& Shingleton, A.W. 2009. Experimental
approaches to studying the evolution of animal form. In:
\emph{Experimental evolution: concepts, methods, and applications of
selection experiments} (T. Garland Jr. \& M. R. Rose, eds), pp.
419--478.

Frankino, W.A., Zwaan, B.J., Stern, D.L. \& Brakefield, P.M. 2005.
Natural Selection and Developmental Constraints in the Evolution of
Allometries. \emph{Science} \textbf{307}: 718--720.

Gelman, A., Carlin, J.B., Stern, H.S. \& Rubin, D.B. 2004.
\emph{Bayesian data analysis}, 2ª ed. CRC Press, New York.

Gould, S.J. 1966. Allometry and Size in Ontogeny and Phylogeny.
\emph{Biological Reviews} \textbf{41}: 587--638.

Gould, S.J. 1974. Allometry in primates, with emphasis on scaling and
the evolution of the brain. \emph{Contributions to primatology}
\textbf{5}: 244--292.

Gould, S.J. \& Lewontin, R.C. 1979. The spandrels of San Marco and the
Panglossian paradigm: a critique of the adaptationist programme.
\emph{Proceedings of the Royal Society of London B: Biological Sciences}
\textbf{205}: 581--598.

Hadfield, J.D. 2010. MCMC Methods for Multi-Response Generalized Linear
Mixed Models: The \{MCMCglmm\} \{R\} Package. \emph{Journa of
Statistical Software} \textbf{33}: 1--22.

Hallgrímsson, B., Jamniczky, H., Young, N.M., Rolian, C., Parsons, T.E.
\& Boughner, J.C.\emph{et al.} 2009. Deciphering the Palimpsest:
Studying the Relationship Between Morphological Integration and
Phenotypic Covariation. \emph{Evolutionary Biology} \textbf{36}:
355--376.

Hansen, T.F. 2011. Epigenetics: adaptation or contingency. In:
\emph{Epigenetics: linking genotype and phenotype in development and
evolution} (B. Hallgrímsson \& B. K. Hall, eds), pp. 357--376.

Houle, D., Pélabon, C., Wagner, G.P. \& Hansen, T.F. 2011. Measurement
and Meaning In Biology. \emph{The Quartely Review of Biology}
\textbf{86}: 3--34.

Huxley, J.S. 1932. \emph{Problems of relative growth}.

Klingenberg, C.P., Barluenga, M. \& Meyer, A. 2002. Shape analysis of
symmetric structures: Quantifying variation among individuals and
asymmetry. \emph{Evolution} \textbf{56}: 1909--1920.

Kodric-Brown, A., Sibly, R.M. \& Brown, J.H. 2006. The allometry of
ornaments and weapons. \emph{Proceedings of the National Academy of
Sciences} \textbf{103}: 8733--8738.

Lande, R. 1985. Genetic and Evolutionary Aspects of Allometry. In:
\emph{Size and Scaling in Primate Biology} (W. L. Jungers, ed), pp.
21--32. Springer US.

Lande, R. 1979. Quantitative genetic analysis of multivariate evolution
applied to brain: body size allometry. \emph{Evolution} \textbf{33}:
402--416.

Lande, R. 1980. The Genetic Covariance Between Characters Maintained by
Pleiotropic Mutations. \emph{Genetics} \textbf{94}: 203--215.

Marroig, G. 2007. When size makes a difference: allometry, life-history
and morphological evolution of capuchins (Cebus) and squirrels (Saimiri)
monkeys (Cebinae, Platyrrhini). \emph{BMC Evolutionary Biology}
\textbf{7}: 20.

Marroig, G. \& Cheverud, J.M. 2001. A comparison of phenotypic variation
and covariation patterns and the role of phylogeny, ecology, and
ontogeny during cranial evolution of new world monkeys. \emph{Evolution}
\textbf{55}: 2576--2600.

Marroig, G. \& Cheverud, J.M. 2004. Cranial evolution in sakis
(Pithecia, Platyrrhini) I: Interspecific differentiation and allometric
patterns. \emph{American Journal of Physical Anthropology} \textbf{125}:
266--278.

Marroig, G. \& Cheverud, J.M. 2009. Size and Shape in Callimico and
Marmoset Skulls: Allometry and Heterochrony in the Morphological
Evolution of Small Anthropoids. In: \emph{The Smallest Anthropoids,
Developments in Primatology: Progress and Prospects} (S. M. Ford, L. M.
Porter, \& L. C. Davis, eds), pp. 331--353. Springer US, Boston.

Marroig, G. \& Cheverud, J.M. 2005. Size as a line of least evolutionary
resistance: Diet and adaptive morphological radiation in new world
monkeys. \emph{Evolution} \textbf{59}: 1128--1142.

Marroig, G. \& Cheverud, J.M. 2010. Size as a line of least resistance
II: direct selection on size or correlated response due to constraints?
\emph{Evolution} \textbf{64}: 1470--1488.

Marroig, G., Melo, D.A.R. \& Garcia, G. 2012. Modularity, Noise and
Natural Selection. \emph{Evolution} \textbf{66}: 1506--1524.

Márquez, E.J., Cabeen, R., Woods, R.P. \& Houle, D. 2012. The
Measurement of Local Variation in Shape. \emph{Evolutionary Biology}
\textbf{39}: 419--439.

Mitteroecker, P., Gunz, P., Bernhard, M., Bookstein, F.L. \& Schaefer,
K. 2004. Comparison of cranial ontogenetic trajectories among great apes
and humans. \emph{Journal of Human Evolution} \textbf{46}: 679--697.

Nijhout, H.F. \& German, R.Z. 2012. Developmental Causes of Allometry:
New Models and Implications for Phenotypic Plasticity and Evolution.
\emph{Integrative and Comparative Biology} \textbf{52}: 43--52.

Oliveira, F.B., Porto, A. \& Marroig, G. 2009. Covariance structure in
the skull of Catarrhini: a case of pattern stasis and magnitude
evolution. \emph{Journal of Human Evolution} \textbf{56}: 417--430.

Pélabon, C., Bolstad, G.H., Egset, C.K., Cheverud, J.M., Pavlicev, M. \&
Rosenqvist, G. 2013. On the Relationship between Ontogenetic and Static
Allometry. \emph{The American Naturalist} \textbf{181}: 195--212.

Pélabon, C., Firmat, C., Bolstad, G.H., Voje, K.L., Houle, D. \&
Cassara, J.\emph{et al.} 2014. Evolution of morphological allometry.
\emph{Annals of the New York Academy of Sciences} n/a--n/a.

Porto, A., Oliveira, F.B., Shirai, L.T., Conto, V. de \& Marroig, G.
2009. The evolution of modularity in the mammalian skull I:
morphological integration patterns and magnitudes. \emph{Evolutionary
Biology} \textbf{36}: 118--135.

Porto, A., Shirai, L.T., Oliveira, F.B. de \& Marroig, G. 2013. Size
Variation, Growth Strategies, and the Evolution of Modularity in the
Mammalian Skull. \emph{Evolution} \textbf{67}: 3305--3322.

R Core Team. 2015. \emph{R: A Language and Environment for Statistical
Computing}. R Foundation for Statistical Computing, Vienna, Austria.

Schluter, D. 1996. Adaptive radiation along genetic lines of least
resistance. \emph{Evolution} \textbf{50}: 1766--1774.

Shirai, L.T. \& Marroig, G. 2010. Skull modularity in neotropical
marsupials and monkeys: size variation and evolutionary constraint and
flexibility. \emph{Journal of experimental zoology. Part B, Molecular
and developmental evolution} \textbf{314B}: 663--683.

Springer, M.S., Meredith, R.W., Gatesy, J., Emerling, C.A., Park, J. \&
Rabosky, D.L.\emph{et al.} 2012. Macroevolutionary Dynamics and
Historical Biogeography of Primate Diversification Inferred from a
Species Supermatrix. \emph{PLoS ONE} \textbf{7}: e49521.

Tobler, A. \& Nijhout, H.F. 2010. Developmental constraints on the
evolution of wing-body allometry in Manduca sexta. \emph{Evolution \&
Development} \textbf{12}: 592--600.

Voje, K.L., Hansen, T.F., Egset, C.K., Bolstad, G.H. \& Pélabon, C.
2013. Allometric Constraints and the Evolution of Allometry.
\emph{Evolution} \textbf{68}: 866--885.

West, G.B., Brown, J.H. \& Enquist, B.J. 1997. A General Model for the
Origin of Allometric Scaling Laws in Biology. \emph{Science}
\textbf{276}: 122--126.

Zelditch, M.L. \& Carmichael, A.C. 1989. Ontogenetic variation in
patterns of developmental and functional integration in skulls of
Sigmodon fulviventer. \emph{Evolution} \textbf{43}: 814--824.

Zelditch, M.L. \& Swiderski, D.L. 2011. Epigenetic interactions: the
developmental route to functional integration. In: \emph{Epigenetics:
linking genotype and phenotype in development and evolution}, pp.
290--316.

\end{document}
