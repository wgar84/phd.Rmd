\usepackage[english,brazilian]{babel} % suporte para línguas
\usepackage[utf8] {inputenc} % codificação

\usepackage{subfig, epsfig}
\captionsetup[subfigure]{style=default, 
  margin=0pt, parskip=0pt, hangindent=0pt, indention=0pt, 
  singlelinecheck=true, labelformat=parens, labelsep=space}

\usepackage{ae}
\usepackage{aecompl}
\usepackage{booktabs}
\usepackage[T1] {fontenc}

\usepackage{footnote}

% Notas criadas nas tabelas ficam no fim das tabelas
\makesavenoteenv{tabular}

\usepackage{fancyhdr}

\fancypagestyle{plain}
{
  \fancyhf{}%
  \renewcommand{\headrulewidth}{0pt}%
  \fancyfoot[C]{\thepage}
}

\usepackage{graphicx,wrapfig} % para incluir figuras
\usepackage[all]{xy} % para incluir diagramas
\usepackage{amsfonts, amssymb, amsthm, amsmath, amscd, textcomp} % pacote AMS
\usepackage{color, float, bbm, multicol, rotating}
\usepackage{verbatim, listings, booktabs}

\usepackage{caption} % Customizar as legendas de figuras e tabelas
\usepackage{array} % Elementos extras para formatação de tabelas

\usepackage{lineno} % números nas linhas

\usepackage {tocvsec2} % controlar profundidade de table of contents
\setcounter {secnumdepth}{0}
\setcounter {tocdepth}{1}

%\widowpenalty10000
%\clubpenalty10000

\usepackage{mathpazo} % fonte palatino
\usepackage{hyperref}

% Adicionar bibliografia, índice e conteúdo na Tabela de conteúdo
% Não inclui lista de tabelas e figuras no índice
\usepackage[nottoc,notlof,notlot, notindex]{tocbibind}

\usepackage{icomma} % Posicionar inteligentemente a vírgula como separador decimal
\usepackage[tight]{units} % Formatar as unidades com as distâncias corretas

\usepackage{setspace}

\usepackage{lastpage} % Conta o número de páginas

\usepackage{pdflscape} % ambiente landscape

\usepackage[round]{natbib}
\usepackage{chapterbib}

\usepackage[flushleft]{threeparttable}
\usepackage {tabularx}

\usepackage[draft]{pdfpages}

\usepackage[draft]{fixme}
%\usepackage{pdfcomment}

\fxsetup{layout={footnote}}

% --- definições gerais ---
\newcommand{\barra}{\backslash}
\newcommand{\To}{\longrightarrow}
\newcommand{\abs}[1]{\left\vert#1\right\vert}
\newcommand{\set}[1]{\left\{#1\right\}}
\newcommand{\seq}[1]{\left<#1\right>}
\newcommand{\norma}[1]{\left\Vert#1\right\Vert}
\newcommand{\hr}{\par\noindent\hrulefill\par}
% --- ---

\newcommand{\titulo}{Perspectivas sobre o reconhecimento de padrões de modularidade e suas implicações para a evolução de morfologias complexas}
\newcommand{\nomedoaluno}{Guilherme Garcia}
\newcommand{\advisor}{Gabriel Marroig} \newcommand{\ano}{2015}
\hypersetup{colorlinks=true, linkcolor=black, citecolor=black,
  filecolor=black, pagecolor=black, urlcolor=black,
  pdfauthor={\nomedoaluno}, pdftitle={\titulo}, pdfsubject={Genética e
    Biologia Evolutiva}, pdfkeywords={Genética Quantitiva, Morfologia
    Craniana, Roedores}, pdfproducer={Latex}, pdfcreator={pdflatex}}
 
\geometry{bindingoffset=15pt}
%\geometry{paperwidth=290mm, paperheight=297mm, margin=1in}
%\setlength{\marginparwidth=80mm}
