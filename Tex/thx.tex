\onehalfspacing

Em primeiro lugar, agradeço ao meu orientador, Gabriel Marroig, por ter me dado a oportunidade de trabalhar neste tema e por sempre contribuir para me devolver ao chão com as nossas discussões e o rigor com que ele encara a ciência. 
Espero que nós tenhamos muitos e muitos anos de colaboração pela frente.

Agradeço também aos meus pares, os demais alunos do LEM, por prover um ambiente fantástico para se fazer ciência, onde as discussões sempre acontecem, inclusive agora, aqui ao lado, enquanto escrevo estes agradecimentos. 
E claro, muitas vezes nossas discussões vão para além da ciência, mas nunca em direção ao senso comum, e estas outras conversas também contribuem ao seu modo para a confecção de um doutor em ciências.

À Aninha, por ser diversas vezes interrompida por um fluxo caótico de ideias inacabadas e ter a paciência de escutá-las, desvantagem de sentar ao meu lado.
À Paulinha, por nunca ter recusado ir tomar um café comigo, mesmo que ela precisasse muito ir embora, e por toda a empolgação que ela traz em relação a tudo que a gente faz.
À Dani, pelo suprimento infinito de polenguinho light, por ler vários pedaços desta tese ainda em formação, e por todos os sorrisos e abraços que ajudaram nas horas mais escuras.
À Tafinha (ou Bárbara), por manter o bom humor mesmo nestas horas escuras e por sempre estar disposta a discutir essencialmente qualquer coisa, independente de quanto tempo isso vai levar.
À Monique, pelas divagações aleatórias a respeito de ciência, que contribuem muito pra esta tese, sem dúvida, e também pelas conversas a respeito das nossas crianças lindas.
Ao Ogro (ou Diogo), pela parceria agora de longa data nas partes mais cabeludas das análises e por ser uma fonte inesgotável de boa música, boa comida, e bom gosto em geral.
Ao Lugar (ou Fábio), pelo humor \emph{nonsense} afiadíssimo que sempre alegra o dia e pelas nossas discussões edificantes a respeito do mundo mágico da morfometria geométrica.
Ao Wally (ou Thiago), pela companhia nas tarefas mais divertidas, por exemplo levar computadores pro conserto, e claro, pelas infinitas discussões sobre ciência, sociedade e muito, muito mais.
E à Papete (ou Anna), por ser uma boa companhia pra todas as horas, da mesinha até a volta pra casa, por escutar o que eu tenho a dizer, independente do que seja, e pelas parcerias que a gente ainda deve levar em frente a respeito desses primatas aí.
Também agradeço aqueles que já deixaram o LEM, em especial Alex, Fino, Edgar e Harley, por todas as contribuições pessoais e profissionais nos anos que passaram.
Nesse embalo, agradeço também o Aríete (ou Gustavo) pelas nossas discussões regadas à café ou cerveja, que com certeza contribuem muito, e também por ler pedaços desta tese em formação.

Agradeço também aos colegas do grupo-irmão do Diogo Meyer, que ofereçem diversas perspectivas diferentes às nossas; em especial à Bárbara e ao Limão (ou Luiz Carlos), pelas conversas muito boas que tivemos recentemente. Também agradeço ao Rui Murrieta, pela oportunidade de ser monitor na disciplina de Filosofia, o que certamente contribuiu para o desenvolvimento desta tese, e ao Sérgio Matioli, por conta dos comentários durante a minha qualificação que foram o ponto de partida para o Capítulo 4. 

Agradeço à FAPESP pela concessão de uma bolsa de doutorado.

Devo também agradecer a minha família. 
Sem todos vocês, seria impossível chegar até aqui. 
Em especial, aos meus pais, Eduardo e Teresa, pelo apoio incondicional durante todos esses anos.
Ao meu irmão Gustavo, que nunca tem receios em me dizer o que eu tenho que ouvir.
E à vó Carlota, pela dieta de banana, cabeça de peixe e taboada na infância.

Finalmente, agradeço a minha companheira, a Giuliana, por estar aqui ao meu lado, mesmo nos piores momentos, e por me oferecer aquele abraço perfeito.
E ao Ramiro, nosso menininho, por conta daquele sorriso lindo que ele deu ontem quando eu cheguei em casa, e por dar tanto sentido a tudo.

\singlespacing