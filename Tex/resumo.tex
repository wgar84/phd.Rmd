A modularidade é uma propriedade característica que sistemas biológicos exibem em relação à distribuição de interações entre seus elementos constituintes; neste contexto, um módulo é um subconjunto de elementos que interagem entre si mais do que com outros subconjuntos.
Em relação aos sistemas morfológicos, tais propriedades referem-se geralmente à estrutura do componente linear do mapa genótipo/fenótipo; no entanto, as interações genéticas, ontogenéticas e funcionais que produzem fenótipos são descritas de forma adequada através de dinâmicas não-lineares, e uma apreciação completa da complexidade destas interações é necessária para a compreensão das propriedades variacionais do fenótipo.
Ademais, dados avanços metodológicos na área da morfometria, é possível escolher diferentes maneiras de representar a variação morfológica, e as diferenças entre as representações podem impactar inferências feitas sobre estas propriedades variacionais.
A presente tese tem como objetivo explorar a relação entre representações morfométricas e a caracterização das propriedades variacionais, focada na análise comparativa de tais propriedades em uma escala macroevolutiva; Primatas Antropóides são utilizados como modelo, dada a disponibilidade de uma grande base de dados de mensurações cranianas destes organismo.
Esta relação foi avaliada sob três perspectivas diferentes.
Em primeiro lugar, estima-se taxas de erro associadas aos testes de hipótese que descrevem padrões de modularidade, relacionadas com três representações morfométricas distintas; tal avaliação é também associada à exploração de um subconjunto da base de dados utilizada aqui, levando-se em consideração a dinâmica de interações ontogenéticas que produzem o crânio dos Antropóides.
Os resultados deste capítulo implicam que uma dessas representações, resíduos de Procrustes, não são capazes de detectar padrões de modularidade neste contexto, considerando suas propriedades matemáticas específicas.
Outras duas representações, distâncias entre marcos anatômicos e variáveis locais de forma, produzem resultados semelhantes, que estão diretamente associados à dinâmica de desenvolvimento, e as diferenças que elas apresentam são consistentes com suas diferenças principais; taxas de erro para os testes sobre as duas representações também são aceitáveis.
O próximo capítulo trata da comparação entre estas duas representações no que diz respeito a estas diferentes propriedades, focado em estimar relações alométricas associadas às variáveis locais de forma e a relação entre estas estimativas e os padrões de modularidade estimados para distâncias entre marcos anatômicos.
Os resultados encontrados enfatizam que os padrões de modularidade observados em distâncias entre marcos são consequência da alometria; linhagens como \emph{Homo} e \emph{Gorilla}, que apresentam padrões distintos de modularidade para as distâncias entre marcos estão associados a mudanças substanciais nas relações alométricas dos caracteres cranianos.
O último capítulo explora a estrutura filogenética de mudanças nas propriedades variacionais fenotípicas na diversificação de Anthropoidea, considerando apenas variáveis de forma locais, uma vez que este capítulo também visa reforçar os resultados anteriores obtidos a partir de distâncias entre marcos, considerando-se um tipo diferente de representação morfométrica.
Este capítulo muda o foco de testes a respeito de padrões de modularidade definidos \emph{a priori} em direção a estimar a incerteza relacionada à estrutura de matrizes de covariância,  decomposta sobre a filogenia de Anthropoidea.
Os resultados obtidos demonstram que as mudanças na estrutura de covariância nesta linhagem são localizadas nas mesmas regiões do crânio ao longo de toda a história evolutiva do grupo, enquanto outras regiões mantêm associações estáveis.
Assim, quando se considera as diferentes propriedades de representações morfométricas cuidadosamente, inferências feitas a partir de tais representações sobre propriedades variacionais são de fato compatíveis.