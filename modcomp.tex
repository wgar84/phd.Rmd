\documentclass[12pt,]{article}
\usepackage{lmodern}
\usepackage{amssymb,amsmath}
\usepackage{ifxetex,ifluatex}
\usepackage{fixltx2e} % provides \textsubscript
\ifnum 0\ifxetex 1\fi\ifluatex 1\fi=0 % if pdftex
  \usepackage[T1]{fontenc}
  \usepackage[utf8]{inputenc}
\else % if luatex or xelatex
  \ifxetex
    \usepackage{mathspec}
    \usepackage{xltxtra,xunicode}
  \else
    \usepackage{fontspec}
  \fi
  \defaultfontfeatures{Mapping=tex-text,Scale=MatchLowercase}
  \newcommand{\euro}{€}
\fi
% use upquote if available, for straight quotes in verbatim environments
\IfFileExists{upquote.sty}{\usepackage{upquote}}{}
% use microtype if available
\IfFileExists{microtype.sty}{%
\usepackage{microtype}
\UseMicrotypeSet[protrusion]{basicmath} % disable protrusion for tt fonts
}{}
\usepackage[margin=1in]{geometry}
\usepackage{graphicx}
\makeatletter
\def\maxwidth{\ifdim\Gin@nat@width>\linewidth\linewidth\else\Gin@nat@width\fi}
\def\maxheight{\ifdim\Gin@nat@height>\textheight\textheight\else\Gin@nat@height\fi}
\makeatother
% Scale images if necessary, so that they will not overflow the page
% margins by default, and it is still possible to overwrite the defaults
% using explicit options in \includegraphics[width, height, ...]{}
\setkeys{Gin}{width=\maxwidth,height=\maxheight,keepaspectratio}
\ifxetex
  \usepackage[setpagesize=false, % page size defined by xetex
              unicode=false, % unicode breaks when used with xetex
              xetex]{hyperref}
\else
  \usepackage[unicode=true]{hyperref}
\fi
\hypersetup{breaklinks=true,
            bookmarks=true,
            pdfauthor={Guilherme Garcia1,2, Felipe Bandoni de Oliveira1 \& Gabriel Marroig1},
            pdftitle={Modularity and Morphometrics: Error Rates in Hypothesis Testing},
            colorlinks=true,
            citecolor=blue,
            urlcolor=blue,
            linkcolor=magenta,
            pdfborder={0 0 0}}
\urlstyle{same}  % don't use monospace font for urls
\setlength{\parindent}{0pt}
\setlength{\parskip}{6pt plus 2pt minus 1pt}
\setlength{\emergencystretch}{3em}  % prevent overfull lines
\setcounter{secnumdepth}{0}

%%% Use protect on footnotes to avoid problems with footnotes in titles
\let\rmarkdownfootnote\footnote%
\def\footnote{\protect\rmarkdownfootnote}

%%% Change title format to be more compact
\usepackage{titling}

% Create subtitle command for use in maketitle
\newcommand{\subtitle}[1]{
  \posttitle{
    \begin{center}\large#1\end{center}
    }
}

\setlength{\droptitle}{-2em}
  \title{Modularity and Morphometrics: Error Rates in Hypothesis Testing}
  \pretitle{\vspace{\droptitle}\centering\huge}
  \posttitle{\par}
  \author{Guilherme Garcia\textsuperscript{1,2}, Felipe Bandoni de
Oliveira\textsuperscript{1} \& Gabriel Marroig\textsuperscript{1}}
  \preauthor{\centering\large\emph}
  \postauthor{\par}
  \predate{\centering\large\emph}
  \postdate{\par}
  \date{16 November 2015}

%--- USEPACKAGES ---
\usepackage[utf8]{inputenc}
\usepackage{natbib}
\usepackage[brazilian,english]{babel}
\usepackage{hyperref}
\usepackage{subfigure, epsfig}
\usepackage{ae}
\usepackage{aecompl}
\usepackage{booktabs}
\usepackage[T1]{fontenc}
\usepackage{graphicx,wrapfig} % para incluir figuras
\usepackage{amsfonts, amssymb,amsthm, amsmath, amscd} % pacote AMS
\usepackage{color, bbm, multicol}
\usepackage{verbatim, listings, booktabs}
\usepackage{fancyhdr} % FANCYHEADER
\usepackage{setspace}
\usepackage{times} % bookman,palatino,courier,times FONTES
\usepackage{lineno}
\usepackage{url}
\usepackage{rotating}
\usepackage{longtable}
% \usepackage{rotfloat}
% \usepackage{appendix}
\usepackage{mathptmx}
% \usepackage{cmbright}
% \usepackage[adobe-utopia]{mathdesign}
\usepackage[flushleft]{threeparttable}
\usepackage{multirow}
% \usepackage{float}
\usepackage {tocvsec2} % controlar profundidade de table of contents
\setcounter {secnumdepth}{0}
\usepackage {caption}
\usepackage {tabularx}
\usepackage {floatrow}
\floatsetup[table]{capposition=top}

\usepackage{mathpazo} % fonte palatino

\newcommand{\barra}{\backslash}
\newcommand{\To}{\longrightarrow}
\newcommand{\abs}[1]{\left\vert#1\right\vert}
\newcommand{\set}[1]{\left\{#1\right\}}
\newcommand{\seq}[1]{\left<#1\right>}
\newcommand{\norma}[1]{\left\Vert#1\right\Vert}
\newcommand{\hr}{\par\noindent\hrulefill\par}

\usepackage {xr}
\externaldocument{sup_modcomp}

\selectlanguage{english}

\hypersetup{colorlinks=false}


\begin{document}

\maketitle


\linenumbers
\modulolinenumbers[2]

\onehalfspacing

\textsuperscript{1}Laboratório de Evolução de Mamíferos, Departamento de
Genética e Biologia Evolutiva, Instituto de Biociências, Universidade de
São Paulo, CP 11.461, CEP 05422-970, São Paulo, Brasil

\textsuperscript{2}\href{mailto:wgar@usp.br}{wgar@usp.br}

running title: Modularity, Error and Morphometrics

keywords: $P$-matrix; statistical power; primates; development;
genotype/phenotype map

\section{Abstract}\label{abstract}

The study of modularity in morphological systems has increased in the
past twenty years, parallel to the popularization of geometric
morphometrics, which has led to the emergence of different criteria for
detecting modularity on landmark data. However, compared to usual
covariance matrix estimators, Procrustes estimators have properties that
hinder their use. Here, we compare different representations of form,
focusing on detecting modularity patterns defined \emph{a priori}; we
also compare two metrics: one derived from traditional morphometrics
(MHI) and another that emerged in the context of landmark data (RV).
Using Anthropoid skulls, we compare these metrics over three
representations of form: interlandmark distances, Procrustes residuals,
and local shape variables. Over Procrustes residuals, both tests fail to
detect modularity patterns, while in remaining representations they show
the distinction between early and late development in skull ontogeny. To
estimate type I and II error rates, we built covariance matrices of
known structure; these tests indicate that, considering both effect and
sample sizes, tests using MHI are more robust than those using RV.
However, both metrics have low statistical power when used on Procrustes
residuals; thus, we conclude that the influence of development and
function is poorly represented on Procrustes estimators for covariance
matrices.

\section{Introduction}\label{introduction}

Modularity is a characteristic property that biological systems exhibit
regarding the distribution of interactions between their composing
elements; that is, in a given system, subsets of elements, denominated
modules, interact more among themselves than with other such subsets
(Newman, 2006; Mitteroecker \& Bookstein, 2007; Wagner \emph{et al.},
2007). This property has been well documented at different levels of
biological organization, from the dynamics of metabolic networks (e.g.
Ravasz \emph{et al.}, 2002; Andrade \emph{et al.}, 2011) to the
structure of interactions among individuals in populations (e.g. Fortuna
\emph{et al.}, 2008) and among species in ecological communities (e.g.
Genini \emph{et al.}, 2010).

Regarding morphological systems, the concept of modularity is associated
with the framework of morphological integration (Olson \& Miller, 1958;
Cheverud, 1996), which refers to the organization of covariances or
correlations between morphological elements and the hypotheses
concerning their relationships. In this context, modularity is related
to the uneven distribution of genetic effects over phenotypic variation
articulated through development (genotype/phenotype map; Wagner, 1996);
in a classical quantitative genetics view, these genetic effects are the
result of pleiotropy and linkage disequilibrium (Falconer \& Mackay,
1996; Lynch \& Walsh, 1998). A genotype/phenotype map composed of
clusters of genes that affect clusters of traits (with little overlap)
exhibits a modular organization; this structure is thought to emerge as
the result of selection for distinct demands (Wagner \& Altenberg, 1996;
Espinosa-Soto \& Wagner, 2010; Rueffler \emph{et al.}, 2012; Melo \&
Marroig, 2015). For instance, the decoupling between fore- and hindlimb
function in certain mammalian lineages such as bats (Young \&
Hallgrímsson, 2005) and apes (Young \emph{et al.}, 2010) is associated
with the modularization of both structures, as shown by reduced
phenotypic correlations between fore- and hindlimbs and increased
correlations between elements within these limbs.

The recognition of variational modules (Wagner \& Altenberg, 1996;
Wagner \emph{et al.}, 2007) using covariance or correlation patterns in
adult populations involves an uderstanding of the underlying
developmental and functional dynamics among morphological traits (Polly,
2008; Zelditch \& Swiderski, 2011). Skull development in mammals occurs
through a series of steps, such as neurocranial growth induced by brain
development, and growth mediated by muscle-bone interactions, with
spatiotemporal overlapping between such steps (Hallgrímsson \&
Lieberman, 2008; Herring, 2011; Cardini \& Polly, 2013). Both timing and
scope of each step is regulated by different profiles of genetic
expression exhibited by cells originated from different embryonic
precursors and their response to signaling factors expressed at the
regional level; the response to these signaling factors further changes
cell expression profiles, thus generating a feedback loop of
diferentiation (Turing, 1952; Marcucio \emph{et al.}, 2005; Meinhardt,
2008; Hallgrímsson \emph{et al.}, 2009; Franz-Odendaal, 2011; Minelli,
2011). Each step in this temporal hierarchy may be regarded as modular,
since they affect a coherent subset of tissues more so than others
(Hallgrímsson \& Lieberman, 2008), although each step affect adjoining
regions through interactions among developing tissues (Cheverud \emph{et
al.}, 1992; Lieberman, 2011; Esteve-Altava \& Rasskin-Gutman, 2014);
thus, the overlapping of such processes throughout development may
complicate their association with correlation patterns (Hallgrímsson
\emph{et al.}, 2009).

Furthermore, variation in growth rates, which emerges into size
variation (Pélabon \emph{et al.}, 2013; Porto \emph{et al.}, 2013), has
a particular importance in the context of mammalian morphological
systems. Here size variation refers to variation in both scale
(isometric variation) and scale relationships (allometric variation).
This source of variation affects the overall level of correlations
between morphological traits (Wagner, 1984; Young \& Hallgrímsson,
2005), and the magnitude of integration has important consequences for
both the evolution of mean phenotypes (Schluter, 1996; Marroig \&
Cheverud, 2005, 2010; Cardini \& Polly, 2013) and the evolution of
morphological integration itself (Oliveira \emph{et al.}, 2009; Porto
\emph{et al.}, 2009, 2013; Shirai \& Marroig, 2010).

In this context, adaptive landscapes may be the central component
governing both the stability and divergence in integration patterns, as
both stabilizing and directional selection have been shown to produce
changes in integration patterns (Jones, 2007; Arnold \emph{et al.},
2008; Jones \emph{et al.}, 2012; Melo \& Marroig, 2015). The empirical
evidence available demonstrates that both stability (e.g. Marroig \&
Cheverud, 2001; Oliveira \emph{et al.}, 2009; Porto \emph{et al.}, 2009;
Willmore \emph{et al.}, 2009) and divergence (e.g. Monteiro \& Nogueira,
2010; Grabowski \emph{et al.}, 2011; Sanger \emph{et al.}, 2012; Haber,
2015) of integration patterns are possible outcomes of the evolutionary
process. Therefore, the question of whether any of these two scenarios
is the rule or exception at macroevolutionary scales remains open,
although some theoretical and methodological differences between these
works with respect to the representation of morphological features need
to be considered.

\subsubsection{Morphometrics}\label{morphometrics}

Traditionally, morphological features have been measured using distances
among elements defined in general terms, such as ``cranial length'' or
``cranial width''. Pearson \& Davin (1924) introduced the notion that
measurements should be restricted to single osteological elements,
preferably as distances between homologous features that could be
identified in a wide taxonomic coverage. Cheverud (1982) accomodates
this notion into the framework of morphological integration, thus
considering individual measurements over single bones as local
representations of regional phenomena, that is, the functional,
developmental and genetic interactions that produce covariances among
these elements. The influence of such interactions over covariance or
correlation matrices can be accessed by defining a subset of
measurements which are associated with particular processes and
estimating a metric that summarizes such partitioning; the null
hypothesis that this partition is undistinguishable from
randomly-defined partitions can then be tested using Monte Carlo methods
(Mantel, 1967; Cheverud \emph{et al.}, 1989).

In the past three decades, geometric morphometrics (Bookstein, 1982,
1991; Kendall, 1984; Rohlf \& Slice, 1990; Goodall, 1991) has been
consolidated as a quantitative framework for the representation of
biological shape as geometric configurations of homologous features
(landmarks; Bookstein, 1991). Two principles are central here: first,
the conceptual and statistical separation of size and shape as
components of biological form; second, the use of superimposition-based
methods (GPA: Generalized Procrustes Analysis; Rohlf \& Slice, 1990;
Goodall, 1991) for the estimation of shape statistical parameters, such
as mean shape and shape covariance structure. Procrustes estimators were
proposed as a solution to the problem that landmark configurations are
arbitrarily rotated and translated; such nuisance parameters are
impossible to estimate without any assumptions (Goodall, 1991; Lele \&
McCulloch, 2002), a situation known as the identifiability problem
(Neyman \& Scott, 1948).

Although the use of Procrustes estimators is currently widespread in
geometric morphometrics toolboxes (e.g. Klingenberg, 2011; Adams \&
Otárola-Castillo, 2013), critics have been made regarding the estimation
of both mean shape configurations (e.g. Lele, 1993; Kent \& Mardia,
1997; Huckemann, 2012) and shape covariance matrices (Walker, 2000;
Adams \emph{et al.}, 2004; Linde \& Houle, 2009; Márquez \emph{et al.},
2012). For configurations in two-dimensional space, Procrustes
estimators for mean shape perform well under isotropic landmark
covariance structure (Kent \& Mardia, 1997), a situation of null
covariances among landmarks and coordinates; however, when this
assumption does not hold, Procrustes estimates for mean shape may lose
either precision and accuracy, especially when shape variation is high
(Huckemann, 2011). The example provided by Linde \& Houle (2009)
demonstrates that when such assumption is broken shape covariance
patterns are also poorly estimated; if the unknown landmark covariance
matrix is structured due to regional differences in
covariance-generating processes, such variation might be displaced and
effectively spread out through the entire landmark configuration.

A number of alternatives for estimating shape covariance matrices have
already been suggested; some of these alternatives (e.g. Monteiro
\emph{et al.}, 2005; Theobald \& Wuttke, 2006; Linde \& Houle, 2009;
Zelditch \emph{et al.}, 2009) propose modifications to the Procrustes
analysis in order to deal with heterogeneity in landmark covariance
structure. Márquez et al. (2012) propose another definition of shape
descriptors using interpolation-based techniques (Cheverud \&
Richtsmeier, 1986; Bookstein, 1989) as a starting point. These
descriptors refer to infinitesimal expansions or retractions in
reference to a unknown mean shape (Woods, 2003) estimated at definite
locations amidst sampled landmarks. The authors argue that such
descriptors are proper local measurements of shape variation, as they
can be directly linked to biological processes that generate covariation
among morphological elements.

Despite these caveats, the use of Procrustes estimators for
investigating aspects of morphological integration has increased in the
past ten years (e.g. Klingenberg \emph{et al.}, 2004; Drake \&
Klingenberg, 2010; Goswami \& Polly, 2010; Martínez-Abadías \emph{et
al.}, 2011; Sanger \emph{et al.}, 2012). The results found by these
authors are sometimes in stark constrast with similar works using
interlandmark distances (e.g. Cheverud \emph{et al.}, 1997; Oliveira
\emph{et al.}, 2009; Porto \emph{et al.}, 2009). For instance,
Martínez-Abadías et al. (2011) found a pattern of strong integration
among partitions in human skull covariance patterns, while Oliveira et
al. (2009) and Porto et al. (2009) demonstrated that humans are among
the most modular examples of mammalian skull covariance patterns.
Likewise, while Cheverud et al. (1997) had shown that 70\% of the
pleiotropic effects are confined to either anterior or posterior
mandibular components, Klingenberg et al. (2004) has found no evidence
for a modular distribution of pleiotropic effects among the partitions
of the mouse mandible using the same strain of intercrossed mice at the
same generation. Furthermore, aside from the issues regarding Procrustes
estimators, these works also propose different methods to quantify the
effects of interactions over covariance patterns. For example,
Martínez-Abadías et al. (2011) and Sanger et al. (2012) use the RV
coefficient, a multivariate correlation coefficient defined by Escoufier
(1973) which has been used to quantify modular relationships over
landmark covariance patterns since Klingenberg (2009) has proposed its
use in this context.

\subsubsection{Objectives}\label{objectives}

In the present work, we compare the methods described by Cheverud et al.
(1989) and Klingenberg (2009) to test \emph{a priori} defined modularity
patterns using anthropoid primates as a model organism. In order to
compare the performance of these methods with respect to different
representations of form, individuals in our sample are represented both
as interlandmark distances and shape variables. Furthermore, we used an
approach based on the construction of theoretical covariance matrices;
such matrices are used in order to estimate Type I and Type II error
rates for both methods. This work aims to analyze both methods, which
have been proposed in different contexts, under the same conceptual and
statistical framework, in order to produce meaningful comparisons.

\section{Methods}\label{methods}

\subsection{Sample}\label{sample}

The database we used here (\autoref{tab:modcomp_otu}) consists of 21
species, distributed across all taxonomic ranks within Anthropoidea
above the genus level. We selected these operational taxonomic units
(OTU) from a broader database (Marroig \& Cheverud, 2001; Oliveira
\emph{et al.}, 2009) in order to reduce the effects of low sample sizes
over estimates of modularity patterns. Individuals in our sample are
represented by 36 registered landmarks, measured using a Polhemus 3Draw
(for Platyrrhini) and a Microscribe 3DS (for Catarrhini). Twenty-two
unique landmarks represent each individual (\autoref{fig:landmarks},
\autoref{tab:lms}), since 14 of the 36 registered landmarks are
bilaterally symmetrical. For more details on landmark registration, see
Marroig \& Cheverud (2001) and Oliveira et al. (2009).

\begin{table}[t]
  \centering
  \begin{threeparttable}
    \caption{Twenty-one species used in the present work. Sample sizes and linear models adjusted over eac OTU are also indicated. \label{tab:modcomp_otu}}
    \begin{tabular}{lccr}
        \toprule
        Species & Group$^a$ & $n$ & Model$^b$ \\ 
      \midrule
        \emph{Alouatta belzebul} & P & 109 & X \\ 
        \emph{Ateles geoffroyi} & P & 78 & - \\ 
        \emph{Cacajao calvus} & P & 48 & S + X \\ 
        \emph{Callicebus moloch} & P & 93 & X \\ 
        \emph{Callithrix kuhlii} & P & 129 & - \\ 
        \emph{Cebus apella} & P & 110 & X \\ 
        \emph{Cercopithecus ascanius} & C & 61 & X \\ 
        \emph{Chiropotes chiropotes} & P & 56 & X \\ 
        \emph{Chlorocebus pygerythrus} & C & 110 & X \\ 
        \emph{Colobus guereza} & C & 140 & X \\ 
        \emph{Gorilla gorilla} & C & 115 & X \\ 
        \emph{Homo sapiens} & C & 160 & S * X \\ 
        \emph{Hylobates lar} & C & 66 & X \\ 
        \emph{Macaca fascicularis} & C & 69 & X \\ 
        \emph{Pan troglodytes} & C & 61 & X \\ 
        \emph{Papio anubis} & C & 46 & X \\ 
        \emph{Piliocolobus foai} & C & 83 & X \\ 
        \emph{Pithecia pithecia} & P & 69 & S + X \\ 
        \emph{Procolobus verus} & C & 88 & X \\ 
        \emph{Saguinus midas} & P & 50 & S \\ 
        \emph{Saimiri sciureus} & P & 87 & X \\ 
      \bottomrule
    \end{tabular}
      \begin{tablenotes}
        \footnotesize
        {
        \item[$a$] C: Catarrhini; P: Platyrrhini
        \item[$b$] S: subspecies/population; X: sex.
        }
      \end{tablenotes}
    \end{threeparttable}
  \end{table}


For each OTU, we estimated phenotypic covariance and correlation
matrices for three different types of variables: tangent space
residuals, estimated from a Procrustes superimposition for the entire
sample, using the set of landmarks described on both \autoref{tab:lms}
and \autoref{fig:landmarks} (henceforth Procrustes residuals);
interlandmark distances, described in \autoref{tab:dist}; and local
shape variables (Márquez \emph{et al.}, 2012), which are measurements of
infinitesimal log volume transformations between each sample unit and a
reference (mean) shape, based upon an interpolation function that
describes shape variation between sampled landmarks. In this context, we
used thin plate splines as interpolating functions (Bookstein, 1989). We
obtained 38 transformations corresponding to the locations of the
midpoints between pairs of landmarks used to define interlandmark
distances, in order to produce a dataset that represents shape (i.e.,
form without isometric variation; Bookstein, 1991; Zelditch \emph{et
al.}, 2004) while retaining the overall properties of the interlandmark
distance dataset, such as dimensionality for example. Furthermore, we
were able to use the same hypotheses of trait associations for both
types of variables since the position of local shape variables through
the skull mirrors the position of interlandmark distances, although they
are conceptually different types of measurements.

Here we considered only covariance or correlation structure for the
symmetrical component of variation; therefore, prior to any analysis, we
controlled the effects of variation in assymmetry. For interlandmark
distances, we averaged bilateral measurements within each individual.
For both Procrustes residuals and local shape variables, we followed the
procedure outlined in Klingenberg et al. (2002) for bilateral structures
by obtaning for each individual a symmetrical landmark configuration,
averaging each actual shape with its reflection along the sagittal
plane; we estimate local shape variables afterwards. With respect to
Procrustes residuals, landmarks placed along the sagittal plane will
have zero variation in the direction normal to this plane; we aligned
all specimens' sagittal plane to the $xz$ plane, thus removing the $y$
component for each of these landmarks from covariance/correlation
matrices.

For each dataset, we estimated covariance and correlation matrices after
removing effects of little interest in the present context, such as
sexual dimorphism, for example. For interlandmark distances and local
shape variables these effects were removed through a multivariate linear
model adjusted for each species, according to \autoref{tab:modcomp_otu};
for Procrustes residuals, the same effects were removed by centering all
group means to each species' mean shape since the loss of degrees of
freedom imposed by the GPA prohibits the use of a full multivariate
linear model over this kind of data to remove fixed effects.

In order to consider the effects of size variation on modularity
patterns we used different procedures to remove the influence of size
from each type of variable. For interlandmark distances we used the
approach established by Bookstein et al. (1985); if $\mathbf{C}$ is a
correlation matrix, we obtained a correlation matrix $\mathbf{R}$
without the effect of size using the equation

\begin{equation}
\mathbf{R} = \mathbf{C} - \lambda_1 v_1 v^{t}_1
\label{eq:iso}
\end{equation}

where $\lambda_1$ and $v_1$ refer respectively to the first eigenvalue
and eigenvector of the spectral decomposition of $\mathbf{C}$, since
this eigenvector commonly represents size variation in mammals,
especially when interlandmark distances are considered (Wagner, 1984;
Mitteroecker \emph{et al.}, 2004; Mitteroecker \& Bookstein, 2007); $t$
denotes matrix transpose.

For Procrustes residuals and local shape variables the effects of
isometric variation were removed by normalizing each individual to unit
centroid size. However, allometric relationships still influence
covariance or correlation structure. In order to remove this effect we
used a procedure based upon Mitteroecker et al. (2004), which relies on
the estimation of an allometric component $a$ for each OTU, composed of
normalized regression coefficents for each of the $m$ shape variables
(either Procrustes residuals or local shape variables) over log Centroid
Size. If $\mathbf{S}$ is a covariance matrix, we obtained a covariance
matrix $\mathbf{R}$ without the influence of allometric relationships
using the equation

\begin{equation}
\mathbf{R} = (I_m - aa^t) \mathbf{S} (I_m - aa^t)
\label{eq:allo}
\end{equation}

where $I_m$ represents the identity matrix of size $m$. Therefore, our
empirical dataset consists of six sets of covariance/correlation
matrices, corresponding to each type of morphometric variables
considering the presence or absence of size variation.

\subsection{Empirical Tests}\label{empirical-tests}

Using these six sets of covariance/correlation matrices, we tested the
hypotheses of trait associations described in \autoref{tab:lms} for
Procrustes residuals and \autoref{tab:dist} for interlandmark distances
and local shape variables. These trait sets are grouped with respect to
their scope; two regional sets (Face and Neurocranium), each divided
into three localized trait sets (Oral, Nasal and Zygomatic for the Face;
Orbit, Base and Vault for the Neurocranium).

For all hypotheses, we estimated Modularity Hypothesis Indexes (MHI;
Porto \emph{et al.}, 2013) and the RV coefficient (Klingenberg, 2009).
Both statistics are estimated by partitioning covariance or correlation
matrices into blocks; if $\mathbf{A}$ is a covariance or correlation
matrix, the partition \[
\mathbf{A} =
\begin{bmatrix}
\mathbf{A}_h & \mathbf{A}_b \\
\mathbf{A}^t_b & \mathbf{A}_c
\end{bmatrix}
\] indicates that the block $\mathbf{A}_h$ contains covariances or
correlations between traits that belong to the trait set being
considered, while $\mathbf{A}_c$ represents the complementary trait set;
$\mathbf{A}_b$ represents the block of covariances or correlations
between the two sets. Thus, covariance ($\mathbf{S}$) or correlation
($\mathbf{C}$) matrices can be partitioned into a similar scheme. We
estimated MHI values using the equation

\begin{equation}
MHI = \frac {\bar{\rho}_{+} - \bar{\rho}_{-}} {ICV}
\label{eq:mi}
\end{equation}

where $\bar{\rho}_{+}$ represents the mean correlation in
$\mathbf{C}_h$, $\bar{\rho}_{-}$ represents the mean correlation in the
remaining sets (both $\mathbf{C}_b$ and $\mathbf{C}_c$), and $ICV$ is
the coefficient of variation of eigenvalues of the associated covariance
matrix, which is a measurement of the overall integration between all
traits (Shirai \& Marroig, 2010). We estimated RV coefficients for each
hypothesis using the relationship

\begin{equation}
RV = \frac{tr(\mathbf{S}_{b}\mathbf{S}^t_{b})}{\sqrt{tr(\mathbf{S}_h \mathbf{S}_h)tr(\mathbf{S}_c \mathbf{S}_c)}}
\label{eq:rv}
\end{equation}

where $tr$ represents the sum of diagonal elements in any given matrix
($tr \mathbf{A} = \sum_i a_{ii}$).

The partitioning scheme outlined above assumes that the complementary
trait set does not represent an actual hypothesis; however, we may
choose to consider that both sets ($\mathbf{A}_h$ and $\mathbf{A}_c$)
represent two distinct hypothesis. The estimation of RV coefficients
remains the same; however, MHI values are estimated considering that
$\bar{\rho}_{+}$ is the average correlation in both $\mathbf{C}_h$ and
$\mathbf{C}_c$, while $\bar{\rho}_{-}$ represents the average
correlation only in $\mathbf{C}_b$. In the case of the distinction
between Facial and Neurocranial traits, we estimated MHI values in this
manner, reporting values for this estimate under the denomination
`Neuroface', following Marroig \& Cheverud (2001), along independent MHI
estimates for each region. Furthermore, since both Face and Neurocranium
are two disjoint trait sets when any morphometric variable type is
considered, RV coefficient values for either set are equal; therefore, a
single RV value is reported for both regions, for each variable type.

In order to test the hypothesis that a trait set represents a
variational module, we used a randomization procedure generating 1000
random trait sets with the same number of traits as the original set,
calculating MHI and RV values for each iteration. For each trait set and
covariance/correlation matrix, we used these values to construct
distributions for both statistics representing the null hypothesis that
a given trait set is a random arrangement without meaningful
relationships; we then compare this null distribution to the real value
obtained. For MHIs we consider this null hypothesis rejected when the
real value is higher than the upper bound for the distribution,
considering the significance level established; for RV coefficients, the
null hypothesis is rejected when the real RV value is lower than the
lower bound for the distribution, also considering significance level.
For Procrustes residuals the randomization procedure maintains
coordinates within the same landmark together in each randomly generated
trait set, following Klingenberg \& Leamy (2001).

While the procedure for estimating significance for MHIs is derived from
Mantel's (1967) approach (as outlined by Cheverud \emph{et al.}, 1989),
we chose to generate null distributions for MHI directly, instead of
estimating matrix correlation values for both real and randomized
matrices. Estimated $p$-values in both cases remain the same, and the
additional step of calculating matrix correlations would produce an
unnecessary difference between the estimation of signficance for MHI and
RV.

\subsection{Estimation of Error Rates}\label{estimation-of-error-rates}

We used a set of theoretical covariance matrices to investigate Type I
and II error rates for either MHI and RV metrics; the construction of
such matrices is detailed in the Supplemental Information. Here, it
suffices to say that we build two different sets of covariance matrices:
one with known modular patterns embedded, referred to as $\mathbf{C}_s$
matrices, and another that represents random covariance structure,
denominated $\mathbf{C}_r$ matrices. For each of the six sets of
empirical matrices we use here, we built a set of $10000$ covariance
matrices of each case (either $\mathbf{C}_s$ or $\mathbf{C}_r$) that
mimic the statistical properties of each set, obtaining from these
matrices samples of increasing sample size (20, 40, 60, 80, 100
individuals).

If samples were generated from a $\mathbf{C}_r$ matrix, this represents
a situation of a true null hypothesis for each test, since the
correlation matrix used to produce the sample was generated by a
permutation of the hypothesis being tested. Therefore, testing
hypotheses over $\mathbf{C}_r$ matrices allows the estimation of Type I
error rates, or the proportion of cases in which a true null hypothesis
is rejected, given a significance level. In an adequate test, we would
expect that both quantities, significance level and Type I error rate,
will be identical.

The opposite case, when we sampled $\mathbf{C}_s$ matrices, represents a
situation in which we know that the null hypothesis of either test is
false, since we are testing the hypothesis that the partitioning scheme
used to generate that particular matrix actually represents two
variational modules. Thus, we estimated Type II error rates, or the
probability that a false null hypothesis is not rejected, given a
significance level; here, we represent Type II error using the power for
each test, by simply calculating the complementar probability to Type II
error rate. In an adequate test, we expect that power would rapidly
reach a plateau when significance level is still close to zero, and
further increasing $P(\alpha)$ will not produce a great increase in
power.

Our estimates of power for both statistics should also be controlled for
effect size, since sampled correlations may generate a correlation
structure that is not detected due to small differences among within-set
and between-set correlations. For each correlation matrix sampled, we
estimate squared between-set correlations ($b^2$), in order to use it as
an estimate of effect size that is not directly associated with either
MHI and RV metrics. We expect that power for either tests decreases with
increasing $b^2$ values, as effect size would also decrease.

\subsubsection{Software}\label{software}

All analysis were performed under R 3.2.2 (R Core Team, 2015). Source
code for all analyses can be found at \url{http://github.com/wgar84}.
Previous tests indicated no differences between our estimation of
empirical RV coefficients, based upon our own code, and estimates
provided by MorphoJ (Klingenberg, 2011). In order to obtain symmetrical
landmarks configurations, we used code provided by Annat Haber,
available at \url{http://life.bio.sunysb.edu/morph/soft-R.html}.

\section{Results}\label{results}

\subsection{Empirical Tests}\label{empirical-tests-1}

Tests performed using MHI for localized trait sets (Oral, Nasal,
Zygomatic, Orbit, Base, and Vault; \autoref{fig:Func}a) detected a
consistent pattern among OTUs for interlandmark distances and local
shape variables. In the first set the Oral subregion was detected as a
modular partition, and, with size removed, the Vault subregion was also
detected; both Orbit and Base region were not detected in any of these
tests. With local shape variables, Oral, Vault Nasal, and Zygomatic sets
were detected consistently across OTUs; the removal of allometric
variation affected only the detection of the Vault in some groups.
Furthermore, the Base sub-region was detected only in 3 of 42 tests
performed over local shape variables, pooling together size and
size-free correlation patterns.

For Procrustes residuals, the pattern of detection among sub-regions and
OTUs was more inconsistent; for instance, the Base sub-region was
detected in several OTUs, which contrasts this type of morphometric
variable with the other two types. Observing the actual MHI values,
\autoref{fig:Func}a also indicates that Procrustes residuals had a low
variance of this metric within each OTU, while interlandmark distances
and local shape variables have a consistent pattern of variation, with
lower values for the Base and, for interlandmark distances, Orbit trait
sets, while the Oral, Nasal and Vault regions displayed higher values
consistently.

\begin{figure}[htbp]
\centering
\includegraphics{Figures/Func-1.pdf}
\caption{MHI (a) and RV (b) values for localized trait sets. Circles
indicate whether a trait set is recognized as a variational module in a
given OTU, with $P(\alpha)$ indicated by the legend. Notice that values
represented in blue are either high MHI values and low RV values, as the
alternate hypothesis for each statistic is formulated accordingly; see
`Methods' for details. \label{fig:Func}}
\end{figure}

Tests performed using RV coefficients (\autoref{fig:Func}b) show a more
irregular pattern for each variable type. When interlandmark distances
were considered, most tests detected the Vault sub-region with size
variation retained, and the Base sub-region when size variation was
removed. For Procrustes residuals, few tests were able to reject the
null hypotheses, detecting only a handful of valid modular partitions.
Tests performed on local shape variables displayed the opposite
behavior: almost all partitions were detected, regardless of whether
allometric variation had been retained or removed. Moreover, RV values
displayed a pattern of marked variation among OTUs, more so than between
values within each OTU; notably, \emph{Macaca fascicularis} and
\emph{Papio anubis} showed RV values much higher than those estimated on
remaining species. Such pattern can be observed both on interlandmark
distances with size retained and in Procrustes residuals.

With respect to regional trait sets (Face and Neurocranium), tests
performed using MHI (\autoref{fig:Dev}a) indicate a pattern consistent
with those found on localized sets (\autoref{fig:Func}a). Considering
interlandmark distances, Facial traits were detected as a valid modular
partition both with size variation retained or removed, while
Neurocranial traits were detected as a valid partition only when size
variation was removed. This pattern mirrors the contrast between Oral
and Vault traits found in the localized sets regarding interlandmark
distances. For Procrustes residuals, Neurocranial traits were detected
as a valid partition with both size retained and removed; once again,
this pattern mirrors the detection of the Basicranial partition found in
the localized sets. Finally, in local shape variables, both Face and
Neurocranium were detected as valid with size retained; with size
removed, only the Face was recognized consistently. The same can be
observed for localized sets, where the removal of allometric variation
affected the detection of the Vault set in some OTUs. The tests for the
distinction of within-set and between-set correlations for these two
sets (designated `Neuroface') show a pattern that is consistent with
tests for the individual sets: if one of the sets was previously
detected, this distinction was also detected as valid.

\begin{figure}[htbp]
\centering
\includegraphics{Figures/Dev-1.pdf}
\caption{MHI (a) and RV (b) values for regional trait sets. Circles
indicate whether a trait set is recognized in a given OTU, with
$P(\alpha)$ indicated by the legend. Notice that values represented in
blue are either high MHI values and low RV values, as the alternate
hypothesis for each statistic is formulated accordingly; see `Methods'
for details. \label{fig:Dev}}
\end{figure}

Testing the distinction between Face and Neurocranium using RV
coefficients (\autoref{fig:Dev}b) showed that in most cases both regions
are considered distinct and valid variational modules for interlandmark
distances and local shape variables; for Procrustes residuals, only a
handful of taxa indicated the same result. In this case, the
correspondence with localized trait sets (\autoref{fig:Func}b) is more
difficult due to the lack of independent tests for each region.

\subsection{Error Rates}\label{error-rates}

Comparing the distributions of MHI and RV values from theoretical
matrices with respect to their structure (\autoref{fig:stat_dist_sim})
indicated marked differences between metrics and morphometric variables
from which correlations were sampled. In general, the distribution of
MHI values obtained from $\mathbf{C}_r$ matrices was the same regardless
of type of variable, while the distributions for $\mathbf{C}_s$ matrices
for this metric were heterogeneous; for either interlandmark distances
and local shape variables, the distribution of MHI values for these two
types of matrices did not overlap, while for Procrustes residuals, these
distributions overlapped to some degree. For RV values, all
distributions overlap to some degree; for local shape variables and
interlandmark distances with size retained, either distributions
($\mathbf{C}_r$ and $\mathbf{C}_s$) overlapped to a lesser extent.

\begin{figure}[htbp]
\centering
\includegraphics{Figures/stat_dist_sim-1.pdf}
\caption{Distribution of Modularity Hypothesis Index and RV Coefficient
for theoretical correlation matrices. \label{fig:stat_dist_sim}}
\end{figure}

The relationship between significance levels and Type I error rates
estimated over $\mathbf{C}_r$ matrices approached an identity
relationship (\autoref{fig:type1}), regardless of whether variational
modularity was estimated using MHI or RV; even at low sample sizes Type
I error rates were very close to significance levels. Furthermore, the
effect of sampling correlations from size-free distributions did not
change Type I error rates.

\begin{figure}[htbp]
\centering
\includegraphics{Figures/type1-1.pdf}
\caption{Type I error rates as a function of the chosen significance
level regarding tests for variational modularity applied on
$\mathbf{C}_r$ correlation matrices. The solid black line represents the
identity relationship. \label{fig:type1}}
\end{figure}

With respect to the relationship between power and significance levels
(estimated over $\mathbf{C}_s$ matrices), there were substantial
differences with respect to the chosen metric (MHI or RV) and to the
type of variable that provides sampled correlations. Considering local
shape variables (\autoref{fig:type2_def}), tests using either MHI and RV
implied high power, even at low sample or effect sizes; increasing these
quantities further increased power. However, for lower effect sizes
(represented by high average squared correlation between sets, $b^2$)
power for tests using MHI was higher than for those using RV; as effect
size increased (lower $b^2$ values), the difference in power between the
two statistics decreased. For local shape variables, sampling from its
associated size-free correlation distribution implied minor differences
in power for both statistics.

\begin{figure}[htbp]
\centering
\includegraphics{Figures/type2_def-1.pdf}
\caption{Power for both MHI and RV statistics as a function of the
chosen significance levels with respect to tests for variational
modularity applied on $\mathbf{C}_s$ matrices sampled from the
distribution of correlations between local shape variables. Lines are
colored with respect to quantiles of the $b^2$ distribution.
\label{fig:type2_def}}
\end{figure}

For interlandmark distances (\autoref{fig:type2_ed}) there were
substantial differences on the relationship between power and
significance level. In general, power for tests using MHI are always
higher than for tests using RV; this effect is more pronounced on
$\mathbf{C}_s$ matrices derived from size-free interlandmark distances,
although tests on these matrices have a substantial decrease in power
for either tests. However, this decrease is more pronounced for tests
based on the RV statistic, since for lower effect sizes (high $b^2$
values) power approaches an identity relationship with significance
level. Sample size also interferes with this relationship since
increasing this quantity also increases power when higher effect sizes
(low $b^2$ values) are considered.

\begin{figure}[htbp]
\centering
\includegraphics{Figures/type2_ed-1.pdf}
\caption{Power for both MHI and RV statistics as a function of the
chosen significance levels with respect to tests for variational
modularity applied on $\mathbf{C}_s$ correlation matrices sampled from
the distribution of correlations between interlandmark distances. Lines
are colored with respect to quantiles of the $b^2$ distribution. The
solid black line represents the identity relationship.
\label{fig:type2_ed}}
\end{figure}

With respect to Procrustes residuals (\autoref{fig:type2_sym}), tests
using either MHI or RV have reduced power regardless of effect or sample
size. Sampling from size-free correlation distributions to build
$\mathbf{C}_s$ matrices also has little effect. In this case, power for
tests performed using RV values approaches an identity relationship with
significance level; increasing sample size has some effect, but it does
not increase power above 50\% in any case.

\begin{figure}[htbp]
\centering
\includegraphics{Figures/type2_sym-1.pdf}
\caption{Power for both MHI and RV statistics as a function of the
chosen significance levels with respect to tests for variational
modularity applied on $\mathbf{C}_s$ matrices sampled from the
distribution of correlations between Procrustes residuals. Lines are
colored with respect to quantiles of the $b^2$ distribution. The solid
black line represents the identity relationship. \label{fig:type2_sym}}
\end{figure}

\section{Discussion}\label{discussion}

Covariance matrices derived from morphological traits are supposed to
represent the pattern of codependence that stems from interactions among
developing morphological elements (Olson \& Miller, 1958; Cheverud,
1996). Such interactions are the expression of local developmental
factors, as they interact with the gene expression profiles of
surrounding cell types, producing coordinated changes in cellular growth
and tissue interaction, thus integrating these elements in the adult
population. Although these events of local integration overlap, and the
composed effect over adult covariance patterns may be confusing
(Hallgrímsson \& Lieberman, 2008; Hallgrímsson \emph{et al.}, 2009), we
believe that a careful comparison of different yet equally proper ways
of measuring and representing form may be informative of the underlying
processes that produce covariances.

Due to the minimization of quadratic distances among homologous
landmarks during GPA, covariance matrices derived from Procrustes
estimators lose the signal of localized effects on covariance patterns
(Linde \& Houle, 2009). Therefore, the use of Procrustes estimators to
investigate morphological integration or modularity implicitly implies
in a divorce between the phenomenom under investigation and the chosen
representation. This disconnection between theory and measurement may
have serious consequences for the hypotheses we wish to test (Houle
\emph{et al.}, 2011); such consequences are observable in both our
empirical tests and those tests performed on theoretical matrices, as we
explore below.

The mammalian Basicranium originates from thirteen precursor tissues
derived from both paraxial mesoderm and neural crest, and they may merge
to form single bones, such as the sphenoid (Jiang \emph{et al.}, 2002;
Lieberman, 2011). Furthermore, these precursors display a mosaic of
endochondral and intramembranous ossification early in development, and,
as the brain grows, it induces a pattern of internal resorption and
exterior deposition on the underlying posterior Basicranium (Lieberman
\emph{et al.}, 2000); meanwhile, the anterior portion suffers influence
from the development of Facial elements (Bastir \& Rosas, 2005). Thus,
since the Basicranium ossifies early in development, the composed effect
of all posterior steps of cranial development will overshadow any
pattern of integration this region might have, as predicted by the
palimpsest model of development (Hallgrímsson \emph{et al.}, 2009).
Moreover, the angulation between anterior and posterior elements of the
Basicranium has significantly changed during primate evolution, and such
property appears to have evolved in coordination with Facial growth
relative to the cranial Vault, accomodating both structures on each
other (Scott, 1958; Lieberman \emph{et al.}, 2000, 2008).

Due to this heterogeneity of developmental processes acting on the
Basicranium, it is not expected that this region would be detected as a
variational module; thus, a test of this property over skull covariance
patterns should fail to reject the null hypothesis of random
association. However, considering the 42 tests performed over covariance
matrices derived from Procrustes residuals regarding localized
hypotheses (midpanels of \autoref{fig:Func}a), the Basicranium is
detected as a valid variational module in 27 cases, distributed through
matrices with size either retained or removed; in some cases (e.g.
\emph{Alouatta}, \emph{Cercopithecus}), only the Basicranium is
detected. Hence, Procrustes residuals show a pattern of detection of
variational modules opposed to the expectation based on developmental
dynamics. In covariances matrices derived from interlandmark distances
(upper panels of \autoref{fig:Func}a), the Basicranium was never
detected; with local shape variables (lower panels of
\autoref{fig:Func}a), the Basicranium was detected only three out of 42
times.

These two remaining types of representations (i.e.~interlandmark
distances and local shape variables) show patterns of detection of
variational modules that are consistent with the expectations derived
from developmental and functional interactions, especially with respect
to how these patterns articulate with the retention or removal of size
variation. Considering that interlandmark distances are on a ratio scale
(Houle \emph{et al.}, 2011), isometric variation will be represented to
a greater extent when compared to subtle allometric relationships, due
to the multiplicative nature of biological growth (Huxley, 1932).
Therefore, the Oral trait set is detected as a valid variational module,
considering covariance matrices among interlandmark distances (upper
left panel of \autoref{fig:Func}a), since this region is strongly
affected by the induction of bone growth due to muscular activity
beginning in the pre-weaning period (Zelditch \& Carmichael, 1989;
Hallgrímsson \emph{et al.}, 2009). Furthermore, in these matrices,
allometric interactions are associated with the strength of association
between traits in Oral, Nasal, Zygomatic and Vault regions in an
evolutionary scale (Chapter 3); thus, allometric relationships also play
some role in the determination of covariance/correlation patterns for
other subregions, although such pattern may be masked in interlandmark
distances by the effect of isometric variation. On the other hand, the
patterns expressed in covariance/correlation matrices for local shape
variables are influenced by allometric relationships defined on a proper
scale (Jolicoeur, 1963; Houle \emph{et al.}, 2011; Márquez \emph{et
al.}, 2012) and local developmental processes. Thus, they reflect this
association between integration and allometry.

By removing allometric effects from local shape variables (lower right
panel of \autoref{fig:Func}a), variational modularity can still be
detected in both Oral and Nasal regions, while in a number of species,
the Vault region is no longer detected as a variational module. Vault
integration may be achieved through both allometric relationships and
the effect of relative brain growth, since Vault elements arise mostly
through intramembranous ossification, induced by the secretion of
signaling factors from the outer brain tissues, with a modest but
necessary contribution of mesoderm-derived tissue that undergoes
endochondral ossification (Jiang \emph{et al.}, 2002; Rice \emph{et
al.}, 2003; Franz-Odendaal, 2011; Lieberman, 2011). In humans, the
growth of Vault osteological elements occur, from a topological
standpoint, without deviations from an isometric growth model, forming
regular connections among bones (Esteve-Altava \& Rasskin-Gutman, 2014),
since the boundary of interactions between brain and Vault bones is also
regular. Thus, as this effect dominates the latter stages of pre-natal
development (Hallgrímsson \& Lieberman, 2008; Lieberman, 2011), the
overall effect of brain growth over skull growth patterns at this stage
may mirror an affine transformation, rendering it undetectable over
covariance patterns derived from local shape variables, since their
estimation excludes shape variation associated with affine
transformations (Márquez \emph{et al.}, 2012). In contrast, Oral and
Nasal elements have more complex patterns of connectivity arising from
their tight integration with the soft tissues that compose the remaining
elements of the Face (Lieberman, 2011; Esteve-Altava \emph{et al.},
2013; Esteve-Altava \& Rasskin-Gutman, 2014), thus producing an
intrincate pattern of associations that may be responsible for the
variational modularity we are able to detect in all species, regardless
of whether size variation is retained or removed.

The differences between the pattern of detection in all representations
for local trait sets for RV values (\autoref{fig:Func}b) makes a similar
interpretation of the results for this metric difficult, as opposed to
the results regarding MHI values. The variance of RV values among
species seems to indicate that RV values are sensitive to the magnitude
of morphological integration, since species with high RV values, such as
\emph{Papio anubis} and \emph{Macaca fascicularis}, are within genera
that also exhibit higher magintudes among Catarrhines (Oliveira \emph{et
al.}, 2009), at least when we consider interlandmark distances.
Interestingly enough, this effect of magnitude of integration (which is
thought to emerge as a consequence of size variation; Wagner, 1984;
Marroig \& Cheverud, 2001, 2005; Porto \emph{et al.}, 2013) seems to
also affect Procrustes residuals regardless of whether covariance
structure arising from allometric relationships is removed or not.

For the set of hypotheses concerning differences between Facial and
Neurocranial traits (\autoref{fig:Dev}), there is substantial agreement
between tests performed using MHI and RV values. For MHI values
(\autoref{fig:Dev}a), the overall pattern of detection is similar to the
pattern detected in local trait sets, for all variable types; for RV
values (\autoref{fig:Dev}b), there is strong support for the hypotheses
that both Face and Neurocranium represent variational modules, in both
local shape variables and interlandmark distances. These regions have
marked differences in developmental timing and pattern formation, as
observed from the behavior of their composing units (Zelditch \&
Carmichael, 1989; Hallgrímsson \emph{et al.}, 2009; Lieberman, 2011;
Esteve-Altava \& Rasskin-Gutman, 2014); therefore, the more general
pattern of distinction between Face and Neurocranium is detected
regardless of the metric chosen to represent modularity.

\subsection{Theoretical Matrices and Error
Rates}\label{theoretical-matrices-and-error-rates}

The distribution of MHI and RV values obtained from the theoretical
matrices (\autoref{fig:stat_dist_sim}) is a starting point for
understanding the differences in power for tests using these two metrics
(Figures \ref{fig:type2_def}--\ref{fig:type2_sym}). For MHI values, the
distribution obtained from random ($\mathbf{C}_r$) matrices is
consistently the same, regardless of what representation we used to
sample correlations, or whether size was retained or removed. On the
other hand, the distribution of RV values for random matrices change
depending on the representation sampled or whether size variation has
been removed or retained. Moreover, this change in behavior for the
distribution of RV values for $\mathbf{C}_r$ matrices implies different
levels of overlap between these null distributions and the distribution
of values obtained for structured ($\mathbf{C}_s$) matrices.

For Procrustes residuals, a substantial overlap occurs regardless of
whether size variation was removed or retained; unsurprisingly, our
estimates of power for RV in this type of representation are very low
(\autoref{fig:type2_sym}). Nonetheless, power estimated for tests based
on MHI is lower than in other types of representation, since the
difference between within-set and between-set correlations in Procrustes
residuals (\autoref{fig:cor_dist}) is the lowest of all representations.
A substantial overlap in RV distributions for $\mathbf{C}_r$ and
$\mathbf{C}_s$ matrices also occurs with interlandmark distances when
size variation is removed, and it results in low power for tests using
RV values in this type of representation (right column of
\autoref{fig:type2_ed}). However, in this case there is a substantial
difference between within-set and between-set correlations
(\autoref{fig:cor_dist}), and tests using MHI to represent modularity
are still able to detect such difference (albeit with reduced power)
when compared to tests over $\mathbf{C}_s$ matrices derived from
interlandmark distances with size retained (\autoref{fig:type2_ed}). In
those cases where both distributions for $\mathbf{C}_r$ and
$\mathbf{C}_s$ matrices do not overlap substantially --- for example,
when local shape variables are considered (\autoref{fig:type2_def}) ---
power for tests performed using MHI values is always higher than for
tests using RV except when sample sizes are very high; in this case
power for both metrics are similar. The same behavior can also be
observed in interlandmark distances when size is retained (left column
of \autoref{fig:type2_ed}).

These results indicate that RV coefficents are more sensitive to the
absolute value of both within-set and between-set correlation
distributions than MHI values. For interlandmark distances
(\autoref{fig:type2_ed}), removing size variation reduced the average
value of both correlation distributions by a similar amount
(\autoref{fig:cor_dist}); the difference between average correlations in
these two sets actually increases, going from 0.042 to 0.061 when size
is removed. However, since the actual average correlations for these two
distributions approach zero, tests based on RV lose power more rapidly
than tests based on MHI. This sensitivity might be associated to the use
of squared covariances, as shown by Equation \ref{eq:rv}, while
Modularity Hypothesis Indexes use correlations directly (Equation
\ref{eq:mi}). Furthermore, as observed by Fruciano et al. (2013), sample
sizes influence the estimation of RV coefficients, and we demostrate
here that such sensitivity also extends to estimates of power for tests
using this metric.

Our estimates of power for tests using MHI indicate that it is more
robust to differences in absolute correlation values or sample sizes,
thus allowing comparisons across more heterogeneous settings, such as
our comparison between different representations of form, with
substantial variation in sample sizes. The detection of variational
modularity is akin to Student's \emph{t} test, since we try to determine
whether two groups of observations (correlations between traits in the
same subset \emph{versus} correlations between traits in different
subsets) have a significant difference in average values; we use
resampling procedures to estimate significance in this case only due to
the interdependency between pairwise correlations. Thus, as the
estimation of MHI values resembles the estimation of \emph{t}-values
(difference between location parameters for two groups, divided by a
scale parameter --- ICV; Equation \ref{eq:mi}), we believe that this
statistic is appropriate to represent variational modularity, and that
is reinforced by the robustness of the tests which employed MHI.

Our approach for constructing theoretical matrices attempts to simulate
the simplest situation: only two subsets of traits, akin to the
distinction between Face and Neurocranium in our empirical dataset
(\autoref{fig:Dev}). In this setting, both statistics are capable of
detecting this distinction, except when both are used on
covariance/correlation patterns derived from Procrustes residuals.
However, even though we built theoretical matrices using correlations
sampled from these estimators, to simulate the interference in
covariance structure that such estimators produce in our theoretical
matrices is quite difficult. Furthermore, constructing such matrices
with more complicated patterns (with three modules, for instance) while
maintaining their connection to the correlation distributions of each
morphometric type is also difficult, due to the restriction on
positive-definiteness we enforce on them. Thus, the lack of differences
in type I error rates for all cases may be a limitation of our scheme
for building theoretical matrices.

The issues we found with the use of Procrustes estimators for covariance
matrices and the use of RV coefficients to estimate and detect
variational modularity may explain results found by other authors. For
instance, Martínez-Abadías et al. (2011) has found no evidence that
genetic and phenotypic covariance structure for human skulls conforms to
a modular structure, since all tests performed by these authors failed
to reject the null hypothesis of random association. These authors use
Procrustes estimators to represent covariance structure, and test their
hypothesis of partitioning (Face, Vault and Base) using RV as the
statistic representing variational modularity. Since this combination
yields very low power (\autoref{fig:type2_sym}), not rejecting the null
hypothesis in their case might be a consequence of the choice of
estimates of both covariance structure and variational modularity; thus,
these authors' conclusion of pervasive genetic integration in the human
skull might be misleading, considering that skull covariance patterns in
humans are one of the most modular examples of such patterns when
compared to other mammals (Porto \emph{et al.}, 2009) or catarrhine
primates (Oliveira \emph{et al.}, 2009).

The approach we explore in the present work is only one of the different
alternatives to investigate the association between genetic, functional
and developmental interactions and correlation structure (Mitteroecker
\& Bookstein, 2007). For example, Perez et al. (2009) relies on
abstracting correlation matrices into networks, then using
community-detection algorithms to search for modular patterns without
\emph{a priori} hypotheses, associating their results with knockout
experiments that support the communities they found among traits;
however, it is not clear how much relevant biological information is
retained in these network representations. Furthermore, the authors use
Procrustes estimators, which may bias the detection of modularity
patterns in this setting in the same manner as we demonstrated here.

Monteiro et al. (2005) assumes that the underlying morphogenetic
components of the rodent mandible behave as modules, further
investigating the patters of correlation between these units in both
within-species variation and between-species variation among Echimids;
Monteiro \& Nogueira (2010), relying on the correspondence of these
units through mammalian diversification did the same to phylostomid
bats. Although using a landmark-based approach to represent
morphological variation, the authors do not use Procrustes estimators to
represent covariance structure among these units, and the pattern of
reorganization of correlation structure among these units associated
with niche diversification in phylostomids seems robust, considering
that this radiation may have been associated with a very heterogeneous
adaptive landscape, and such heterogeneity may lead to a reorganization
of correlation patterns (Jones, 2007; Jones \emph{et al.}, 2012; Melo \&
Marroig, 2015).

Another valid approach is to model certain aspects of development as
null hypotheses; Esteve-Altava \& Rasskin-Gutman (2014), investigating
the pattern of connections among human cranial bones, formulates the
null hypothesis that unconstrained bone growth is sufficient to explain
the observed patterns, further constructing a null network based on this
expectation which is compared to the actual network. Such approach could
be extended to investigate covariance structure; considering the
geometric properties of the features we measure (Mitteroecker \&
Bookstein, 2007), one could formulate the null hypothesis that
topological proximity is a sufficient explanation for the observed
covariance structure, against the alternative hypothesis that local
developmental processes coupled with functional interactions produce
stronger relationships among close elements that surpass topological
interactions. Alternatively, one could actively look for variational
boundaries between regions, as boundary formation is a phenomenom that
has been well studied under a dynamical perspective on development (e.g.
Turing, 1952; Meinhardt, 1983; Tiedemann \emph{et al.}, 2012).

The approach of partitioning covariance matrices into blocks that
correspond to inferred modular associations has the advantage that it is
operationally simple; however, modularity patterns are almost certainly
not expressed in phenotypic data as the binary hypotheses we used here
(Hallgrímsson \emph{et al.}, 2009). Thus, hypotheses and inferences
derived from them have to be contextualized in the light of
developmental dynamics, since the measurements we make and the
parameters we estimate have to be properly connected to the models we
are considering; otherwise, inferences made from such models may be
devoid of meaning (Wagner, 2010; Houle \emph{et al.}, 2011).

\subsection{Conclusion}\label{conclusion}

Here we show that Procrustes estimators for covariance matrices fail to
capture the modularity patterns embedded in phenotypic data, regardless
of which metric is chosen to represent such patterns, although the
combination of this type of variable with RV coefficients for
investigating modularity has even more problems than either has alone.
Both interlandmark distances and local shape variables seem valid
options to represent morphological variation, if their limitations are
taken into consideration. We wish to stress this point: any
representation of morphological variation has limitations since they are
themselves models --- at the very least of what it is important to
represent --- not fully capturing the phenomena we may be interested in.

\subsubsection{Acknowledgements}\label{acknowledgements}

We thank G. Burin, D. Melo, and A. Porto for comments on an early draft.
This work has been funded by CNPq (Conselho Nacional de Pesquisa e
Desenvolvimento Tecnológico) and FAPESP (Fundação de Apoio à Pesquisa do
Estado de São Paulo).

\section*{References}\label{references}
\addcontentsline{toc}{section}{References}

Adams, D.C. \& Otárola-Castillo, E. 2013. geomorph: an r package for the
collection and analysis of geometric morphometric shape data.
\emph{Methods in Ecology and Evolution} \textbf{4}: 393--399.

Adams, D.C., Rohlf, F.J. \& Slice, D.E. 2004. Geometric morphometrics:
Ten years of progress following the ``revolution''. \emph{Italian
Journal of Zoology} \textbf{71}: 5--16.

Andrade, R.F.S., Rocha-Neto, I.C., Santos, L.B.L., Santana, C.N. de,
Diniz, M.V.C. \& Lobão, T.P.\emph{et al.} 2011. Detecting Network
Communities: An Application to Phylogenetic Analysis. \emph{PLoS
Computational Biology} \textbf{7}: e1001131.

Arnold, S.J., Bürger, R., Hohenlohe, P.A., Ajie, B.C. \& Jones, A.G.
2008. Understanding the evolution and stability of the G-matrix.
\emph{Evolution} \textbf{62}: 2451--2461.

Bastir, M. \& Rosas, A. 2005. Hierarchical nature of morphological
integration and modularity in the human posterior face. \emph{American
Journal of Physical Anthropology} \textbf{128}: 26--34.

Bookstein, F.L. 1982. Foundations of Morphometrics. \emph{Annual Review
of Ecology and Systematics} \textbf{13}: 451--470.

Bookstein, F.L. 1991. \emph{Morphometric tools for landmark data:
geometry and biology}. Cambridge University Press, Cambridge.

Bookstein, F.L. 1989. Principal warps: Thin plate splines and the
decomposition of deformations. \emph{IEEE Transactions on Pattern
Analysis and Machine Intelligence} \textbf{11}: 567--585.

Bookstein, F.L., Chernoff, B., Elder, R., Humphries, Smith, G. \&
Strauss, R. 1985. \emph{Morphometrics in Evolutionary Biology}. The
Academy of Natural Sciences of Philadelphia, Philadelphia.

Cardini, A. \& Polly, P.D. 2013. Larger mammals have longer faces
because of size-related constraints on skull form. \emph{Nature
Communications} \textbf{4}.

Cheverud, J.M. 1996. Developmental integration and the evolution of
pleiotropy. \emph{American Zoology} \textbf{36}: 44--50.

Cheverud, J.M. 1982. Phenotypic, genetic, and environmental
morphological integration in the cranium. \emph{Evolution} \textbf{36}:
499--516.

Cheverud, J.M. \& Richtsmeier, J.T. 1986. Finite-Element Scaling Applied
to Sexual Dimorphism in Rhesus Macaque (Macaca Mulatta) Facial Growth.
\emph{Systematic Biology} \textbf{35}: 381--399.

Cheverud, J.M., Kohn, L.A.P., Konigsberg, L.W. \& Leigh, S.R. 1992.
Effects of fronto-occipital artificial cranial vault modification on the
cranial base and face. \emph{American Journal of Physical Anthropology}
\textbf{88}: 323--345.

Cheverud, J.M., Routman, E.J. \& Irschick, D.J. 1997. Pleiotropic
Effects of Individual Gene Loci on Mandibular Morphology.
\emph{Evolution} \textbf{51}: 2006--2016.

Cheverud, J.M., Wagner, G.P. \& Dow, M.M. 1989. Methods for the
comparative analysis of variation patterns. \emph{Evolution}
\textbf{38}: 201--213.

Drake, A.G. \& Klingenberg, C.P. 2010. Large Scale Diversification of
Skull Shape in Domestic Dogs: Disparity and Modularity. \emph{The
American Naturalist} \textbf{175}: 289--301.

Escoufier, Y. 1973. Le Traitement des Variables Vectorielles.
\emph{Biometrics} \textbf{29}: 751--760.

Espinosa-Soto, C. \& Wagner, A. 2010. Specialization can drive the
evolution of modularity. \emph{PLoS Comput. Biol.} \textbf{6}: e1000719.

Esteve-Altava, B. \& Rasskin-Gutman, D. 2014. Beyond the functional
matrix hypothesis: a network null model of human skull growth for the
formation of bone articulations. \emph{Journal of Anatomy} \textbf{225}:
306--316.

Esteve-Altava, B., Marugán-Lobón, J., Botella, H., Bastir, M. \&
Rasskin-Gutman, D. 2013. Grist for Riedl's mill: A network model
perspective on the integration and modularity of the human skull.
\emph{Journal of Experimental Zoology Part B: Molecular and
Developmental Evolution} \textbf{320}: 489--500.

Falconer, D.S. \& Mackay, T.F.C. 1996. \emph{Introduction to
Quantitative Genetics}, 4th ed. Addison Wesley Longman, Harlow, Essex.

Fortuna, M.A., García, C., Guimarães Jr., P.R. \& Bascompte, J. 2008.
Spatial mating networks in insect-pollinated plants. \emph{Ecology
Letters} \textbf{11}: 490--498.

Franz-Odendaal, T.A. 2011. Epigenetics in Bone and Cartilage
Development. In: \emph{Epigenetics: Linking Genotype and Phenotype in
Development andEvolution} (B. Hallgrímsson \& B. K. Hall, eds), pp.
195--220. University of California Press.

Fruciano, C., Franchini, P. \& Meyer, A. 2013. Resampling-Based
Approaches to Study Variation in Morphological Modularity. \emph{PLoS
ONE} \textbf{8}: e69376.

Genini, J., Morellato, L.P.C., Guimarães Jr., P.R. \& Olesen, J.M. 2010.
Cheaters in mutualism networks. \emph{Biology Letters} \textbf{6}:
494--497.

Goodall, C. 1991. Procrustes methods in the statistical analysis of
shape. \emph{Journal of the Royal Statistical Society. Series B
(Methodological)} \textbf{53}: 285--339.

Goswami, A. \& Polly, P.D. 2010. The influence of modularity on cranial
morphological disparity in Carnivora and Primates (Mammalia). \emph{PLoS
ONE} \textbf{5}: e9517.

Grabowski, M.W., Polk, J.D. \& Roseman, C.C. 2011. Divergent patterns of
integration and reduced constraint in the human hip and the origins of
bipedalism. \emph{Evolution} \textbf{65}: 1336--1356.

Haber, A. 2015. The Evolution of Morphological Integration in the
Ruminant Skull. \emph{Evolutionary Biology} \textbf{42}: 99--114.

Hallgrímsson, B. \& Lieberman, D.E. 2008. Mouse models and the
evolutionary developmental biology of the skull. \emph{Integrative and
Comparative Biology} \textbf{48}: 373--384.

Hallgrímsson, B., Jamniczky, H., Young, N.M., Rolian, C., Parsons, T.E.
\& Boughner, J.C.\emph{et al.} 2009. Deciphering the Palimpsest:
Studying the Relationship Between Morphological Integration and
Phenotypic Covariation. \emph{Evolutionary Biology} \textbf{36}:
355--376.

Herring, S.W. 2011. Muscle-Bone Interactions and the Development of
Skeletal Phenotype. In: \emph{Epigenetics: Linking Genotype and
Phenotype in Development andEvolution} (B. Hallgrímsson \& B. K. Hall,
eds), pp. 221--237. University of California Press.

Houle, D., Pélabon, C., Wagner, G.P. \& Hansen, T.F. 2011. Measurement
and Meaning In Biology. \emph{The Quartely Review of Biology}
\textbf{86}: 3--34.

Huckemann, S. 2011. Inference on 3D Procrustes Means: Tree Bole Growth,
Rank Deficient Diffusion Tensors and Perturbation Models: Inference on
3D Procrustes means. \emph{Scandinavian Journal of Statistics} no--no.

Huckemann, S.F. 2012. On the meaning of mean shape: manifold stability,
locus and the two sample test. \emph{Annals of the Institute of
Statistical Mathematics} \textbf{64}: 1227--1259.

Huxley, J.S. 1932. \emph{Problems of relative growth}.

Jiang, X., Iseki, S., Maxson, R.E., Sucov, H.M. \& Morriss-Kay, G.M.
2002. Tissue Origins and Interactions in the Mammalian Skull Vault.
\emph{Developmental Biology} \textbf{241}: 106--116.

Jolicoeur, P. 1963. The Multivariate Generalization of the Allometry
Equation. \emph{Biometrics}.

Jones, A.G. 2007. The mutation matrix and the evolution of evolvability.
\emph{Evolution} \textbf{61}: 727--745.

Jones, A.G., Bürger, R., Arnold, S.J., Hohenlohe, P.A. \& Uyeda, J.C.
2012. The effects of stochastic and episodic movement of the optimum on
the evolution of the G-matrix and the response of the trait mean to
selection. \emph{Journal of evolutionary biology} 1--22.

Kendall, D.G. 1984. Shape manifolds, procrustean metrics, and complex
projective spaces. \emph{Bulletin of the London Mathematical Society}
\textbf{16}: 81--121.

Kent, J.T. \& Mardia, K.V. 1997. Consistency of Procrustes Estimators.
\emph{Journal of the Royal Statistical Society: Series B (Statistical
Methodology)} \textbf{59}: 281--290.

Klingenberg, C.P. 2011. MorphoJ: an integrated software package for
geometric morphometrics. \emph{Molecular Ecology Resources} \textbf{11}:
353--357.

Klingenberg, C.P. 2009. Morphometric integration and modularity in
configurations of landmarks: tools for evaluating a priori hypotheses.
\emph{Evolution \& Development} \textbf{11}: 405--421.

Klingenberg, C.P. \& Leamy, L.J. 2001. Quantitative genetics of
geometric shape in the mouse mandible. \emph{Evolution} \textbf{55}:
2342--2352.

Klingenberg, C.P., Barluenga, M. \& Meyer, A. 2002. Shape analysis of
symmetric structures: Quantifying variation among individuals and
asymmetry. \emph{Evolution} \textbf{56}: 1909--1920.

Klingenberg, C.P., Leamy, L.J. \& Cheverud, J.M. 2004. Integration and
Modularity of Quantitative Trait Locus Effects on Geometric Shape in the
Mouse Mandible. \emph{Genetics} \textbf{166}: 1909--1921.

Lele, S. 1993. Euclidean distance matrix analysis (EDMA): estimation of
mean form and mean form difference. \emph{Mathematical Geology}
\textbf{25}: 573--602.

Lele, S.R. \& McCulloch, C.E. 2002. Invariance, Identifiability, and
Morphometrics. \emph{Journal of the American Statistical Association}
\textbf{97}: 796--806.

Lieberman, D.E. 2011. Epigenetic Integration, Complexity and
Evolvability of the Head: Rethinking the Functional Matrix Hypothesis.
In: \emph{Epigenetics: Linking Genotype and Phenotype in Development and
Evolution} (B. Hallgrímsson \& B. K. Hall, eds), pp. 271--289.
University of California Press.

Lieberman, D.E., Hallgrímsson, B., Liu, W., Parsons, T.E. \& Jamniczky,
H.A. 2008. Spatial packing, cranial base angulation, and craniofacial
shape variation in the mammalian skull: testing a new model using mice.
\emph{Journal of Anatomy} \textbf{212}: 720--735.

Lieberman, D.E., Ross, C.F. \& Ravosa, M.J. 2000. The primate cranial
base: Ontogeny, function, and integration. \emph{American Journal of
Physical Anthropology} \textbf{113}: 117--169.

Linde, K. van der \& Houle, D. 2009. Inferring the Nature of Allometry
from Geometric Data. \emph{Evolutionary Biology} \textbf{36}: 311--322.

Lynch, M. \& Walsh, B. 1998. \emph{Genetics and analysis of quantitative
traits}. Sinauer Associates, Sunderland.

Mantel, N. 1967. The detection of disease clustering and a generalized
regression approach. \emph{Cancer Res} \textbf{27}: 209--220.

Marcucio, R.S., Cordero, D.R., Hu, D. \& Helms, J.A. 2005. Molecular
interactions coordinating the development of the forebrain and face.
\emph{Developmental Biology} \textbf{284}: 48--61.

Marroig, G. \& Cheverud, J.M. 2001. A comparison of phenotypic variation
and covariation patterns and the role of phylogeny, ecology, and
ontogeny during cranial evolution of new world monkeys. \emph{Evolution}
\textbf{55}: 2576--2600.

Marroig, G. \& Cheverud, J.M. 2005. Size as a line of least evolutionary
resistance: Diet and adaptive morphological radiation in new world
monkeys. \emph{Evolution} \textbf{59}: 1128--1142.

Marroig, G. \& Cheverud, J.M. 2010. Size as a line of least resistance
II: direct selection on size or correlated response due to constraints?
\emph{Evolution} \textbf{64}: 1470--1488.

Martínez-Abadías, N., Esparza, M., vold, T. Sjø, González-José, R.,
Hernández, M. \& Klingenberg, C.P. 2011. Pervasive genetic integration
directs the evolution of human skull shape. \emph{Evolution}
\textbf{66}: 1010--1023.

Márquez, E.J., Cabeen, R., Woods, R.P. \& Houle, D. 2012. The
Measurement of Local Variation in Shape. \emph{Evolutionary Biology}
\textbf{39}: 419--439.

Meinhardt, H. 1983. A boundary model for pattern formation in vertebrate
limbs. \emph{Journal of Embryology and Experimental Morphology}
\textbf{76}: 115--137.

Meinhardt, H. 2008. Models of biological pattern formation: from
elementary steps to the organization of embryonic axes. \emph{Current
topics in developmental biology} \textbf{81}: 1--63.

Melo, D. \& Marroig, G. 2015. Directional selection can drive the
evolution of modularity in complex traits. \emph{Proceedings of the
National Academy of Sciences} \textbf{112}: 470--475.

Minelli, A. 2011. A principle of developmental inertia.
\emph{Epigenetics: Linking Genotype and Phenotype in Development and
Evolution} 116--133.

Mitteroecker, P. \& Bookstein, F.L. 2007. The conceptual and statistical
relationship between modularity and morphological integration.
\emph{Systematic Biology} \textbf{56}: 818--836.

Mitteroecker, P., Gunz, P., Bernhard, M., Bookstein, F.L. \& Schaefer,
K. 2004. Comparison of cranial ontogenetic trajectories among great apes
and humans. \emph{Journal of Human Evolution} \textbf{46}: 679--697.

Monteiro, L.R. \& Nogueira, M.R. 2010. Adaptive radiations, ecological
specialization, and the evolutionary integration of complex
morphological structures. \emph{Evolution} \textbf{64}: 724--744.

Monteiro, L.R., Bonato, V. \& Reis, S.F. 2005. Evolutionary integration
and morphological diversification in complex morphological structures:
mandible shape divergence in spiny rats (Rodentia, Echimyidae).
\emph{Evolution \& Development} \textbf{7}: 429--439.

Newman, M.E.J. 2006. Modularity and community structure in networks.
\emph{Proceedings of the National Academy of Sciences} \textbf{103}:
8577--8582.

Neyman, J. \& Scott, E.L. 1948. Consistent Estimates Based on Partially
Consistent Observations. \emph{Econometrica} \textbf{16}: 1--32.

Oliveira, F.B., Porto, A. \& Marroig, G. 2009. Covariance structure in
the skull of Catarrhini: a case of pattern stasis and magnitude
evolution. \emph{Journal of Human Evolution} \textbf{56}: 417--430.

Olson, E. \& Miller, R. 1958. \emph{Morphological integration}.
University of Chicago Press, Chicago.

Pearson, K. \& Davin, A.G. 1924. On the Biometric Constants of the Human
Skull. \emph{Biometrika} \textbf{16}: 328--363.

Perez, S.I., Aguiar, M.A.M., Guimarães Jr., P.R. \& Reis, S.F. dos.
2009. Searching for Modular Structure in Complex Phenotypes: Inferences
from Network Theory. \emph{Evolutionary Biology}, doi:
\href{http://dx.doi.org/10.1007/s11692-009-9074-7}{10.1007/s11692-009-9074-7}.

Pélabon, C., Bolstad, G.H., Egset, C.K., Cheverud, J.M., Pavlicev, M. \&
Rosenqvist, G. 2013. On the Relationship between Ontogenetic and Static
Allometry. \emph{The American Naturalist} \textbf{181}: 195--212.

Polly, P.D. 2008. Developmental Dynamics and G-Matrices: Can
Morphometric Spaces be Used to Model Phenotypic Evolution?
\emph{Evolutionary Biology} \textbf{35}: 83--96.

Porto, A., Oliveira, F.B., Shirai, L.T., Conto, V. de \& Marroig, G.
2009. The evolution of modularity in the mammalian skull I:
morphological integration patterns and magnitudes. \emph{Evolutionary
Biology} \textbf{36}: 118--135.

Porto, A., Shirai, L.T., Oliveira, F.B. de \& Marroig, G. 2013. Size
Variation, Growth Strategies, and the Evolution of Modularity in the
Mammalian Skull. \emph{Evolution} \textbf{67}: 3305--3322.

R Core Team. 2015. \emph{R: A Language and Environment for Statistical
Computing}. R Foundation for Statistical Computing, Vienna, Austria.

Ravasz, E., Somera, A.L., Mongru, D.A., Oltvai, Z.N. \& Barabási, A.L.
2002. Hierarchical organization of modularity in metabolic networks.
\emph{Science} \textbf{297}: 1551--1555.

Rice, D.P.C., Rice, R. \& Thesleff, I. 2003. Molecular mechanisms in
calvarial bone and suture development, and their relation to
craniosynostosis. \emph{The European Journal of Orthodontics}
\textbf{25}: 139--148.

Rohlf, F.J. \& Slice, D. 1990. Extensions of the Procrustes Method for
the Optimal Superimposition of Landmarks. \emph{Systematic Biology}
\textbf{39}: 40--59.

Rueffler, C., Hermisson, J. \& Wagner, G.P. 2012. Evolution of
functional specialization and division of labor. \emph{Proceedings of
the National Academy of Sciences} \textbf{109}: E326--E335.

Sanger, T.J., Mahler, D.L., Abzhanov, A. \& Losos, J.B. 2012. Roles for
modularity and constraint in the evolution of cranial diversity among
Anolis lizards. \emph{Evolution} \textbf{66}: 1525--42.

Schluter, D. 1996. Adaptive radiation along genetic lines of least
resistance. \emph{Evolution} \textbf{50}: 1766--1774.

Scott, J.H. 1958. The cranial base. \emph{American Journal of Physical
Anthropology} \textbf{16}: 319--348.

Shirai, L.T. \& Marroig, G. 2010. Skull modularity in neotropical
marsupials and monkeys: size variation and evolutionary constraint and
flexibility. \emph{Journal of experimental zoology. Part B, Molecular
and developmental evolution} \textbf{314B}: 663--683.

Theobald, D.L. \& Wuttke, D.S. 2006. Empirical Bayes hierarchical models
for regularizing maximum likelihood estimation in the matrix Gaussian
Procrustes problem. \emph{Proceedings of the National Academy of
Sciences} \textbf{103}: 18521--18527.

Tiedemann, H.B., Schneltzer, E., Zeiser, S., Hoesel, B., Beckers, J. \&
Przemeck, G.K.H.\emph{et al.} 2012. From dynamic expression patterns to
boundary formation in the presomitic mesoderm. \emph{PLoS computational
biology} \textbf{8}: e1002586.

Turing, A.M. 1952. The Chemical Basis of Morphogenesis.
\emph{Philosophical Transactions of the Royal Society of London}
\textbf{237}: 37--72.

Wagner, G.P. 1996. Homologues, natural kinds and the evolution of
modularity. \emph{The American Zoologist} \textbf{36}: 36--43.

Wagner, G.P. 1984. On the eigenvalue distribution of genetic and
phenotypic dispersion matrices: evidence for a nonrandom organization of
quantitative character variation. \emph{Journal of Mathematical Biology}
\textbf{21}: 77--95.

Wagner, G.P. 2010. The Measurement Theory of Fitness. \emph{Evolution}
\textbf{64}: 1358--1376.

Wagner, G.P. \& Altenberg, L. 1996. Perspective: complex adaptations and
the evolution of evolvability. \emph{Evolution} \textbf{50}: 967--976.

Wagner, G.P., Pavlicev, M. \& Cheverud, J.M. 2007. The road to
modularity. \emph{Nature reviews. Genetics} \textbf{8}: 921--931.

Walker, J.A. 2000. Ability of Geometric Morphometric Methods to Estimate
a Known Covariance Matrix. \emph{Systematic Biology} \textbf{49}:
686--696.

Willmore, K.E., Roseman, C.C., Rogers, J., Cheverud, J.M. \&
Richtsmeier, J.T. 2009. Comparison of Mandibular Phenotypic and Genetic
Integration between Baboon and Mouse. \emph{Evolutionary Biology}
\textbf{36}: 19--36.

Woods, R.P. 2003. Characterizing volume and surface deformations in an
atlas framework: theory, applications, and implementation.
\emph{NeuroImage} \textbf{18}: 769--788.

Young, N.M. \& Hallgrímsson, B. 2005. Serial homology and the evolution
of mammalian limb covariation structure. \emph{Evolution} \textbf{59}:
2691--2704.

Young, N.M., Wagner, G.P. \& Hallgrímsson, B. 2010. Development and the
evolvability of human limbs. \emph{Proceedings of the National Academy
of Sciences} \textbf{107}: 3400--3405.

Zelditch, M.L. \& Carmichael, A.C. 1989. Ontogenetic variation in
patterns of developmental and functional integration in skulls of
Sigmodon fulviventer. \emph{Evolution} \textbf{43}: 814--824.

Zelditch, M.L. \& Swiderski, D.L. 2011. Epigenetic interactions: the
developmental route to functional integration. In: \emph{Epigenetics:
linking genotype and phenotype in development and evolution}, pp.
290--316.

Zelditch, M.L., Swiderski, D.L., Sheets, H.D. \& Fink, W.L. 2004.
\emph{Geometric Morphometrics for Biologists: A Primer}, 1st ed.
Elsevier.

Zelditch, M.L., Wood, A.R. \& Swiderski, D.L. 2009. Building
Developmental Integration into Functional Systems: Function-Induced
Integration of Mandibular Shape. \emph{Evolutionary Biology}
\textbf{36}: 71--87.

\end{document}
